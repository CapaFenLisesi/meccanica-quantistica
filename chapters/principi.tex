\chapter{I princ\`ipi della meccanica quantistica}
Vediamo in questo capitolo di porre le fondamenta di questa nuova teoria, inquadrando in una struttura matematica quello che avevamo già visto nell'introduzione, come gli stati, le osservabili e il loro comportamento.
Abbiamo innanzitutto visto la necessità di descrivere gli stati in uno spazio vettoriale, in modo che valga il \emph{principio di sovrapposizione}, e questo stato dovrà essere sul campo complesso per dar luogo a fenomeni di interferenza come nella doppia fenditura.
Lo spazio degli stati, che per ora indichiamo con $\hilbert$, avrà le seguenti proprietà:
\begin{itemize}
	\item è uno spazio vettoriale complesso;
	\item in esso è definito un prodotto scalare definito positivo;
	\item è \emph{completo}, come spazio metrico rispetto alla distanza definita dal prodotto scalare;
	\item è \emph{separabile}, ossia esiste un suo sottoinsieme denso e numerabile;\footnote{Un esempio ``classico'' di spazio separabile è semplicemente $\R$, in cui troviamo $\Q\subset\R$ che è denso e numerabile.} 
\end{itemize}
La completezza implica che tutte le successioni di Cauchy in $\hilbert$ sono anche convergenti, mentre la separabilità assicura l'esistenza di una base ortonormale che abbia cardinalità al più numerabile.
Lo spazio degli stati assuma quindi la struttura di uno \emph{spazio di Hilbert} complesso separabile.

Nella notazione introdotta da Dirac, i vettori di $\hilbert$ sono indicati come \emph{ket}, scritti come $\ket{\cdot}$.
Dobbiamo notare innanzitutto che la corrispondenza tra gli stati fisici e i vettori di questo spazio \emph{non è biunivoca}: mentre un vettore $\ket{x}\in\hilbert$ rappresenta uno e un solo stato fisico, lo stesso stato fisico è rappresentato da $\ket{x}$ e tutti i suoi multipli $\alpha\ket{x}$ per qualsiasi $\alpha\in\C$.
Perciò tutte le informazioni che ricaveremo dal modello quantistico che costruiamo dovranno essere indipendenti da questa arbitrarietà.
Se anzich\'e guardare ai vettori però guardiamo alla classe di vettori che sono tutti multipli (per uno scalare), cioè ai \emph{raggi} dello spazio, allora la corrispondenza diventa s\`i biunivoca.
I ket, quindi, fanno parte di uno spazio vettoriale complesso, dunque ogni combinazione lineare $\alpha\ket{x}+\beta\ket{y}$ è ancora in $\hilbert$.
Il prodotto scalare tra due \emph{ket} è indicato come $\braket{x}{y}$: questa notazione affianca il ket, a destra, con l'elemento a sinistra detto \emph{bra}.
La notazione differente dei due elementi suggerisce una qualche differenza, e in effetti i \emph{ket} e i bra, seppur molto simili, appartengono a due spazi diversi; il prodotto scalare tra due \emph{ket} diventa quindi l'operazione del \emph{bra} associato al primo \emph{ket} sul secondo.
Se i \emph{ket} sono dei vettori di $\hilbert$, possiamo pensare ai \emph{bra} come agli elementi del duale, cioè a funzionali lineari sui ket.
Diamo quindi le proprietà del prodotto scalare cos\`i definito:
\begin{itemize}
	\item è antilineare, ossia $\braket{x}{y}=\braket{y}{x}^*$;
	\item è lineare nella prima variabile, $\braket{x}{\alpha y_1+\beta y_2}=\alpha\braket{x}{y_1}+\beta\braket{x}{y_2}$;
	\item è definito positivo, $\braket{x}{x}\geq 0$ per ogni $\ket{x}\in\hilbert$.
\end{itemize}
Dalle prime due troviamo che è antilineare nella seconda variabile: se vogliamo scomporre il prodotto scalare, dovremo introdurre il coniugato dei componenti, ossia $\braket{\alpha x_1+\beta x_2}{y}=\alpha^*\braket{x_1}{y}+\beta^*\braket{x_2}{y}$.
Dalla terza notiamo infine che il prodotto scalare di un \emph{ket} con se stesso è sempre un numero reale.
Vediamo come costruire questo prodotto scalare: se prendiamo una base ortonormale $\{\ket{e_i}\}$ (quindi $\braket{e_i}{e_j}=\delta_{ij}$), che per semplicità supponiamo finita di cardinalità $n$, allora possiamo esprimere un vettore $\ket{x}$ nei termini di questa base come
\begin{equation}
	\ket{x}=\sum_{i=1}^nx_i\ket{e_i}=x_i\ket{e_i}.
\end{equation}
Analogamente si dà l'espressione di $\ket{y}$, mentre il \emph{bra} associato ad esso è il trasposto coniugato del ket, ossia
\begin{equation}
	\bra{y}=\adj{\ket{y}}=y_i^*\bra{e_i}
\end{equation}
dunque il prodotto scalare tra i due è dato da
\begin{equation}
	\braket{y}{x}=y_j^*x_k\braket{e_j}{e_k}=y_j^*x_k\delta_{jk}=y_j^*x_j,
\end{equation}
che si può anche scrivere come $(y_jx_j^*)^*=\braket{x}{y}^*$, verificando quindi la prima proprietà.

\section{Autovalori e autostati}
Dopo aver inquadrato gli stati del sistema nello spazio di Hilbert con le dovute proprietà, passiamo a studiare le osservabili.
Da un punto di vista prettamente pratico e non matematico possiamo definirle come degli apparecchi che agiscono sul sistema come delle \emph{scatole nere}, di cui non possiamo indagare il funzionamento; successivamente vedremo un modo di definirle anche matematicamente.
Possiamo preoccuparci ora dell'\emph{effetto} che esse producono sul sistema: ad ogni osservabile è associata, chiaramente, una grandezza fisica del sistema, come l'energia, la posizione, eccetera, quindi agendo con un'osservabile sul sistema otteniamo un numero.
Su \emph{come} sia fatto questo numero, però, c'è da discutere.
Sappiamo bene che uno strumento di misura reale ci fornisce una misura con un numero di cifre inevitabilmente finito, quindi questa misura è un numero \emph{razionale}, e il numero di cifre è collegato alla precisione dello strumento.
Possiamo immaginare di aumentare sempre di più questa precisione della misura, ad esempio utilizzando strumenti sempre più sofisticati: nella meccanica classica non si afferma l'esistenza di un limite da porre a questa precisione.
Poich\'e il limite di una successione di numeri razionali è, in generale, un numero reale (dato che $\Q$ non è completo), se aumentiamo indefinitamente la precisione degli strumenti la misura, pur essendo un numero razionale, convergerà ad un numero reale; in effetti la meccanica classica studia i sistemi fisici tramite i numeri reali senza preoccuparsi di questo fatti.

In meccanica quantistica, assumeremo che le osservabili, quindi i nostri strumenti di misura, restituiscano soltanto valori discreti (ma non c'è da preoccuparsi: lo spaziotempo è ancora considerato uno spazio continuo).
Miglioriamo quindi la definizione precedente: diremo che $\xi$ è un'osservabile se è uno ``strumento di misura'' i cui risultati formano un insieme discreto $\{\xi_1,\xi_2,\dots,\xi_n,\dots\}$ (anche infinito).
I valori $\xi_i$ sono detti \emph{autovalori} dell'osservabile $\xi$, e costituiscono quindi i \emph{possibili} risultati della misura.

Torniamo all'esperimento con la lastra polaroid: l'osservabile è la lastra, e il risultato della misura è che il fotone ``passa'' o ``non passa'', e questi due risultati sono quindi gli autovalori.
Associamo al passaggio del fotone l'autovalore 1 e al non passaggio l'autovalore 0.
Non possiamo pretendere di predire il risultato della misura, in generale: sappiamo però che se il fotone è polarizzato linearmente lungo l'asse della lastra, sicuramente passerà, quindi in tale caso avremo con certezza l'autovalore 1.
Analogamente se sappiamo che il fotone è polarizzato ortogonalmente all'asse della lastra, certamente avremo l'autovalore 0.

Più in generale, esistono degli stati del sistema con questa caratteristica: chiamiamo \emph{autostato} un particolare stato del sistema in cui il risultato della misura è certo.
Identificando lo stato con il corrispondente raggio in $\hilbert$, indicheremo con $\ket{\xi_i}$ tale raggio.
Per la corrispondenza biunivoca tra i due elementi, però, spesso chiameremo (impropriamente) stati i raggi di $\hilbert$, per semplificare un poco la notazione.
Dunque indicheremo con $\ket{\xi_i}$ l'autostato dell'osservabile $\xi$ relativo all'autovalore $\xi_i$.
Misurando $\xi$ mentre il sistema è nello stato $\ket{\xi_i}$ siamo certi che otterremo l'autovalore $\xi_i$.
Una volta compiuta la misura, inoltre, abbiamo rimosso ogni incertezza sullo stato attuale del sistema, perch\'e ora lo conosciamo, perciò continuando a misurare la medesima osservabile è chiaro che (almeno negli istanti successivi) otterremo sempre lo stesso risultato: possiamo affermare dunque che se effettuiamo la misura di $\xi$ sul sistema e otteniamo l'autovalore $\xi_i$, allora dopo la misura il sistema si troverà nello stato $\ket{\xi_i}$, che è l'autostato associato all'autovalore.
In generale quindi non ci è dato sapere lo stato del sistema \emph{prima} della misura, ma sapremo in che stato si trova \emph{dopo} averla compiuta: questo è essenzialmente il motivo per cui, compiendo una misura, perturbiamo il sistema.

Prendiamo stavolta due lastre polaroid, allineate lungo lo stesso asse di polarizzazione, e prendiamo gli autovalori 0 e 1 come prima.
Se un fotone passa per le due lastre, è come se misurassimo due volte la sua polarizzazione, quindi rappresentano la stessa osservabile misurata una dopo l'altra.
All'inizio, non sappiamo lo stato del fotone: quando esso passa per la prima lastra, però, può o passare o non passare.
Se passa, cioè se otteniamo 1, il fotone risulta polarizzato lungo l'asse (chiamiamo questo stato $\ket{1}$), quindi certamente passerà anche per la seconda lastra.
Quindi, ricapitolando, misurando il fotone otteniamo 1, dunque siamo sicuri che lo stato del fotone sia $\ket{1}$, che allora passerà anche per la seconda lastra, perch\'e si trova nell'autostato associato all'autovalore 1.
Il discorso è analogo se il fotone non passa attraverso la prima lastra, anche se è una questione un po' delicata perch\'e per la seconda lastra non esiste nemmeno più il fotone da misurare; questo esperimento serve almeno per fissare le idee su questi concetti di base.

Non sempre l'autostato di un autovalore è unico, ma possono esisterne anche di più, anche infiniti: in tal caso, la conoscenza dell'autovalore non determina ancora completamente lo stato del sistema.
Distinguiamo quindi due tipi di osservabili:
\begin{itemize}
	\item quelle \emph{degeneri}, per cui eseguendo la misura non si conosce complessivamente lo stato del sistema (lo è la lastra polaroid: sappiamo la polarizzazione, ma non possiamo conoscere con questo esperiento la frequenza dei fotoni);
	\item quelle \emph{non degeneri}, i cui autovalori determinano completamente lo stato del sistema (corrispondenza biunivoca tra autovalori e stati di $\hilbert$).
\end{itemize}

\section{Probabilità di transizione tra gli stati}
Sappiamo che non possiamo predire il risultato di una misura, a meno che il sistema non sia in un certo autostato: negli altri casi possiamo però sapere la probabilità che un certo risultato si presenti rispetto agli altri.
Il risultato dipenderà quindi sia dallo stato iniziale che da quello di arrivo: cerchiamo ora un metodo per calcolare la probabilità di passare da uno stato $\ket{x}$ allo stato $\ket{y}$.
Questo valore è detto \emph{probabilità di transizione} (dal primo stato all'altro) e lo indicheremo con $P(\ket{x}\to\ket{y})$.
Per determinare questa probabilità usiamo il prodotto scalare: per cominciare, ci serve un numero reale, quindi prendiamo il prodotto scalare $\braket{x}{y}$ per il suo coniugato, ottenendo $P(\ket{x}\to\ket{y})=\abs{\braket{x}{y}}^2$.
Questo però non basta, perch\'e potremmo ottenere un numero maggiore di uno, mentre la probabilità deve essere in $[0,1]$: dobbiamo allora normalizzarla, ottenendo
\begin{equation}
	P(\ket{x}\to\ket{y})=\frac{\abs{\braket{x}{y}}^2}{\braket{x}{x}\braket{y}{y}}.
	\label{eq:probabilita-transizione}
\end{equation}
Possiamo trascurare la normalizzazione se assumiamo tutti gli stati normalizzati: questa assunzione è del tutto legittima, poich\'e sappiamo che se $\ket{x}$ rappresenta un certo stato fisico del sistema allora anche $\alpha\ket{x}$ lo rappresenta per qualsiasi $\alpha\in\C$.
Allora possiamo gli stati $\ket{x}$ e $\frac1{\braket{x}{x}}\ket{x}$ sono equivalenti nella rappresentazione, e il secondo è normalizzato.
Per il momento però consideriamo ancora dei vettori generici, non già normalizzati.
La \eqref{eq:probabilita-transizione} è una buona definizione di probabilità: è certamente positiva, perch\'e $\braket{x}{x}$ e $\braket{y}{y}$ sono positivi (in quanto non sono lo zero di $\hilbert$ e il prodotto scalare non è degenere), e per la disuguaglianza di Schwarz $\abs{\braket{x}{y}}^2\leq\braket{x}{x}\braket{y}{y}$ quindi è anche minore o uguale di 1.
Inoltre essa non dipende dal fattore scalare arbitrario nella corrispondenza tra stati fisici e vettori di $\hilbert$: se al posto di $\ket{x}$ e $\ket{y}$ prendiamo $a\ket{x}$ e $b\ket{y}$ per qualunque $a,b\in\C$, abbiamo
\begin{equation}
	P(a\ket{x}\to b\ket{y})=\frac{\abs{a}^2\abs{b}^2\abs{\braket{x}{y}}^2}{\abs{a}^2\braket{x}{x}\abs{y}^2\braket{y}{y}}=\frac{\abs{\braket{x}{y}}^2}{\braket{x}{x}\braket{y}{y}}=P(\ket{x}\to\ket{y}).
\end{equation}
Infine, questa probabilità è commutativa, cioè $P(\ket{x}\to\ket{y})=P(\ket{y}\to\ket{x})$ come si verifica facilmente scambiando i due ket nella \eqref{eq:probabilita-transizione}.

Prendiamo dunque un'osservabile non degenere $\xi$, e un generico stato $\ket{x}$.
Non sappiamo quale sarà il risultato, ma ogni possibile autovalore $\xi_i$ dell'osservabile avrà una certa probabilità $P_i$ di accadere.
Eseguendo una misura di $\xi$, la probabilità di trovare l'autovalore $\xi_i$ equivale alla probabilità di trovare, dopo la misura, il sistema nell'autostato $\ket{\xi_i}$, quindi è la probabilità di transizione tra $\ket{x}$ e questo autostato:
\begin{equation}
	P_i=\frac{\abs{\braket{\xi_i}{x}}^2}{\braket{\xi_i}{\xi_i}\braket{x}{x}}.
\end{equation}
Ovviamente ogni volta che compiamo la misura dobbiamo ottenere uno degli autovalori, qualunque esso sia, quindi $\sum_kP_k=1$ (gli indici $k$ possono anche essere infiniti).

Se partiamo da un autostato $\ket{\xi_i}$, misurando $\xi$ otterremo sempre con certezza l'autovalore $\xi_i$, quindi la probabilità $P(\ket{\xi_i}\to\ket{\xi_i})$ vale 1.
Come conseguenza abbiamo allora $P(\ket{\xi_i}\to\ket{\xi_k})=0$ per $i\neq k$, dato che la somma di tutte le probabilità è 1, ma allora
\begin{equation}
	P(\ket{\xi_i}\to\ket{\xi_k})=\frac{\abs{\braket{\xi_i}{\xi_k}}^2}{\braket{\xi_i}{\xi_i}\braket{\xi_k}{\xi_k}}=0 \iff \braket{\xi_i}{\xi_k}=0.
	\label{eq:autostati-ortogonali}
\end{equation}
Troviamo allora che
\begin{equation}
	\braket{\xi_i}{\xi_k}=\delta_{ik}
	\label{eq:autostati-ortonormali}
\end{equation}
cioè \emph{due autostati relativi a differenti autovalori sono ortonormali}.

\section{Il postulato di Von Neumann}
Abbiamo dimostrato che gli autostati di un'osservabile non degenere formano un insieme ortonormale.
Esso è anche completo: infatti se esistesse un altro stato $\ket{\phi}$ ortogonale a tutti gli autostati $\ket{\xi_i}$, la probabilità di passare da uno qualunque di questi autostati a $\ket{\phi}$ sarebbe nulla, ma allora $\sum_i\abs{\braket{\xi_i}{\phi}}^2=0$ che è assurdo perch\'e deve essere 1.
Quindi gli autostati (o meglio gli autovettori) di un'osservabile non degenere formano una \emph{base ortonormale} di $\hilbert$.
Chiamiamo \emph{autospazio} relativo all'autovalore $\xi_i$ il sottospazio di $\hilbert$ costituito dagli autostati di tale autovalore.
Se l'osservabile non è degenere, ad ogni autovalore corrisponde un unico autostato, quindi l'autospazio corrispondente ha dimensione 1 (ricordiamo che tutti gli autovettori multipli per uno scalare rappresentano il medesimo stato).
Tutti questi autostati sono anche linearmente indipendenti, in quanto ortonormali, quindi gli autospazi di ciascun autovalore sono tra loro ortogonali (e, ancora, linearmente indipendenti).
Allora detto $\hilbert_i$ l'autospazio dell'autovalore $\xi_i$, possiamo scomporre l'intero spazio $\hilbert$ come somma diretta
\begin{equation}
	\hilbert=\bigoplus_i\hilbert_i.
\end{equation}

Se quindi prendiamo una base di autostati $\{\ket{\xi_i}\}$, possiamo rappresentare ogni stato in termini di questi autostati come
\begin{equation}
	\ket{x}=\sum_ic_i\ket{\xi_i}
\end{equation}
dove i coefficienti $c_i\in\C$ devono soddisfare la condizione $\braket{x}{x}=\sum_ic_i^*c_i=\sum_i\abs{c_i}^2<+\infty$.
Per l'ortonormalità della base possiamo quindi individuare tali coefficienti prendendo il prodotto scalare dello stato $\ket{x}$ con gli autostati, ossia
\begin{equation}
	\braket{\xi_k}{x}=\bra{\xi_k}\sum_ic_i\ket{xi_i}=\sum_ic_i\braket{\xi_k}{\xi_i}=\sum_ic_i\delta_{ik}=c_k,
\end{equation}
ma allora
\begin{equation}
	\ket{x}=\sum_i\braket{\xi_i}{x}\ket{\xi_i}=\sum_i\ket{\xi_i}\braket{\xi_i}{x}=\Big(\sum_i\ket{\xi_i}\bra{\xi_i}\Big)\ket{x}
\end{equation}
da cui ricaviamo che l'operatore $\sum_i\ket{\xi_i}\bra{\xi_i}$ è l'identità su $\hilbert$.
In particolare, il singolo addendo $\ket{\xi_i}\bra{\xi_i}$ è un operatore di \emph{proiezione}, in quanto applicandolo due volte si ha
\begin{equation}
	\ket{\xi_i}\bra{\xi_i}\ket{\xi_i}\bra{\xi_i}=\ket{\xi_i}\braket{\xi_i}{\xi_i}\bra{\xi_i}=\ket{\xi_i}\bra{\xi_i}.
\end{equation}
Inoltre applicandolo ad un vettore si ottiene la sua componente nella direzione di $\ket{\xi_i}$, vale a dire la proiezione di $\ket{x}$ nell'autospazio $\hilbert_i$.
Chiaramente sommando le proiezioni su tutti gli autospazi si ottiene il vettore di partenza, dunque si ha l'operatore identità: questa condizione indica la completezza della base degli autostati.

I discorsi fatti finora valgono solo nel caso di osservabili non degeneri: nell'altro caso, sappiamo che ad un autovalore corrispondono più autostati, che chiaramente dovranno essere linearmente indipendenti (altrimenti rappresenterebbero lo stesso stato fisico).
In questo caso non possiamo sapere in quale autostato si trovi il sistema, ma come al solito possiamo conoscere la probabilità che il sistema si trovi in uno o in un altro.
Per semplicità, consideriamo il caso in cui esistano soltanto di autostati linearmente indipendenti relativi all'autovalore $\xi_1$ di $\xi$: chiamiamo tali autostati $\ket{\xi_1'}$ e $\ket{\xi_1''}$, e assumiamo che gli autovalori $\xi_k$ per $k>1$ non siano degeneri.
Date queste ipotesi allora l'autospazio $\hilbert_1$ ha dimensione 2, ed è chiaramente ortogonale a tutti gli $\hilbert_k$ con $k>1$, quindi anche in questo caso tutti gli $\hilbert_i$ sono linearmente indipendenti e si può scrivere $\hilbert=\bigoplus_i\hilbert_i$.
Possiamo comunque scegliere due vettori linearmente indipendenti da $\hilbert_1$ che insieme agli altri $\ket{\xi_k}$ (sempre $k>1$) formano ancora una base di $\hilbert$; la differenza con le osservabili non degeneri è che ora la base non è più unica, anche a meno di normalizzazioni dei vettori, questo perch\'e in $\hilbert_1$ esistono infinite coppie di vettori linearmente indipendenti anche normalizzati.\footnote{Un caso più ``quotidiano'' analogo a questo lo troviamo nelle basi di $\R$ e di $\R^2$: se consideriamo i vettori a meno di fattori scalari, allora in $\R$ la base può essere una sola (è l'unità) mentre in $\R^2$ ne troviamo infinite, ottenute ruotando la base canonica di un angolo qualsiasi.}

Ad ogni modo, possiamo rappresentare anche in questo caso uno stato $\ket{x}$ in termini della base scelta: abbiamo $\ket{x}=c_1'\ket{\xi_1'}+c_1''\ket{\xi_1''}+\sum_{k>1}c_k\ket{\xi_k}$.
Se abbiamo scelto una base ortonormale, avremo anche in questo caso partendo dallo stato $\ket{x}$ che assumiamo normalizzato, cos\`i come gli autostati, che
\begin{equation}
	\sum_iP_i=\sum_i\abs{\braket{\xi_i}{x}}^2=\sum_i\abs{c_i}^2=\abs{c_1'}^2+\abs{c_1''}^2+\sum_{k>1}\abs{c_k}^2=1,
\end{equation}
dunque per esclusione la probabilità di passare da $\ket{x}$ ad uno stato in $\hilbert_1$ è data da $P_1=1-\sum_{k>1}P_k=1-\sum_{k>1}\abs{c_k}^2=\abs{c_1'}^2+\abs{c_1''}^2$.
Questo discorso è quindi riassunto nel \emph{postulato di Von Neumann}, che afferma
\begin{quote}
	Se una misura di $\xi$ nello stato $\ket{x}$ risulta nell'autovalore $\xi_i$, allora lo stato dopo al misura è dato dalla proiezione di $\ket{x}$ nell'autospazio di $\xi_i$.
\end{quote}
Nel caso precedente tale proiezione è proprio $c_1'\ket{\xi_1'}+c_1''\ket{\xi_1''}$, quindi la probabilità di trovare lo stato nell'autospazio $\hilbert_1$ è $\abs{c_1'}^2+\abs{c_1''}^2$.
Il postulato si può anche riformulare affermando che dopo la misura lo stato del sistema è una combinazione lineare degli autostati relativi all'autovalore trovato: la differenza tra i due enunciati è che nel secondo, per avere gli autostati, dobbiamo specificare una base.
La probabilità di transizione, dunque, non dipende dalla base scelta per $\hilbert$ (come era lecito aspettarsi).

\section{Osservabili e operatori}
Chiamiamo $E_i$ l'operatore dato da $\ket{\xi_i}\bra{\xi_i}$, con $\xi_i$ autovalore di un'osservabile $\xi$ generica.
Per le proprietà viste in precedenza, $E_i^2=E_i$ perch\'e è una proiezione, inoltre ogni stato si può scrivere come $\ket{x}=\sum_i\ket{\xi_i}\braket{\xi_i}{x}=\sum_iE_i\ket{x}$, se scegliamo gli autostati $\ket{\xi_i}$ in modo da avere una base ortonormale.
Assumiamo $\ket{x}$ e gli autostati della base normalizzati: allora se l'osservabile non è degenere la probabilità $P_i$ di trovare l'autovalore $\xi_i$ è la probabilità di transizione da $\ket{x}$ a $\ket{\xi_i}$, data da
\begin{equation}
	P_i=\abs{\braket{x}{\xi_i}}^2=\braket{x}{\xi_i}\braket{x}{\xi_i}^*=\braket{x}{\xi_i}\braket{\xi_i}{x}=\bra{x}E_i\ket{x}.
	\label{eq:probabilita-proiezione-non-degenere}
\end{equation}
Anche nel caso l'osservabile non sia degenere, il risultato non cambia, se prendiamo comunque la proiezione su ogni vettore della base: per linearità, la proiezione di $\ket{x}$ nell'autospazio è la somma delle proiezioni su ciascun vettore della base che appartiene a tale autospazio.
Più praticamente, riprendendo l'autovalore $\xi_1$ visto in precedenza, e assumendo che esso abbia due autovettori linearmente indipendenti $\ket{\xi_1'}$ e $\ket{\xi_1''}$ ortonormali, allora
\begin{equation}
	P_1=\abs{\braket{x}{\xi_1'}}^2+\abs{\braket{x}{\xi_1''}}^2=\bra{x}E_1'\ket{x}+\bra{x}E_1''\ket{x}=\bra{x}(E_1'+E_1'')\ket{x}.
	\label{eq:probabilita-proiezione-degenere}
\end{equation}

Se compiamo un numero $N$ di misure di $\xi$ sullo stato $\ket{x}$ (ovviamente non possiamo farlo consecutivamente: dobbiamo avere $N$ sistemi tutti nello stesso stato e compiere la misura si ciascuno di essi), misureremo ogni autovalore $\xi_i$ un certo numero $n_i$ di volte, con $\sum_in_i=N$.
Possiamo allora definire il valore medio $\avg{\xi}$ dell'osservabile come
\begin{equation}
	\avg{\xi}=\frac1{N}\sum_in_i\xi_i
	\label{eq:valore-medio-finito}
\end{equation}
e per $N\to+\infty$, cioè con un numero infinito di misure, il rapporto $n_i/N$ tende alla probabilità di trovare $\xi_i$, perciò
\begin{equation}
	\avg{\xi}=\sum_iP_i\xi_i
	\label{eq:valore-medio}
\end{equation}
dove le $P_i$ sono definite come al solito finora.
Se il sistema è nello stato $\ket{x}$, dunque, il valore medio è
\begin{equation}
	\avg{\xi}=\sum_i\xi_i\bra{x}E_i\ket{x}=\bra{x}\sum_iE_i\xi_i\ket{x}.
\end{equation}
Definiamo dunque l'\emph{operatore associato} all'osservabile $\xi$ come
\begin{equation}
	\op{\xi}=\sum_i\xi_iE_i.
	\label{eq:definizione-operatore}
\end{equation}
Si ha dunque, se $\ket{x}$ è normalizzato, che $\avg{\xi}=\bra{x}\op{\xi}\ket{x}$, il quale viene infatti chiamato \emph{valore di aspettazione} dell'osservabile $\xi$ sullo stato $\ket{x}$.

Osserviamo subito che gli operatori associati sono hermitiani, ossia $\adj{\op\xi}=\op\xi$: infatti data una base ortonormale di autostati anche degeneri $\{\ket{\xi_i}\}$ si ha
\begin{equation}
	\bra{x}\op{\xi}\ket{y}=\bra{x}\sum_i\xi_iE_i\ket{y}=\sum_i\xi_i\bra{x}E_i\ket{y}=\sum_i\xi_i\braket{x}{\xi_i}\braket{\xi_i}{y},
\end{equation}
e allo stesso tempo
\begin{equation}
	\bra{x}\adj{\op{\xi}}\ket{y}=\bra{y}\op{\xi}\ket{x}^*=\Big(\sum_i\xi_i\braket{y}{\xi_i}\braket{\xi_i}{x}\Big)^*=\sum_i\xi_i\big(\braket{y}{\xi_i}\braket{\xi_i}{x}\big)^*=\sum_i\xi_i\braket{x}{\xi_i}\braket{\xi_i}{y}
\end{equation}
($\sum_i\xi_i\in\R$ quindi è il coniugato di s\'e stesso) che è dunque uguale a $\bra{x}\op{\xi}\ket{y}$.
Inoltre gli autovalori $\xi_i$, che avevamo definito come i possibili risultati della misura dell'osservabile, sono in realtà anche gli autovalori, nel senso matematico, dell'operatore, perch\'e
\begin{equation}
	\op{\xi}\ket{\xi_i}=\sum_k\xi_kE_k\ket{\xi_i}=\sum_k\xi_k\ket{\xi_k}\braket{\xi_k}{\xi_i}=\sum_k\xi_k\ket{\xi_k}\delta_{ik}=\xi_i\ket{\xi_i}.
\end{equation}

Vediamo due importanti proprietà degli operatori hermitiani:
\begin{itemize}
	\item i suoi autovalori sono tutti reali.
		Se infatti $\op\xi\ket{\xi}=\xi\ket{\xi}$, in generale l'autovalore $\xi$ è complesso, ma in questo caso se moltiplichiamo a sinistra per il \emph{bra} $\bra{\xi}$ otteniamo $\bra{\xi}\op\xi\ket{\xi}=\xi\braket{\xi}{\xi}$, ma $\op\xi=\adj{\op\xi}$ quindi $\bra{\xi}\op\xi\ket{\xi}=\bra{\xi}\adj{\op\xi}\ket{\xi}=\bra{\xi}\op\xi\ket{\xi}^*=(\xi\braket{\xi}{\xi})^*$, ossia
		\begin{equation}
			(\xi\braket{\xi}{\xi})^*=\xi\braket{\xi}{\xi}\quad\then\quad\xi^*=\xi
		\end{equation}
		perch\'e $\braket{\xi}{\xi}\in\R$, ma allora $\xi\in\R$.
	\item gli autospazi corrispondenti a differenti autovalori sono ortogonali.
		Prendiamo due autovalori $\eta'$ e $\eta''$ (ovviamente $\eta'\neq\eta''$) dell'operatore $\op\eta$ hermitiano.
		Prendiamo inoltre l'azione di $\op\eta$ su due autostati corrispondenti ai due autovalori, cioè $\op\eta\ket{\eta'}=\eta'\ket{\eta'}$ e $\op\eta=\eta''\ket{\eta''}$.
		Moltiplicando a sinistra per rispettivamente per $\bra{\eta''}$ e $\bra{\eta'}$ otteniamo
		\begin{equation}
			\begin{cases}
				\bra{\eta''}\op\eta\ket{\eta'}=\eta'\braket{\eta''}{\eta'}\\
				\bra{\eta'}\op\eta\ket{\eta''}=\eta''\braket{\eta'}{\eta''}
			\end{cases}
			\quad\then\quad
			\begin{cases}
				\bra{\eta''}\op\eta\ket{\eta'}=\eta'\braket{\eta''}{\eta'}\\
				\bra{\eta''}\op\eta\ket{\eta'}=\eta''^*\braket{\eta''}{\eta'}
			\end{cases}
		\end{equation}
		ma gli autovalori sono reali quindi nella seconda $\eta''^*=\eta''$.
		Allora sottraendo la seconda equazione dalla prima troviamo
		\begin{equation}
			(\eta'-\eta'')\braket{\eta''}{\eta'}=0
		\end{equation}
		ma se $\eta'\neq\eta''$ allora $\braket{\eta''}{\eta'}=0$, ossia i due autostati sono ortogonali.
\end{itemize}

È dunque vero che tutte le osservabili sono operatori hermitiani?
Per quanto visto finora e per come abbiamo associato gli operatori alle osservabili, s\`i, è vero.
Non è vero però il contrario: sappiamo che con gli autostati di un'osservabile, degenere o meno, possiamo costruire una base ortonormale dello spazio degli stati, ma ciò non è sempre possibile con gli operatori hermitiani.
Se lo spazio $\hilbert$ fosse di dimensione finita, allora per il teorema spettrale ogni operatore hermitiano ammette un insieme di autovettori che forma una base dello spazio, ma questo non è più vero in dimensione infinita.
Alcuni operatori infatti, pur essendo hermitiani, non possiedono abbastanza autovettori linearmente indipendenti da poter costruire una base: gli operatori per cui questo accade, invece, si dicono \emph{autoaggiunti}.
Solo quest'ultimo tipo di operatori corrisponde ad un'osservabile; analogamente tutte le osservabili sono operatori autoaggiunti.
