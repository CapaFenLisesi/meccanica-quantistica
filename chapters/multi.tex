\chapter{Sistemi con più gradi di libertà}
Salvo brevi eccezioni, finora abbiamo trattato solamente dei sistemi ad un grado di libertà, tipicamente una particella in moto lungo una retta.
Passiamo ora a studiare sistemi con più gradi di libertà, che possono corrispondere ad un maggior numero di particelle, all'aggiunta di altre dimensioni spaziali, o entrambe.
In questo capitolo ci occuperemo di introdurre lo schema matematico in cui inquadrare questi problemi, in via generale, per poterlo poi applicare a problemi più complessi di quelli già visti, come le rotazioni e gli atomi.

\section{Prodotto diretto di spazi vettoriali}
Il problema principale è capire come costruire lo spazio degli stati di tali sistemi.
Sappiamo che con un grado di libertà possiamo associare ad ogni stato un raggio in uno spazio di Hilbert complesso, separabile e infinito dimensionale.
Quando il sistema possiede più di una dimensione spaziale, tali dimensioni sono indipendenti e apportano ciascuna un grado di libertà; per ciascuna di esse abbiamo, ad esempio, un operatore della posizione e uno dell'impulso, e sappiamo già dai commutatori fondamentali che posizione e impulso di coordinate differenti commutano.
Analogamente, componendo più particelle in un sistema ciascuna di esse ha associata la coppia di operatori posizione e impulso (ad esempio) e tutti quelli che ne derivano.
/bin/bash: :wq: command not found
\begin{equation}
	\hilbert=\hilbert_1\otimes\hilbert_2\otimes\cdots\otimes\hilbert_d.
\end{equation}

Per studiare la definizione e le proprietà di base ci restringiamo ora al prodotto di due spazi $\hilbert_1$ e $\hilbert_2$.
Dati gli stati (qualsiasi) $\ket{A}\in\hilbert_1$ e $\ket{B}\in\hilbert_2$, possiamo costruire il loro prodotto diretto $\ket{A}\otimes\ket{B}$ che rappresenta lo stato del sistema combinato $\hilbert=\hilbert_1\otimes\hilbert_2$ in cui il sottosistema $\hilbert_1$ è nello stato $\ket{A}$ e il sottosistema $\hilbert_2$ è nello stato $\ket{B}$.
Chiaramente il prodotto non è commutativo, perch\'e in generale non è detto nemmeno che esista uno stato $\ket{A}$ in $\hilbert_2$ in modo da poter scrivere $\ket{B}\otimes\ket{A}$.
Lo spazio vettoriale $\hilbert_1\otimes\hilbert_2$ è dotato in modo naturale delle operazioni di somma e di moltiplicazione per scalare (un numero complesso) esattamente come gli spazi di partenza: esso contiene le coppie $\ket{A}\otimes\ket{B}$ per \emph{ogni} stato $\ket{A}\in\hilbert_1$ e ogni stato $\ket{B}\in\hilbert_2$, e tutte le possibili combinazioni lineari di essi.
Il \textit{bra} associato al \textit{ket} $\ket{A}\otimes\ket{B}$ è semplicemente $\bra{A}\otimes\bra{B}$; con questo, il prodotto interno nello spazio prodotto è definito come
\begin{equation}
	(\bra{A}\otimes\bra{B})(\ket{C}\otimes\ket{D})=\braket{A}{C}\braket{B}{D}
\end{equation}
che, in termini di probabilità, mostra come gli ``eventi'' nei due sistemi siano indipendenti, poich\'e la probabilità dell'evento composto risulta essere il prodotto delle singole probabilità.
Spesso lo stato composto si scrive omettendo il simbolo $\otimes$, o anche in un unico \textit{ket}, come
\begin{equation}
	\ket{A}\otimes\ket{B}\hspace{2cm}\ket{A}\ket{B}\hspace{2cm}\ket{A,B}.
\end{equation}
Utilizzeremo per un po' la prima e la terza notazione insieme.

Anche se possiamo costruire uno stato composto dal prodotto di due stati, non è vero che ogni stato del sistema composto si possa scrivere in tale modo: in generale, uno stato $\ket{\psi}\in\hilbert=\hilbert_1\otimes\hilbert_2$ è una combinazione lineare di stati composti.
Date due basi di autostati (eventualmente anche continui) $\ket{i}$ e $\ket{j}$ per gli spazi, una base dello spazio prodotto è l'insieme $\{\ket{i}\otimes\ket{j}=\ket{i,j}\}$, quindi in generale
\begin{equation}
	\hilbert\ni\ket{\psi}=\sum_{i,j}c_{ij}\ket{i,j}=\sum_{i,j}c_{ij}\ket{i}\otimes\ket{j}
\end{equation}
dove come al solito $c_{ij}=\braket{i,j}{\psi}=(\bra{i}\otimes\bra{j})\ket{\psi}$.
Solo nel caso in cui $c_{ij}$ si può scrivere nella forma $a_ib_j$ allora si ha
\begin{equation}
	\ket{\psi}=\sum_{i,j}a_ib_j\ket{i,j}=\sum_{i,j}a_i\ket{i}\otimes b_j\ket{j}=\bigg(\sum_ia_i\ket{i}\bigg)\otimes\bigg(\sum_jb_j\ket{j}\bigg)
\end{equation}
cioè $\ket{\psi}$ si può scrivere come prodotto di due stati semplici, e si dice talvolta \emph{separabile}.
Quando non esiste una tale fattorizzazione, lo stato è detto \emph{entangled}.\footnote{
	Letteralmente \textit{ingrovigliato}, \textit{intrecciato}.
	Il termine inglese è di uso comune, e raramente si trova tradotto in italiano.
}

Gli operatori nello spazio composto in generale sono, anche qui, combinazioni lineari di prodotti diretti di operatori di ciascuno spazio: dati due operatori $\op\xi^{(1)}$ e $\op\eta^{(2)}$ rispettivamente su $\hilbert_1$ e su $\hilbert_2$, possiamo associare ad essi l'operatore $\op\xi^{(1)}\otimes\op\eta^{(2)}$ sullo spazio composto tale che
\begin{equation}
	(\op\xi^{(1)}\otimes\op\eta^{(2)})(\ket{A}\otimes\ket{B})=(\op\xi^{(1)}\ket{A})\otimes(\op\eta^{(2)}\ket{B}).
\end{equation}
Notare come ogni operatore agisce soltanto sullo stato del ``proprio'' spazio, indipendentemente dal resto.
Ogni operatore di ciascuno spazio, d'altronde, si estende ad un operatore sullo spazio composto in modo naturale: l'operatore $\op\xi^{(1)}$ può essere esteso all'operatore
\begin{equation}
	\op\xi^{(1\otimes 2)}=\op\xi^{(1)}\otimes 1^{(2)}
\end{equation}
dove $1^{(2)}$ è l'identità dello spazio $\hilbert_2$.
Analogamente gli operatori di $\hilbert_2$ si estendono ad operatori del tipo $1^{(1)}\otimes\op\eta^{(2)}$.
Spesso ometteremo gli indici nei pedici, dato che nella maggior parte dei casi $\op\xi^{(1)}$ e $\op\xi^{(1)}\otimes 1^{(2)}$ rappresentano la stessa osservabile, solo definita in spazi diversi.\footnote{
	In generale però l'osservabile $\op\xi$ potrebbe non essere definita nell'altro spazio.
}
L'identità dello spazio composto è ovviamente $1^{(1)}\otimes 1^{(2)}$.
Come già anticipato, le osservabili $\op\xi$ e $\op\eta$ in questo caso fanno riferimento a sistemi differenti, dunque sono \emph{sempre compatibili}.
Una semplice conseguenza di questo fatto è che se $\op\xi\ket{\xi}=\xi\ket{\xi}$ e $\op\eta\ket{\eta}=\eta\ket{\eta}$, dove $\ket{\xi}\in\hilbert_1$ e $\ket{\eta}\in\hilbert_2$, allora
\begin{equation}
	\op\xi\op\eta\ket{\xi,\eta}=(\op\xi\otimes 1)(1\otimes\op\eta)(\ket{\xi}\otimes\ket{\eta})=\xi\eta\ket{\xi,\eta}
\end{equation}
cioè l'autovalore dello stato composto è il prodotto degli autovalori.
Un altro esempio è
\begin{equation}
	(\op\xi+\op\eta)\ket{\xi,\eta}=(\op\xi\otimes 1+1\otimes\op\eta)(\ket{\xi}\otimes\ket{\eta})=(\xi+\eta)\ket{\xi,\eta}
\end{equation}
Cos\`i come per gli stati, in generale un operatore non è il prodotto diretto di due operatori, e $\op\xi+\op\eta$ ne è un esempio.
Gli elementi di matrice di un operatore $\op\xi$ sullo spazio composto, data una base $\ket{i,j}$, si definiscono nel modo solito come $\xi_{iji'j'}\defeq\bra{i,j}\op\xi\ket{i',j'}$ in modo che
\begin{equation}
	\op\xi=\sum_{i,j,i',j'}\xi_{iji'j'}\ket{i,j}\bra{i',j'}
\end{equation}
e se $\op\xi$ si fattorizza nel prodotto $\op\eta\op\zeta=(\op\eta\otimes 1)(1\otimes\op\zeta)$ allora
\begin{equation}
	\bra{i,j}\op\xi\ket{i',j'}=\bra{i,j}\op\eta\op\zeta\ket{i',j'}=\bra{i}\op\eta\ket{i'}\bra{j}\op\zeta\ket{j'}
\end{equation}
cioè l'elemento di matrice dell'operatore prodotto non è altro che il prodotto degli elementi di matrice dei due fattori, nelle basi di ciascun singolo spazio.

\section{Oscillatore armonico isotropo}
Come primo facile esempio di sistema con più gradi di libertà riprendiamo l'oscillatore armonico in più dimensioni già visto a pagina \pageref{sec:oscillatore-armonico-multidimensionale}.
Per semplicità consideriamo solo due gradi di libertà, dato che la generalizzazione ad un numero maggiore è molto semplice.
Un oscillatore armonico è dato dal potenziale
\begin{equation}
	V(x_1,x_2)=\frac12m(\omega_1x_1^2+\omega_2x_2^2)
\end{equation}
ed è \emph{isotropo} se tutte le frequenze di oscillazione coincidono, in questo caso $\omega_1=\omega_2=\omega$.
L'hamiltoniano del sistema è dunque
\begin{equation}
	\op H=\frac1{2m}(p_1^2+p_2^2)+\frac12m\omega^2(x_1^2+x_2^2).
\end{equation}
Indichiamo i due gradi di libertà con dei generici indici 1 e 2 in quanto non sono necessariamente la due coordinate cartesiane di una particella in un piano, ma possono avere altri significati: i risultati che troveremo si applicano, senza alcun cambiamento, anche ad un sistema di due particelle in moto unidimensionale.
Con la notazione dei prodotti tensoriali che abbiamo introdotto, è chiaro che $\op x_1$ corrisponde all'operatore $\op x\otimes 1$ nello spazio composto, e analogamente $\op x_2=1\otimes\op x$, $\op p_1=\op p\otimes 1$ e $\op p_2=1\otimes\op p$.\footnote{
	Nell'indicare gli operatori composti omettiamo l'indice che indica a quale spazio si riferiscono, dato che è la stessa osservabile in spazi diversi.
	Lo spazio a cui fanno riferimento risulta comunque è ovvio in base alla posizione dell'operatore nel prodotto: $\op p\otimes 1$ è l'impulso del primo spazio, e cos\`i via.
}
Risulta inoltre che $(\op x\otimes 1)^2=\op x^2\otimes 1$ e analogamente per gli altri.
Allora l'hamiltoniano si riscrive come
\begin{equation}
	\begin{split}
		\op H&=\frac1{2m}(\op p^2\otimes 1+1\otimes\op p^2)+\frac12m\omega^2(\op x^2\otimes 1+1\otimes\op x^2)=\\
		&=\frac1{2m}(\op p^2\otimes 1+m^2\omega^2\op x^2\otimes 1+1\otimes\op p^2+m^2\omega^2 1\otimes\op x^2)=\\
		&=\frac1{2m}(\op p^2+m^2\omega^2\op x^2)\otimes 1+\frac1{2m}1\otimes(\op p^2+m^2\omega^2\op x^2)=\\
		&=\op H^{(1)}\otimes 1+1\otimes\op H^{(2)}
	\end{split}
\end{equation}
cioè l'hamiltoniano composto non è altro che la somma dei due hamiltoniani dei sottosistemi.
Dato che i due sistemi sono indipendenti, non interagiscono, ed è lecito aspettarsi che i due hamiltoniani rimangano ``separati''.
Se dunque $\ket{E_1}$ e $\ket{E_2}$ sono autostati, in ordine, dei due hamiltoniani con autovalori $E_1$ ed $E_2$, allora
\begin{multline}
	\op H(\ket{E_1}\otimes\ket{E_2})=
	(\op H^{(1)}\otimes 1)(\ket{E_1}\otimes\ket{E_2})+(1\otimes\op H^{(2)})(\ket{E_1}\otimes\ket{E_2})=\\=
	\op H^{(1)}\ket{E_1}\otimes\ket{E_2}+\ket{E_1}\otimes\op H^{(2)}\ket{E_2}=E_1\ket{E_1}\otimes\ket{E_2}+\ket{E_1}\otimes E_2\ket{E_2}=(E_1+E_2)\ket{E_1}\otimes\ket{E_2}
\end{multline}
cioè $\ket{E_1}\otimes\ket{E_2}$ è un autostato dell'hamiltoniano nel sistema composto.
Conosciamo già gli autostati dei singoli sistemi, ossia gli stati $\ket{n}$ tali che $\op H\ket{n}=(n+\frac12)\hbar\omega\ket{n}$: allora prese le basi $\{\ket{n_1}\}_{n_1=0}^{+\infty}$ e $\{\ket{n_2}\}_{n_2=0}^{+\infty}$ possiamo costruire la base del sistema composto data dagli stati
\begin{equation}
	\ket{n_1}\otimes\ket{n_2}=\ket{n_1,n_2}
\end{equation}
che sono tutti autostati dell'hamiltoniano, con autovalore $(n_1+n_2+1)\hbar\omega$.

La funzione d'onda, nelle variabili $x_1,x_2$ della posizione, dello stato composto è data da 
\begin{equation}
	\braket{x_1,x_2}{n_1,n_2}=(\bra{x_1}\otimes\bra{x_2})(\ket{n_1}\otimes\ket{n_2})=\braket{x_1}{n_1}\braket{x_2}{n_2}
\end{equation}
perciò la funzione d'onda dello stato composto è semplicemente il prodotto delle funzioni d'onda dei due sottosistemi: la notazione con il prodotto tensoriale rende subito evidente questo fatto.
\begin{figure}
	\tikzsetnextfilename{autofunzione-oscillatore2d-isotropo}
	\centering
	\begin{tikzpicture}
		\begin{axis}[
				xlabel=$x_1$,
				ylabel=$x_2$,
				view={-30}{50}
			]
			\addplot3[surf,domain=-3:3,domain y=-3:3,samples=50] function {(2*sqrt(pi))**(-1)*(2*x)**2*exp(-x**2)*(2**2*2!*sqrt(pi))**(-1)*(4*y**2-2)**2*exp(-y**2)}; % |1,2>
		\end{axis}
	\end{tikzpicture}
	\caption{Densità di probabilità $\abs{\psi(x_1,x_2)}^2$ dello stato $\ket{1}\otimes\ket{2}$ dell'oscillatore armonico isotropo in due dimensioni.}
	\label{fig:autofunzione-oscillatore2d-isotropo}
\end{figure}


\section{Buca di potenziale quadrata}
Consideriamo il sistema composto da una buca di potenziale quadrata e infinita, di lato $a$: è descritta dal potenziale
\begin{equation}
	V(x_1,x_2)=
	\begin{cases*}
		0		&se $0<x<a,\ 0<y<a$\\
		+\infty	&altrimenti
	\end{cases*}
\end{equation}
Possiamo interpretarla equivalentemente come una particella racchiusa in una buca bidimensionale o due particelle identiche nella medesima buca unidimensionale.
Sappiamo già che all'esterno della buca la funzione d'onda deve essere identicamente nulla, quindi consideriamo d'ora in poi solo l'interno.
L'hamiltoniano del sistema è
\begin{equation}
	\op H=\frac1{2m}(\op p_1^2+\op p_2^2)=\frac1{2m}(\op p^2\otimes 1+1\otimes\op p^2)
\end{equation}
che si scompone in modo ovvio in due hamiltoniani, uno per ciascun grado di libertà.
Gli autostati di $\op H$ del sistema composto saranno allora il prodotto tensoriale degli autostati dei singoli sistemi: poich\'e una buca di potenziale unidimensionale, con tale potenziale, ammette le autofunzioni
\begin{equation}
	\psi_n(x)=\sqrt{\frac2{a}}\sin\frac{n\pi x}{a}
	\hspace{1cm}\text{con }
	E_n=\frac{\hbar^2n^2\pi^2}{2ma^2}
\end{equation}
per l'autostato $\ket{n}$, il sistema composto ammetterà le autofunzioni
\begin{equation}
	\braket{x_1,x_2}{n_1,n_2}=\braket{x_1}{n_1}\braket{x_2}{n_2}=\frac2{a}\sin\frac{n_1\pi x_1}{a}\sin\frac{n_2\pi x_2}{a}
\end{equation}
con l'energia data dalla somma degli autovalori
\begin{equation}
	E=E_{n_1}+E_{n_2}=\frac{\hbar^2(n_1^2+n_2^2)\pi^2}{2ma^2}.
\end{equation}

