\chapter{Sistemi con più gradi di libertà}
Salvo brevi eccezioni, finora abbiamo trattato solamente dei sistemi ad un grado di libertà, tipicamente una particella in moto lungo una retta.
Passiamo ora a studiare sistemi con più gradi di libertà, che possono corrispondere ad un maggior numero di particelle, all'aggiunta di altre dimensioni spaziali, o entrambe.
In questo capitolo ci occuperemo di introdurre lo schema matematico in cui inquadrare questi problemi, in via generale, per poterlo poi applicare a problemi più complessi di quelli già visti, come le rotazioni e gli atomi.

\section{Prodotto diretto di spazi vettoriali}
Il problema principale è capire come costruire lo spazio degli stati di tali sistemi.
Sappiamo che con un grado di libertà possiamo associare ad ogni stato un raggio in uno spazio di Hilbert complesso, separabile e infinito dimensionale.
Quando il sistema possiede più di una dimensione spaziale, tali dimensioni sono indipendenti e apportano ciascuna un grado di libertà; per ciascuna di esse abbiamo, ad esempio, un operatore della posizione e uno dell'impulso, e sappiamo già dai commutatori fondamentali che posizione e impulso di coordinate differenti commutano.
Analogamente, componendo più particelle in un sistema ciascuna di esse ha associata la coppia di operatori posizione e impulso (ad esempio) e tutti quelli che ne derivano.
/bin/bash: :wq: command not found
\begin{equation}
	\hilbert=\hilbert_1\otimes\hilbert_2\otimes\cdots\otimes\hilbert_d.
\end{equation}

Per studiare la definizione e le proprietà di base ci restringiamo ora al prodotto di due spazi $\hilbert_1$ e $\hilbert_2$.
Dati gli stati (qualsiasi) $\ket{A}\in\hilbert_1$ e $\ket{B}\in\hilbert_2$, possiamo costruire il loro prodotto diretto $\ket{A}\otimes\ket{B}$ che rappresenta lo stato del sistema combinato $\hilbert=\hilbert_1\otimes\hilbert_2$ in cui il sottosistema $\hilbert_1$ è nello stato $\ket{A}$ e il sottosistema $\hilbert_2$ è nello stato $\ket{B}$.
Chiaramente il prodotto non è commutativo, perch\'e in generale non è detto nemmeno che esista uno stato $\ket{A}$ in $\hilbert_2$ in modo da poter scrivere $\ket{B}\otimes\ket{A}$.
Lo spazio vettoriale $\hilbert_1\otimes\hilbert_2$ è dotato in modo naturale delle operazioni di somma e di moltiplicazione per scalare (un numero complesso) esattamente come gli spazi di partenza: esso contiene le coppie $\ket{A}\otimes\ket{B}$ per \emph{ogni} stato $\ket{A}\in\hilbert_1$ e ogni stato $\ket{B}\in\hilbert_2$, e tutte le possibili combinazioni lineari di essi.
Il \textit{bra} associato al \textit{ket} $\ket{A}\otimes\ket{B}$ è semplicemente $\bra{A}\otimes\bra{B}$; con questo, il prodotto interno nello spazio prodotto è definito come
\begin{equation}
	(\bra{A}\otimes\bra{B})(\ket{C}\otimes\ket{D})=\braket{A}{C}\braket{B}{D}
\end{equation}
che, in termini di probabilità, mostra come gli ``eventi'' nei due sistemi siano indipendenti, poich\'e la probabilità dell'evento composto risulta essere il prodotto delle singole probabilità.
Spesso lo stato composto si scrive omettendo il simbolo $\otimes$, o anche in un unico \textit{ket}, come
\begin{equation}
	\ket{A}\otimes\ket{B}\hspace{2cm}\ket{A}\ket{B}\hspace{2cm}\ket{A,B}.
\end{equation}
Utilizzeremo per un po' la prima e la terza notazione insieme.

Anche se possiamo costruire uno stato composto dal prodotto di due stati, non è vero che ogni stato del sistema composto si possa scrivere in tale modo: in generale, uno stato $\ket{\psi}\in\hilbert=\hilbert_1\otimes\hilbert_2$ è una combinazione lineare di stati composti.
Date due basi di autostati (eventualmente anche continui) $\ket{i}$ e $\ket{j}$ per gli spazi, una base dello spazio prodotto è l'insieme $\{\ket{i}\otimes\ket{j}=\ket{i,j}\}$, quindi in generale
\begin{equation}
	\hilbert\ni\ket{\psi}=\sum_{i,j}c_{ij}\ket{i,j}=\sum_{i,j}c_{ij}\ket{i}\otimes\ket{j}
\end{equation}
dove come al solito $c_{ij}=\braket{i,j}{\psi}=(\bra{i}\otimes\bra{j})\ket{\psi}$.
Solo nel caso in cui $c_{ij}$ si può scrivere nella forma $a_ib_j$ allora si ha
\begin{equation}
	\ket{\psi}=\sum_{i,j}a_ib_j\ket{i,j}=\sum_{i,j}a_i\ket{i}\otimes b_j\ket{j}=\bigg(\sum_ia_i\ket{i}\bigg)\otimes\bigg(\sum_jb_j\ket{j}\bigg)
\end{equation}
cioè $\ket{\psi}$ si può scrivere come prodotto di due stati semplici, e si dice talvolta \emph{separabile}.
Quando non esiste una tale fattorizzazione, lo stato è detto \emph{entangled}.\footnote{
	Letteralmente \textit{ingrovigliato}, \textit{intrecciato}.
	Il termine inglese è di uso comune, e raramente si trova tradotto in italiano.
}

Gli operatori nello spazio composto in generale sono, anche qui, combinazioni lineari di prodotti diretti di operatori di ciascuno spazio: dati due operatori $\op\xi^{(1)}$ e $\op\eta^{(2)}$ rispettivamente su $\hilbert_1$ e su $\hilbert_2$, possiamo associare ad essi l'operatore $\op\xi^{(1)}\otimes\op\eta^{(2)}$ sullo spazio composto tale che
\begin{equation}
	(\op\xi^{(1)}\otimes\op\eta^{(2)})(\ket{A}\otimes\ket{B})=(\op\xi^{(1)}\ket{A})\otimes(\op\eta^{(2)}\ket{B}).
\end{equation}
Notare come ogni operatore agisce soltanto sullo stato del ``proprio'' spazio, indipendentemente dal resto.
Ogni operatore di ciascuno spazio, d'altronde, si estende ad un operatore sullo spazio composto in modo naturale: l'operatore $\op\xi^{(1)}$ può essere esteso all'operatore
\begin{equation}
	\op\xi^{(1\otimes 2)}=\op\xi^{(1)}\otimes 1^{(2)}
\end{equation}
dove $1^{(2)}$ è l'identità dello spazio $\hilbert_2$.
Analogamente gli operatori di $\hilbert_2$ si estendono ad operatori del tipo $1^{(1)}\otimes\op\eta^{(2)}$.
Spesso ometteremo gli indici nei pedici, dato che nella maggior parte dei casi $\op\xi^{(1)}$ e $\op\xi^{(1)}\otimes 1^{(2)}$ rappresentano la stessa osservabile, solo definita in spazi diversi.\footnote{
	In generale però l'osservabile $\op\xi$ potrebbe non essere definita nell'altro spazio.
}
L'identità dello spazio composto è ovviamente $1^{(1)}\otimes 1^{(2)}$.
Come già anticipato, le osservabili $\op\xi$ e $\op\eta$ in questo caso fanno riferimento a sistemi differenti, dunque sono \emph{sempre compatibili}.
Una semplice conseguenza di questo fatto è che se $\op\xi\ket{\xi}=\xi\ket{\xi}$ e $\op\eta\ket{\eta}=\eta\ket{\eta}$, dove $\ket{\xi}\in\hilbert_1$ e $\ket{\eta}\in\hilbert_2$, allora
\begin{equation}
	\op\xi\op\eta\ket{\xi,\eta}=(\op\xi\otimes 1)(1\otimes\op\eta)(\ket{\xi}\otimes\ket{\eta})=\xi\eta\ket{\xi,\eta}
\end{equation}
cioè l'autovalore dello stato composto è il prodotto degli autovalori.
Un altro esempio è
\begin{equation}
	(\op\xi+\op\eta)\ket{\xi,\eta}=(\op\xi\otimes 1+1\otimes\op\eta)(\ket{\xi}\otimes\ket{\eta})=(\xi+\eta)\ket{\xi,\eta}
\end{equation}
Cos\`i come per gli stati, in generale un operatore non è il prodotto diretto di due operatori, e $\op\xi+\op\eta$ ne è un esempio.
Gli elementi di matrice di un operatore $\op\xi$ sullo spazio composto, data una base $\ket{i,j}$, si definiscono nel modo solito come $\xi_{iji'j'}\defeq\bra{i,j}\op\xi\ket{i',j'}$ in modo che
\begin{equation}
	\op\xi=\sum_{i,j,i',j'}\xi_{iji'j'}\ket{i,j}\bra{i',j'}
\end{equation}
e se $\op\xi$ si fattorizza nel prodotto $\op\eta\op\zeta=(\op\eta\otimes 1)(1\otimes\op\zeta)$ allora
\begin{equation}
	\bra{i,j}\op\xi\ket{i',j'}=\bra{i,j}\op\eta\op\zeta\ket{i',j'}=\bra{i}\op\eta\ket{i'}\bra{j}\op\zeta\ket{j'}
\end{equation}
cioè l'elemento di matrice dell'operatore prodotto non è altro che il prodotto degli elementi di matrice dei due fattori, nelle basi di ciascun singolo spazio.

