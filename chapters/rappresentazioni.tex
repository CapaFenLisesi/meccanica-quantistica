\chapter{Rappresentazioni}
Poich\'e i risultati possibili delle misure di osservabili sono gli autovalori dell'operatore associato, è di fondamentale importanza nella meccanica quantistica saper ricavare lo spettro degli operatori.
Gli autovalori più ``importanti'' sono quelli di energia, che formano lo spettro dell'operatore hamiltoniano: dobbiamo quindi risolvere l'\emph{equazione agli autovalori}
\begin{equation}
	\op H\ket{E}=E\ket{E}.
	\label{eq:equazione-autovalori-hamiltoniano}
\end{equation}

Cos\`i come in algebra lineare, per agevolare i calcoli, si è soliti fissare una base e trasformare i vettori di $\R^n$ alle $n$-uple di numeri reali, anche in questo caso conviene passare da una trattazione generica di stati e operatori ad un formalismo più pratico per i calcoli.
Anche in questo caso troviamo degli isomorfismi che trasformano lo spazio degli stati in spazi di Hilberti più comodi: questi isomorfismi sono detti \emph{rappresentazioni}.
Tra le varie rappresentazioni, vedremo quella di Heisenberg e quella di Schr\"odinger.

\section{Rappresentazione di Heisenberg}
Prendiamo un insieme completo ortonormale $\{\ket{i}\}_{i\in\N}$, che forma quindi una base per lo spazio degli stati: possiamo pensare a questa base come al sistema completo di autostati di un'osservabile non degenere.
Abbiamo già visto che si può rappresentare uno stato qualsiasi $\ket{x}$ nei termini di questa base, tramite le proiezioni sui suoi elementi:
\begin{equation}
	\ket{x}=\ser{i}\ket{i}\braket{i}{x}=\ser{i}x_i\ket{i}.
	\label{eq:rappresentazione-stato-heisenberg}
\end{equation}
Il braket $\braket{i}{x}$ fornisce quindi il coefficiente dell'$i$-esimo stato della base.
Il prodotto scalare è dato come, sempre inserendo l'operatore identità,
\begin{equation}
	\braket{x}{y}=\bra{x}\Big(\ser{i}\ket{i}\bra{i}\Big)\ket{y}=\ser{i}\braket{x}{i}\braket{i}{y}=\ser{i}\braket{i}{x}^*\braket{i}{y}=\ser{i}x_i^*y_i
	\label{eq:prodotto-scalare}
\end{equation}
e la norma di uno stato (al quadrato)
\begin{equation}
	\braket{x}{x}=\ser{i}x_i^*x_i=\ser{i}\abs{x_i}^2.
	\label{eq:norma-stato}
\end{equation}
Gli stati $\ket{x}$ e $\ket{y}$ chiaramente esistono e hanno un loro significato indipendentemente dalla base in cui li rappresentiamo.
Con queste associazioni, però, fissata una base possiamo identificare qualsiasi stato tramite i suoi coefficienti $x_i$ che appaiono nella \eqref{eq:rappresentazione-stato-heisenberg}: l'insieme ordinato $\{x_i\}_{i\in\N}$ forma dunque una successione, o un vettore colonna di infinite componenti.
Abbiamo però un ulteriore vincolo, ossia la limitatezza della norma di uno stato: $\braket{x}{x}<+\infty$.
Dunque avremo che
\begin{equation}
	\ser{i}\abs{x_i}^2<+\infty
	\label{eq:successione-quadrato-sommabile}
\end{equation}
ossia la successione $\{x_i\}$ è a \emph{quadrato sommabile}.

Abbiamo trovato dunque un isomorfismo tra lo spazio degli stati e lo spazio delle successioni complesse a quadrato sommabile, che si indica con $\ell^2(\C)$, anch'esso ovviamente uno spazio di Hilbert.
Questo isomorfismo non è canonico, nel senso che è possibile farlo solo tramite la scelta di una base: perciò è sbagliato affermare che la successione $\{x_i\}$ \emph{è} lo stato $\ket{x}$, ma è corretto dire che la successione \emph{rappresenta} lo stato (in una certa base).
Questo isomorfismo è la \emph{rappresentazione di Heisenberg}.

Fissata la rappresentazione degli stati, dobbiamo vedere ora come rappresentare gli operatori: anche qui, l'espressione $\ket{y}=\op A\ket{x}$ non necessita della scelta di una base.
Prendiamo la base $\{\ket{i}\}_{i\in\N}$: per trovare la $i$-esima componente, in questa base, di $\ket{y}$, moltiplichiamo a sinistra i membri per il bra fondamentale $\bra{i}$, ottenendo
\begin{equation}
	\braket{i}{y}=y_i=\bra{i}\op A\ket{x}.
\end{equation}
Introducendo la risoluzione dell'identità tra $\op A$ e $\ket{x}$ troviamo dunque
\begin{equation}
	y_i=\bra{i}\op A\Big(\ser{j}\ket{j}\bra{j}\Big)\ket{x}=\ser{j}\bra{i}\op A\ket{j}\braket{j}{x}=\ser{j}\bra{i}\op A\ket{j}x_j.
	\label{eq:rappresentazione-operatore-matrice}
\end{equation}
Vediamo quindi che l'azione di $\op A$ sullo stato $\ket{x}$ si esprime tramite gli elementi $\bra{i}\op A\ket{j}\defeq A_{ij}$, che sono dei numeri complessi con un doppio indice: chiaramente questi indicano una matrice (di dimensioni infinite). 
L'applicazione dell'operatore $\op A$ allo stato $\ket{x}$ si rappresenta dunque come il prodotto riga per colonna tra la matrice rappresentativa di $\op A$ e il vettore/successione rappresentativo di $\ket{x}$.
Dato un altro operatore $\op B$, il prodotto $\op A\op B$ è rappresentato dalla matrice di componenti $(AB)_{ij}=\bra{i}\op A\op B\ket{j}$, e vediamo che
\begin{equation}
	(AB)_{ij}=\bra{i}\op A\op B\ket{j}=\bra{i}\op A\ser{k}\ket{k}\bra{k}\op B\ket{j}=\ser{k}\bra{i}\op A\ket{k}\bra{k}\op B\ket{j}=\ser{k}A_{ik}B_{kj}
	\label{eq:prodotto-operatori-matrici}
\end{equation}
che è proprio il prodotto riga per colonna delle due matrici rappresentative.
L'aggiunto dell'operatore inoltre è tale che $\bra{i}\adj{\op A}\ket{j}=\bra{j}\op A\ket{i}^*$, cioè $(\adj A)_{ij}=A_{ji}^*$: la matrice rappresentativa dell'aggiunto è la trasposta coniugata di $A$.

Chiarito dunque come si rappresentano stati e operatori, torniamo all'equazione agli autovalori vista all'inizio del capitolo: nella rappresentazione di Heisenberg che abbiamo visto, il metodo più conveniente per risolverla è esprimerla in una base di autostati dell'operatore coinvolto.
Se prendiamo un'osservabile $\xi$ non degenere e i suoi autostati $\{\ket{\xi_i}\}$, che prendiamo normalizzati, abbiamo che
\begin{equation}
	\bra{\xi_i}\op\xi\ket{\xi_j}=\xi_j\braket{\xi_i}{\xi_j}=\xi_j\delta_{ij}
\end{equation}
quindi la matrice rappresentativa è diagonale.

Per la non degenerazione dell'osservabile, ad ogni autovalore corrisponde un solo autovettore, a meno di costanti scalari: questo però non è un problema se ricordiamo che ad uno stato fisico corrisponde un \emph{raggio} nello spazio dei ket, quindi l'autovettore e i suoi multipli rappresentano in realtà lo stesso stato fisico.
La normalizzazione degli autostati già risolve parzialmente questo problema, ma lo stato $\ket{\xi_i}$ e $e^{i\phi_i}\ket{\xi_i}$ rappresentano lo stesso stato fisico e hanno entrambi norma unitaria, di conseguenza c'è ancora una certa arbitrarietà, nella scelta di queste fasi $\phi_i$.
In ogni caso, se al posto di $\ket{\xi_i}$ e $\ket{\xi_j}$ prendiamo gli stati $\ket{\xi_i'}=e^{i\phi_i}\ket{\xi_i}$ e $\ket{\xi_j'}=e^{i\phi_j}\ket{\xi_j}$ troviamo
\begin{equation}
	\braket{\xi_i'}{\xi_j'}=e^{-i\phi_i}\bra{\xi_i}e^{i\phi_j}\ket{\xi_j}=e^{i(\phi_j-\phi_i)}\braket{\xi_i}{\xi_j}=e^{i(\phi_j-\phi_i)}\delta_{ij}
\end{equation}
quindi sono ancora ortonormali: se $i=j$ le fasi si cancellano, mentre se $i\ne j$ si ha $e^{i(\phi_j-\phi_i)}\ne 1$ ma $\delta_{ij}=0$ quindi è comunque nullo.

Se l'osservabile invece è degenere, la scelta di un sistema ortonormale completo di autostati non è più univocamente determinata (a meno di fattori scalari), perch\'e ad ogni autovalore possono corrispondere anche più autostati linearmente indipendenti.
Prendiamo un autovalore degenere $\xi_0$ dell'osservabile $\xi$: se scriviamo la matrice rappresentativa di $\op\xi$ nella base di autostati, l'autospazio $\hilbert_0$ di tale autovalore è lasciato invariato dall'operatore, di conseguenza il blocco di matrice relativo all'autovalore $\xi_0$ è un multiplo dell'identità, e ha dimensione pari al grado di degenerazione dell'autovalore.
La matrice rappresentativa è quindi una matrice a blocchi.
Se prendiamo ora un'osservabile $\eta$ compatibile con $\xi$, le due condividono una base di autostati, perciò possiamo cambiare base in quest'ultima.
In generale, la matrice di $\xi$ anche in questo caso sarà ancora a blocchi: ma lo è anche quella di $\eta$, perch\'e sono compatibili.
Quindi, se $\xi$ è rappresentata da una matrice diagonale, ossia in cui tutti i blocchi hanno dimensione 1, allora anche la matrice di $\eta$ è diagonale.

Aggiungendo altre osservabili compatibili a $\xi$ fino a raggiungere un sistema completo di osservabili, si giunge infine ad una base di autostati ben determinata (senza ``libertà di scelta'' tra autostati linearmente indipendenti in un autospazio) di conseguenza la matrice rappresentativa sarà finalmente diagonale.

Questa rappresentazione di Heisenberg risulta forse familiare, perch\'e le successioni di $\ell^2(\C)$ possono sembrare una naturale estensione dei vettori di numeri complessi.
In realtà l'uso è poco pratico, perch\'e individuare gli autovalori è generalmente difficile: l'equazione $\op H\ket{x}=E\ket{x}$ si rappresenta come
\begin{equation}
	\ser{m}H_{nm}x_m=Ex_n\quad\then\quad\ser{m}(H_{nm}-E\delta_{nm})x_m=0
\end{equation}
che è un sistema lineare.
Il problema è che questo sistema ha un numero infinito di equazioni!
Ulteriormente, gli autovalori $E$ si ricavano imponendo $\det(H-E\id)=0$, ma un determinante di dimensione infinita (e di conseguenza un polinomio di grado infinito) semplicemente non esiste: dobbiamo quindi ricercare un metodo più astratto di calcolarli.
Inoltre, posizione e impulso sono osservabili particolari: sappiamo che sono impossibili da determinare con infinita precisione, per il principio di Heisenberg; gli autovalori di $\op q$ o $\op p$ meritano quindi una discussione particolare, soprattutto perch\'e tutte le altre osservabili si possono costruire come funzioni di queste due.

\section{Operatori di traslazione}
Costruiamo dall'operatore di impulso $\op p$ l'operatore
\begin{equation}
	\op T(z)=\exp\Big(-\frac{i}{\hbar}\op p z\Big)=\serz{n}\frac1{n!}\Big(-\frac{i}{\hbar}\op p z\Big)^k
	\label{eq:operatore-traslazione}
\end{equation}
in dipendenza dal parametro $z\in\R$.
Verifichiamo subito una sua importante proprietà, ossia\footnote{Se $A$ e $B$ non commutano, non è detto che $e^Ae^B=e^{A+B}$, ma deve essere usata la più generale formula di Baker-Campbell-Hausdorff. In questo caso, i due operatori chiaramente commutano quindi l'operazione è lecita.}
\begin{equation}
	\adj{\op T}(z)\op T(z)=\exp\Big(\frac{i}{\hbar}\op pz\Big)\exp\Big(-\frac{i}{\hbar}\op pz\Big)=\exp(\op 0)=\op 1
	\label{eq:unitarieta-traslazione}
\end{equation}
e analogamente $\op T(z)\adj{\op T}(z)=\op 1$: dunque $\op T(z)$ (indipendente dal parametro $z$) è un operatore unitario, e preserva il prodotto scalare.\footnote{Più in generale, è vero che se $\op K$ è antihermitiano allora il suo esponenziale $\exp\op K$ è unitario, con una dimostrazione analoga a questa (gli operatori antihermitiani sono sempre operatori normali, quindi commutano con il loro aggiunto). In questo caso, l'operatore antihermitiano, a meno di altre costanti, è $i\op p$, dato che $\op p$ è hermitiano.} 

Ammettiamo ora che esista un autostato dell'operatore posizione, che chiamiamo $\ket{q}$, tale che $\op q\ket{q}=q\ket{q}$.
Vogliamo vedere come l'operatore $\op T$ modifica questo stato, ossia calcoliamo $\op q\op T(z)\ket{q}$.
Non conoscendo l'effetto di $\op T(z)$ su tale stato, scambiamo l'ordine degli operatori scrivendo $\op q\op T(z)=\op q\op T(z)+\op T(z)\op q-\op T(z)\op q=[\op q,\op T(z)]+\op T(z)\op q$.
Il commutatore che abbiamo ricavato vale
\begin{equation}
	\begin{split}
		[\op q,\op T(z)]&=\Big[\op q,\serz{n}\frac1{n!}\Big(-\frac{i}{\hbar}\op pz\Big)^n\Big]=\\
		&=\serz{n}\frac1{n!}\Big[\op q,\Big(-\frac{i}{\hbar}\op pz\Big)^n\Big]=\\
		&=\serz{n}\frac1{n!}\Big(-\frac{iz}{\hbar}\Big)^n[\op q,\op p^n]=\\
		&=\serz{n}\frac1{n!}\Big(-\frac{iz}{\hbar}\Big)^ni\hbar n\op p^{n-1}=\\
		&=z\ser{n}\frac1{(n-1)!}\Big(-\frac{iz}{\hbar}\Big)^{n-1}\hbar\op p^{n-1}=\\
		&=z\serz{k}\frac1{k!}\Big(-\frac{iz}{\hbar}\Big)^ki\hbar \op p^k=\\
		&=z\op T(z).
	\end{split}
	\label{eq:commutatore-posizione-traslazione}
\end{equation}
Di conseguenza risulta
\begin{equation}
	\op q\op T(z)\ket{q}=(z\op T(z)+\op T(z)\op q)\ket{q}=z\op T(z)\ket{q}+\op T(z)q\ket{q}=(z+q)\op T(z)\ket{q},
\end{equation}
ossia il nuovo stato $\op T(z)\ket{q}$ è ancora un autostato di $\op q$, ma con autovalore $q+z$: $\op T(z)$ dunque \emph{trasla} gli autostati di $\op q$ di un fattore $z$.
Solitamente, indicando l'autovalore direttamente nel ket, questo si scrive come $\op T(z)\ket{q}=\ket{q+z}$.
L'operatore $\op T$ cos\`i definito è a tutti gli effetti un \emph{operatore di traslazione}, e infatti l'impulso (a cui abbiamo applicato una funzione esponenziale) è proprio il generatore delle traslazioni.

Questo risultato però ci porta ad un grave problema: poich\'e non c'è alcun vincolo sulla lunghezza $z$ della traslazione, che può essere un qualsiasi numero reale, abbiamo trovato che se esiste un autostato $\ket{q}$ di $\op q$ allora ne esiste un'\emph{infinità non numerabile}, poich\'e anche $\ket{q+z}$ è un autostato e $z\in\R$ qualunque!
Questo è in contraddizione con la struttura dello spazio degli stati, che è uno spazio di Hilbert \emph{separabile}: di conseguenza, non possono esistere autostati di $\op q$.
Lo stesso si può dire di $\op p$, con una costruzione analoga.
Nella prossima sezione vedremo come risolvere questo problema, analizzando più a fondo questi due particolari operatori.

\section{Operatori posizione e impulso}
Abbiamo dunque visto tramite un operatore di traslazione che $\op q$ (cos\`i come $\op p$) non può ammettere autostati.
Ciò potrebbe sembrare un grave problema, dal momento che è su questi due operatori che si costruiscono tutte le osservabili, ma nasconde in realtà un concetto diverso.

Se assumiamo di poter scrivere equazioni come $\op q\ket{q}=q\ket{q}$, stiamo ammettendo che $\op q$ possiede uno \emph{spettro continuo}.
Ricordiamo il principio di indeterminazione di Heisenberg, che afferma che il prodotto delle indeterminazioni di posizione e impulso è
\begin{equation}
	\Delta q\Delta p\geq\frac{\hbar}2.
\end{equation}
Da esso ricaviamo che è impossibile sapere con assoluta precisione la posizione o l'impulso del sistema in esame.
Di conseguenza dire che $\op q$ o $\op p$ hanno degli autovalori significherebbe ammettere che dopo una misura sul sistema possiamo conoscere con infinita precisione il valore della posizione o dell'impulso.
Dovrebbe essere ormai chiaro che una situazione del genere è impossibile, pertanto degli operatori con queste proprietà non hanno senso fisico.
Vediamo come possiamo risolvere il problema.

Quando misuriamo una posizione, quello che possiamo sapere è in realtà soltanto se la particella (o chi per essa) si trovi entro un certo intervallo di spazio, più o meno preciso.
Matematicamente, se diciamo che la particella è in $x$ stiamo in verità affermando che essa si trova nell'intervallo $[x-\Delta x, x+\Delta x]$ per un certo $\Delta x$: questa ampiezza $\Delta x$ indica la precisione della misura effettuata, ma per quanto detto tale $\Delta x$ non potrà mai essere zero.
Un'osservabile che restituisca questo tipo di risultato è assolutamente lecita: se lo strumento ha una sensibilità di $2\epsilon$ (fissata: possiamo prenderla quanto piccola vogliamo, ma mai tendente a zero), possiamo partizionare l'asse reale in tanti intervalli di ampiezza $\epsilon$, ossia in $(2n\epsilon-\epsilon,2n\epsilon+\epsilon]\equiv \big((2n-1)\epsilon,(2n+1)\epsilon\big]$, con $n\in\Z$.
L'insieme di questi intervalli è ovviamente numerabile, e la loro unione per tutti gli $n\in\Z$ ricopre tutto l'asse reale.
L'atto della misura consiste nel determinare in quale di questi intervalli si trova la particella: possiamo indicare ogni intervallo con il suo centro $q_n\defeq 2n\epsilon$, e con questo indicare il risultato della misura, ossia gli autovalori della posizione.
Possiamo quindi definire un operatore $\op q_\epsilon$ i cui autovalori sono questi $q_n$, ossia se otteniamo $\op q_\epsilon\ket{x}=q_n\ket{x}$ significa che la particella al momento della misura si trova nell'intervallo $[q_n-\epsilon,q_n+\epsilon]$.
Questi autovalori formano uno spettro discreto, e a questo operatore possiamo applicare tutte le regole già studiate senza alcun problema di definizione.
Sebbene questo operatore rispecchi fedelmente la realtà, però, è evidente che diventa scomodissimo non appena bisogna usarlo nei calcoli.
Oltretutto, esisterebbero infiniti operatori rappresentanti la posizione, uno per ogni valore della sensibilità $\epsilon$ assegnabile!

Cosa rappresenta dunque il ``vero'' operatore posizione?
Possiamo vederlo come un limite dell'operatore $\op q_\epsilon$ che abbiamo introdotto per una precisione sempre più piccola, cioè per $\epsilon\to 0$.
Questo passaggio al limite è però soltanto un'astrazione matematica, non rappresenta qualcosa di reale, di ``fisico''.
Tutto il discorso svolto fino a questo punto vale, ovviamente, anche per l'operatore impulso.
Tenendo ciò in mente possiamo riportare al caso continuo i risultati ottenuti finora nel caso discreto, e procedere con essi.
Accettiamo, dunque, \emph{con riserva} l'esistenza di questi operatori posizione e impulso con spettro continuo.

