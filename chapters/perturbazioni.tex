\chapter{Teoria delle perturbazioni}
Nel capitolo precedente abbiamo calcolato i livelli energetici dell'atomo di idrogeno.
Per quanto sia di grande importanza ``accademica'', però, sperimentalmente ha un interesse relativamente basso: per misurare queste energie, e in generale le proprietà del sistema, dobbiamo infatti farlo interagire con un agente esterno, tipicamente un campo elettromagnetico.
In generale lo studio dell'interazione tra materia e radiazione è molto importante, ma in questa trattazione troviamo molti problemi, come:
\begin{itemize}
	\item a rigore, tutto il sistema andrebbe quantizzato, ma il campo elettromagnetico ha un numero infinito di gradi di libertà;
	\item il campo elettromagnetico è invariante sotto ad un diverso gruppo di trasformazioni (quelle di Lorentz), che non abbiamo introdotto;
	\item anche introducendo le teorie relativistiche, dovremmo porre sullo stesso piano le quattro coordinate spaziotemporali, mentre finora il tempo è sempre stato un parametro e la posizione un operatore.
\end{itemize}
Dobbiamo quindi trovare un approccio più semplificato.
Se trattiamo il campo elettromagnetico in modo ``classico'', ad esempio, ci basterà aggiungere dei termini all'hamiltoniano.
Più in generale possiamo vedere quest'ultimo come composto da un termine $\op H_0$ che corrisponde al sistema isolato più un termine $\op H'$ che caratterizza la sua interazione con il campo,
\begin{equation}
	H=H_0+H'
	\label{eq:perturbazione}
\end{equation}
dove in generale $\op H'$ può anche dipendere esplicitamente dal tempo.
In quest'ottica possiamo vedere $\op H'$ come una \emph{perturbazione} del sistema isolato descritto da $\op H_0$.
Nel caso in cui $\op H'$ sia poi ``piccolo'' rispetto a $\op H_0$ possiamo poi analizzare il problema come si analizzerebbe un problema classico nell'ambito delle piccole oscillazioni.

Rimane però da capire cosa si intende con ``piccolo'': $\op H'$ è un operatore, non un numero, che in tutta generalità può anche essere illimitato.
Solitamente $\op H'$ e $\op H_0$ non sono compatibili, per cui non possiamo diagonalizzarli simultaneamente e confrontare gli autovalori.
Prendiamo una base $\{\ket{n}\}_{n\in\N}$ in cui $\op H_0$ è diagonale, e in questa base scriviamo la \eqref{eq:perturbazione} come
\begin{equation}
	\bra{n}\op H\ket{m}=H_{nm}=E_n\delta_{nm}+H'_{nm},
	\label{eq:perturbazione-base}
\end{equation}
dove $E_i$ sono gli autovalori di $\op H_0$.
Supponiamo ora che lo spazio degli stati sia bidimensionale, cos\`i da scrivere $\op H$ in forma matriciale:
\begin{equation}
	H=
	\begin{pmatrix}
		E_1+H'_{11}	&H'_{12}\\
		H'_{21}		&E_2+H'_{22}
	\end{pmatrix}.
\end{equation}
Dato che deve essere hermitiano, si ha $H'_{21}={H'_{12}}^*$ e $E_i+H'_{ii}\in\R$.
Il polinomio caratteristico è
\begin{multline}
	\chi_H(\lambda)=(E_1+H_{11}'-\lambda)(E_2+H'_{22}-\lambda)-\abs{H'_{12}}^2=\\
	=\lambda^2-(E_1+E_2+H'_{11}+H'_{22})\lambda+(E_1+H'_{11})(E_2+H'_{22})-\abs{H'_{12}}^2
	\label{eq:polinomio-caratteristico-H-perturbato-2d}
\end{multline}
da cui troviamo i due autovalori
\begin{equation}
	\begin{split}
		\lambda
		&=\frac12\bigg[E_1+E_2+H'_{11}+H'_{22}\pm\sqrt{(E_1+E_2+H'_{11}+H'_{22})^2-4(E_1+H'_{11})(E_2+H'_{22})+4\abs{H'_{12}}^2}\bigg]=\\
		&=\frac12\bigg[E_1+E_2+H'_{11}+H'_{22}\pm(E_1+H'_{11}-E_2-H'_{22})\sqrt{1+\frac{4\abs{H'_{12}}^2}{(E_1+H'_{11}-E_2-H'_{22})^2}}\bigg].
	\end{split}
	\label{eq:autovalori-H-perturbato-2d}
\end{equation}
Se $\abs{H'_{12}}^2\ll\abs{E_1+H'_{11}-E_2-H'_{22}}$ possiamo allora approssimare al primo ordine gli autovalori, in pratica trascurando la radice.
in tal caso, essi sono $E'_1\defeq E_1+H'_{11}$ e $E'_2\defeq E_2+H'_{22}$.
La condizione affinch\'e $\op H'$ sia piccolo può dunque essere formulata richedendo che gli elementi, in modulo, fuori dalla diagonale di esso siano molto minori della differenza $\abs{E'_1-E'_2}$.
Se inoltre $\abs{H'_{ii}}\ll\abs{E_i}$, allora $E'_i\approx E_i$, dunque la condizione può essere semplificata affermando che deve risultare
\begin{equation}
	\abs{H'_{ij}}\ll\abs{E_i-E_j},
	\label{eq:condizione-approssimazione-perturbazione-2d}
\end{equation}
potendo cos\`i usare i livelli energetici direttamente di $\op H_0$ anzich\'e di $\op H$ (i primi li conosciamo, se il sistema dato da $\op H_0$ è già noto, a differenza dei secondi).
Allora il valore di aspettazione di $\op H'$, calcolato sull'autostato imperturbato $\ket{E_i}$ di $\op H_0$, è proprio il termine da aggiungere al livello energetico corrispondente di $\op H_0$, che è l'autovalore in tale autostato.

\section{L'effetto Stark nell'atomo di idrogeno}
Consideriamo un sistema con un atomo di idrogeno posto in un campo elettrico costante nella direzione $z$ ($\vec E=\mathcal E\vec e_3$).
L'hamiltoniano del sistema è
\begin{equation}
	\op H=\op H_0+e\mathcal E\op x_3
	\label{eq:hamiltoniano-stark-idrogeno}
\end{equation}
dove $\op H_0$ è l'hamiltoniano, già visto, dell'atomo di idrogeno.
Nel sistema dell'atomo idrogeno isolato conosciamo le simmetrie, che portano alle degenerazioni con $n^2$ stati all'$n$-esimo livello di energia e $2l+1$ stati con momento angolare $l(l+1)\hbar^2$.
La perturbazione indotta da $\op H'$ potrebbe rimuovere alcune di queste degenerazioni, rompendo delle simmetrie.
Abbiamo infatti $[\op{\vec L},\op H_0]=0$, ma $[\op{\vec L},\op H']$ non è nullo: potremmo pensare inizialmente che, dato che $\op H'$ comprende il prodotto scalare $\scalar{\vec E}{\vec x}=\mathcal Ex_3$, esso commuti con $\op{\vec L}$, dato che i prodotti scalari sono invarianti per rotazioni dello spazio.
Ma tale momento angolare non è quello del sistema complessivo (formato dall'atomo e dal campo) ma solo dell'atomo, quindi esso genera solamente le rotazioni di quest'ultimo; poich\'e il campo ovviamente non ruota insieme all'atomo ma rimane fisso, non è invariante per questa rotazione, quindi $[\op{\vec L},\op H']\ne 0$.

L'elemento di matrice del termine $\op H'=e\mathcal E\op x_3$ è determinato da $\bra{n',l',m'}\op x_3\ket{n,l,m}$, dove $n,l,m$ sono i numeri quantici associati agli autovalori (per l'atomo di idrogeno imperturbato) rispettivamente di $\op H_0$, $\op{\vec L}^2$ e $\op L_3$.
Innanzitutto, ovviamente, risulta $\bra{n',l',m}\op x_3\ket{n,l,m}\leq\bra{n',l',m'}\op r\ket{n,l,m}$; trascurando per un attimo i numeri $l$ e $m$, dato che gli stati $\ket{n'}$ formano un insieme completo si ha
\begin{equation}
	\abs{\bra{n'}\op r\ket{n}}^2=\bra{n}\op r\ket{n'}\bra{n'}\op r\ket{n}\leq\sum_n\bra{n}\op r\ket{n'}\bra{n'}\op r\ket{n}=\bra{n}\op r^2\ket{n}
\end{equation}
ma allora
\begin{equation}
	\abs{\bra{n'}\op r\ket{n}}\leq\sqrt{\bra{n}\op r^2\ket{n}}=\sqrt{\frac12n^2[5n^2-3l(l+1)]}.
\end{equation}
Per grandi valori di $n$ dunque questo valore è asintotico a $n^2$, mentre se prendiamo stati con $n$ basso l'elemento di perturbazione sarà piccolo rispetto agli autovalori, e varrà la condizione \eqref{eq:condizione-approssimazione-perturbazione-2d}.\footnote{Ovviamente dovremo scegliere, nell'applicare la teoria, due livelli energetici differenti, altrimenti nella \eqref{eq:condizione-approssimazione-perturbazione-2d} troveremmo $\abs{E_n-E_n}=0$.}
Calcoliamo dunque gli elementi di matrice per i primi valori di $n$.

Per $n=1$ abbiamo lo stato fondamentale, che è simmetrico radialmente: allora, poich\'e non esistono direzioni ``privilegiate'', vediamo subito che $\bra{1,0,0}\op x_3\ket{1,0,0}=0$.
Al primo ordine, dunque, la perturbazione non influenza il sistema

Per $n=2$ abbiamo quattro stati, cioè l'insieme $\{\ket{2,0,0},\ket{2,1,0},\ket{2,1,-1},\ket{2,1,1}\}$ che prendiamo come base per calcolare i sedici elementi di matrice di $\op H'$.
Possiamo ridurre subito il numero di calcoli da fare notando che $\op H'$ è reale e hermitiano, perciò simmetrico.
Sempre per il discorso per cui non esistono direzioni privilegiate in stati a simmetria radiale, vale anche $\bra{2,0,0}\op x_3\ket{2,0,0}$.
Questo chiaramente vale per ogni componente di $\op{\vec x}$ in qualsiasi stato con $l=0$.
Detto $\ket{\psi}$ un tale stato, sarà combinazione lineare di stati del tipo $\ket{n,0,0}$, perch\'e se $l=0$ allora necessariamente $m=0$; quindi $\ket{\psi}=\sum_{n\in\N}a_n\ket{n,0,0}$.
Chiaramente vale $\op{\vec L}^2\ket{\psi}=0$: allora $\op x_i\ket{\psi}$ è ancora un suo autostato?
Poich\'e, come già visto, $\bra{\psi}\op x_i\ket{\psi}=0$, si ha che $\op x_i\ket{\psi}$ si trova in un sottospazio ortogonale a $\ket{\psi}$, in cui quindi $l\ne 0$.
Dopotutto, sappiamo che le tre armoniche sferiche con $l=1$ sono combinazioni lineari (linearmente indipendenti) di $x_1,x_2,x_3$ quindi possiamo, al contrario, scrivere $\op{\vec x}$ come combinazione lineare delle tre funzioni $Y_{1,0}$, $Y_{1,-1}$ e $Y_{1,1}$.
Allora $\op x_i\ket{n,0,0}$ è autostato di $\op{\vec L}^2$ con $l=1$.

Sappiamo inoltre che $\op H_0$ è invariante per parità, ossia $[\op H_0,\op I]=0$: allora i due operatori avranno autostati simultanei.
Infatti la parte radiale è simmetrica, e $\op I$ agisce soltanto sul fattore angolare della funzione d'onda come $\op IY_{l,m}=(-1)^lY_{l,m}$.
Di conseguenza abbiamo, dato che gli operatori $\op x_i$ sono dispari ossia $\adj{\op I}\op x_i\op I=-\op x_i$, che
\begin{equation}
	\bra{2,0,0}\op x_i\ket{2,0,0}=
	\bra{2,0,0}\op I^2\op x_i\op I^2\ket{2,0,0}=
	(-1)^{2l}\bra{2,0,0}\op I\op x_i\op I\ket{2,0,0}=
	-\bra{2,0,0}\op x_i\ket{2,0,0}
\end{equation}
da cui $\bra{2,0,0}\op x_i\ket{2,0,0}$, come ulteriore riprova.
Con lo stesso ragionamento troviamo che tra stati con uguale $l$ il valor medio di $\op x_i$ è nullo.

Ora, gli stati $\ket{2,0,0}$ e $\ket{2,1,0}$ hanno valori di $l$ differenti, e in particolare il secondo contiene nella funzione d'onda il termine $e^{i\phi}$, che integrato in $\phi$ dà zero.
Più in generale, per $n,l$ qualsiasi, $\op x_3\ket{m}$ è ancora autostato di $\op L_3$ (ma non necessariamente uguale, ma con lo stesso autovalore), poich\'e $[\op L_3,\op x_3]=0$.
Se $m_1\ne m_2$ allora si ha necessariamente che $\bra{2,1,m_1}\op x_3\ket{2,1,m_2}=0$ e in modo simile per gli altri stati.

Riassumendo, abbiamo trovato che alcune transizioni tra autostati sono proibite: abbiamo delle \emph{regole di selezione} con gli operatori $\op{\vec L}^2$, $\op L_3$ e $\op I$ che ci permettono di semplificare in modo veloce gli elementi di matrice di $\op H'$.
Rimane, infine, soltanto l'elemento $\bra{2,0,0}\op x_3\ket{2,1,0}$, che (insieme al suo simmetrico) è l'unico non nullo e vale $3r_B$.
Di conseguenza, la matrice che rappresenta l'hamiltoniano perturbato dal campo elettrico in questo sottospazio con $n=2$ è
\begin{equation}
	\begin{pmatrix}
		E_2				&3r_B\mathcal E	&0		&0	\\
		3r_B\mathcal E	&E_2			&0		&0	\\
		0				&0				&E_2	&0	\\
		0				&0				&0		&E_2\\
	\end{pmatrix}
\end{equation}
che ha come autovalori $E_2$, doppio (evidentemente per gli stati $\ket{2,1,1}$ e $\ket{2,1,-1}$), e $E_2\pm 3r_B\mathcal E$ per gli stati $\frac{\ket{2,0,0}\pm\ket{2,1,0}}{\sqrt{2}}$.
Quelli che prima erano quattro stati nello stesso livello (degenere) di energia per effetto della perturbazione ora formano tre livelli distinti.
La perturbazione, quindi, al primo ordine causa la rottura della degenerazione del livello di energia, per momenti angolari differenti con $m=0$.

\section{Perturbazioni indipendenti dal tempo}
\subsection{Caso non degenere}
Passiamo ora ad un approccio più generale, con un sistema descritto da un'hamiltoniano $\op H^{(0)}$ non degenere, di cui conosciamo lo spettro $\{E^{(0)}_n\}$, a cui è aggiunta una perturbazione $\epsilon\op V$, nessuno dei due dipendente esplicitamente dal tempo.
Preso l'insieme (che è completo) degli autostati $\ket{n^{(0)}}$ di $\op H^{(0)}$, possiamo esprimere uno stato perturbato $\ket{n}$ qualsiasi in tale base come
\begin{equation}
	\ket{n}=\sum_mc_{nm}\ket{m^{(0)}}
	\label{eq:perturbazione-autostati-non-degeneri}
\end{equation}
per cui l'equazione di Schr\"odinger (con l'hamiltoniano perturbato) $\op H\ket{n}=E_n\ket{n}$ si scrive come
\begin{equation}
	\sum_mc_{nm}(\op H^{(0)}+\epsilon\op V)\ket{m^{(0)}}=\sum_mc_{nm}E_n\ket{m^{(0)}}
\end{equation}
e poich\'e $\op H^{(0)}\ket{m^{(0)}}=E^{(0)}_m\ket{m^{(0)}}$, moltiplicando a sinistra per $\bra{k^{(0)}}$ otteniamo
\begin{equation}
	\begin{aligned}
		\sum_mc_{nm}(E^{(0)}_m\braket{k^{(0)}}{m^{(0)}}+\epsilon\bra{k^{(0)}}\op V\ket{m^{(0)}})&=\sum_mc_{nm}E_n\braket{k^{(0)}}{m^{(0)}}\\
		\sum_mc_{nm}(E^{(0)}_m\delta_{km}+\epsilon V_{km})&=\sum_mc_{nm}E_n\delta_{km}\\
		c_{nk}E^{(0)}_k+\epsilon\sum_mc_{nm}V_{km}&=c_{nk}E_n\\
		c_{nk}(E_n-E^{(0)}_k)&=\epsilon\sum_mc_{nm}V_{km}
	\end{aligned}
	\label{eq:perturbazione-non-degenere}
\end{equation}
definendo l'elemento di matrice $V_{km}\defeq\bra{\psi_k}\op V\ket{\psi_n}$.
Immaginiamo ora di poter sviluppare i termini della perturbazione in serie come
\footnote{
	Avevamo scritto il termine di perturbazione come $\epsilon\op V$, in vista di questi sviluppi in serie, per indicare che --- essendo $\op V$ una piccola correzione di $\op H^{(0)}$ --- i termini $c^{(1)}$ ed $E^{(1)}$ sono dello stesso ordine di grandezza della perturbazione, e le quantità $c^{(2)}$ ed $E^{(2)}$ di un ordine maggiore, e cos\`i via.
}
\begin{equation}
	\begin{gathered}
		c_{ij}=c^{(0)}_{ij}+\epsilon c^{(1)}_{ij}+\epsilon^2 c^{(2)}_{ij}+o(\epsilon^2)\\
		E_{i}=E^{(0)}_i+\epsilon E^{(1)}_i+\epsilon^2 E^{(2)}_i+o(\epsilon^2)
	\end{gathered}
\end{equation}
All'ordine zero abbiamo innanzitutto che gli $E^{(0)}_i$ sono proprio gli autovalori di $\op H^{(0)}$ (questo giustifica la notazione usata), poich\'e la perturbazione lascia, a quest'ordine, invariati i livelli energetici.
Analogamente dalla \eqref{eq:perturbazione-non-degenere}, trascurando tutti i termini $o(1)$, risulta $c^{(0)}_{nk}(E^{(0)}_n-E^{(0)}_k)=0$ da cui $c^{(0)}_{nk}=\delta_{nk}$.
Al primo ordine, invece, sostituendo gli sviluppi nella \eqref{eq:perturbazione-non-degenere} otteniamo
\begin{equation}
	(\delta_{nk}+\epsilon c^{(1)}_{nk})(E^{(0)}_n+\epsilon E^{(1)}_n-E^{(0)}_k)=\epsilon\sum_m(\delta_{nm}+\epsilon c^{(1)}_{nm})V_{km}\\
\end{equation}
che trascurando tutti i termini $o(\epsilon)$ si riduce a
\begin{equation}
	\delta_{nk}(E^{(0)}_n+\epsilon E^{(1)}_n-E^{(0)}_k)+\epsilon c^{(1)}_{nk}(E^{(0)}_n-E^{(0)}_k)=\epsilon\sum_m\delta_{nm}V_{km}
\end{equation}
e il secondo membro è poi uguale a $\epsilon V_{kn}$.
Distinguiamo ora i casi
\begin{itemize}
	\item $k=n$, per cui si ha $E^{(1)}_n=V_{nn}=\bra{n^{(0)}}\op V\ket{n^{(0)}}$
	\item $k\ne n$, per cui risulta
		\begin{equation}
			\epsilon c^{(1)}_{nk}(E^{(0)}_n-E^{(0)}_k)=\epsilon V_{kn}\qqq c^{(1)}_{nk}=\frac{V_{kn}}{E^{(0)}_n-E^{(0)}_k}.
			\label{eq:perturbazione-non-degenere-primo-ordine}
		\end{equation}
		La non degenerazione di $\op H^{(0)}$ assicura che $E^{(0)}_k\ne E^{(0)}_n$ quando $k\ne n$.
		Notiamo come questa formula assomigli alla \eqref{eq:condizione-approssimazione-perturbazione-2d} trovata nel caso bidimensionale.
\end{itemize}
Al primo ordine, dunque, la correzione da apportare agli autovalori corrisponde al valore di aspettazione della perturbazione nell'autostato imperturbato corrispondente all'autovalore.
Le correzioni agli autostati sono date invece dai coefficienti $c^{(1)}_{nm}$, che abbiamo trovato per $n\ne m$, mentre il coefficiente $c^{(1)}_{nn}$ è arbitrario e si calcola normalizzando lo stato (sempre a meno di termini $o(\epsilon)$): risulta quindi dalla \eqref{eq:perturbazione-autostati-non-degeneri}
\begin{equation}
	\ket{n^{(1)}}=c^{(1)}_{nn}\ket{n^{(0)}}+\sum_{m\ne n}\frac{V_{kn}}{E^{(0)}_n-E^{(0)}_m}\ket{m^{(0)}}.
	\label{eq:perturbazione-autostati-non-degeneri-primo-ordine}
\end{equation}

\section{Perturbazioni dipendenti dal tempo}
Considieriamo ora un hamiltoniano $\op H(t)=\op H^{(0)}+\op V(t)$, con $\op H^{(0)}$ indipendente (esplicitamente) dal tempo.
Supponiamo di avere uno stato stazionario $\ket{E_i^{(0)}}$ presente da un tempo molto remoto (anche $t=-\infty$) e che la perturbazione $\op V(t)$ si «attivi» ad un certo istante $t_0$.
Siamo interessati a calcolare la probabilità che il sistema si trovi in un autostato (fissato) $\ket{E_f^{(0)}}$ di $\op H^{(0)}$ per effetto della perturbazione: anche se due autostati di $\op H^{(0)}$ sono ortogonali, per cui si avrebbe una probabilità di transizione tra i due nulla, per effetto di $\op V(t)$ essi non sono più stazionari quindi anche le loro probabilità cambiano nel tempo, e non si possono più considerare a priori ortogonali.
Dovremo quindi caratterizzare l'evoluzione temporale dello stato $\ket{E_i^{(0)}}$ per effetto di $\op H(t)$: chiamando lo stato progredito $\ket{E_i(t)}$, la probabilità di passare dallo stato iniziale a quello finale che abbiamo fissato è, istante per istante, data da $\abs{\braket{E_i(t)}{E_f}}^2$.
All'hamiltoniano $\op H^{(0)}$, indipendente dal tempo, possiamo associare l'operatore di evoluzione temporale
\begin{equation}
	\op U_0(t,t_0)=\exp\bigg(-\frac{i}{\hbar}(t-t_0)\op H^{(0)}\bigg)
\end{equation}
Ci affidiamo dunque al postulato secondo il quale gli stati evolvono secondo l'equazione differenziale\footnote{Naturalmente anche $\op U_0(t,t_0)$ soddisfa quest'equazione, con $\op H^{(0)}$ al posto di $\op H(t)$, se poniamo la naturale condizione al contorno $U_0(t_0,t_0)=1$; per sistemi indipendenti dal tempo è equivalente assumere come postulato l'equazione differenziale o direttamente l'espressione dell'operatore.}
\begin{equation}
	i\hbar\drp{}{t}\ket{E_i(t)}=\op H(t)\ket{E_i(t)}.
	\label{eq:evouzione-temporale-autostato-iniziale}
\end{equation}
Cerchiamo di trovare, stavolta con delle inevitabili approssimazioni, un operatore $\op U(t,t_0)$ che descriva l'evoluzione del sistema nel tempo, ossia in questo caso tale che $\op U(t,t_0)\ket{E_i(t_0)}=\ket{E_i(t)}$ (in cui $\ket{E_i(t_0)}$ è ancora lo stato iniziale, cioè è proprio $\ket{E_i^{(0)}}$).
Per la \eqref{eq:evouzione-temporale-autostato-iniziale} tale operatore dovrà quindi soddisfare l'equazione
\begin{equation}
	i\hbar\drp{\op U(t,t_0)}{t}=\op H(t)\op U(t,t_0).
\end{equation}
La probabilità di transizione tra i due stati è dunque
\begin{equation}
	P_{fi}=\abs{\braket{E_f^{(0)}}{E_i(t)}}^2=\abs{\bra{E_f^{(0)}}\op U(t,t_0)\ket{E_i^{(0)}}}^2=\abs{\bra{E_f^{(0)}}\adj{\op U_0}(t,t_0)\op U(t,t_0)\ket{E_i^{(0)}}}^2
\end{equation}
poich\'e $\op U_0(t,t_0)\ket{E_f^{(0)}}=e^i\phi(t)\ket{E_f^{(0)}}$, essendo un autostato di $\op H^{(0)}$, e la fase non influisce sul modulo, quindi sulla probabilità di transizione.

\paragraph{Rappresentazione di interazione}
Definiamo a questo punto l'operatore
\begin{equation}
	\op{\mathbb U}(t,t_0)\defeq\adj{\op U_0}(t,t_0)\op U(t,t_0).
\end{equation}
Ora, per studiare l'evoluzione temporale degli stati o degli operatori, anzich\'e adottare lo schema di Heisenberg o di Schr\"odinger è conveniente usare la cosiddetta \emph{rappresentazione di interazione}, nella quale gli stati evolvono con l'operatore $\op{\mathbb U}$ mentre le osservabili con l'operatore $\op U_0$.
Ad esempio, presa un'osservabile $\xi$, il suo valor medio nel tempo è
\begin{equation}
	\avg{\xi(t)}=
	\bra{\psi(t)}\op\xi\ket{\psi(t)}=
	\bra{\psi}\adj{\op U}(t)\op\xi\op U(t)\ket{\psi}=
	\bra{\psi}\adj{\op{\mathbb U}}\adj{\op U_0}\op\xi\op U_0\op{\mathbb U}\ket{\psi}
	\label{eq:esempio-rappresentazione-interazione}
\end{equation}
per cui è utile assumere $\ket{\psi(t)}=\op{\mathbb U}(t,t_0)\ket{\psi(t_0)}$ e $\op\xi(t)=\adj{\op U}(t,t_0)\op\xi(t_0)\op U(t,t_0)$.

La derivata temporale dell'operatore $\op{\mathbb U}$ è inoltre
\begin{multline}
	\drp{}{t}\op{\mathbb U}=\drp{}{t}(\adj{\op U_0}\op U)=\drp{\adj{\op U_0}}{t}\op U+\adj{\op U_0}\drp{\op U}{t}=\frac{i}{\hbar}\adj{\op U_0}\op H^{(0)}\op U+\adj{\op U_0}\bigg(-\frac{i}{\hbar}\op H\op U\bigg)=\\=
	-\frac{i}{\hbar}\adj{\op U_0}(-\op H^{(0)}+\op H)\op U=-\frac{i}{\hbar}\adj{\op U_0}\op V\op U=-\frac{i}{\hbar}\adj{\op U_0}\op V\op U_0\op U.
\end{multline}
Chiamando dunque $\op{\mathbb V}\defeq\adj{\op U_0}\op V\op U_0$ abbiamo dunque l'equazione
\begin{equation}
	i\hbar\drp{\op{\mathbb U}}{t}=\op{\mathbb V}\op{\mathbb U}.
	\label{eq:evoluzione-temporale-perturbazione-dipendende-tempo}
\end{equation}
Notiamo che se $\op V=0$ ritroviamo correttamente che il sistema è stazionario.
Soffermiamoci un attimo su questa equazione.
Possiamo anche in questo caso immaginarci uno sviluppo formale di $\op{\mathbb U}$ del tipo $\op{\mathbb U}^{(0)}+\epsilon\op{\mathbb U}^{(1)}+\epsilon^2\op{\mathbb U}^{(2)}+o(\epsilon^2)$, come spesso abbiamo già fatto per altre grandezze.
Poich\'e all'ordine zero si ha $\op{\mathbb U}^{(0)}=1$, dato che sia $\op U$ che $\op U_0$ si riducono all'identità, da questa equazione ritroviamo che all'ordine zero $\op{\mathbb V}=0$, come giustamente deve risultare.
Al primo ordine, al primo membro troviamo $\drp{}{t}(\op{\mathbb U}^{(0)}+\epsilon\op{\mathbb U}^{(1)})=\drp{}{t}(1+\epsilon\op{\mathbb U}^{(1)})=\epsilon\drp{}{t}\op{\mathbb U}^{(1)}$; al secondo membro, invece, ricordando che $\op{\mathbb V}$ è già del primo ordine (scriviamo $\epsilon\op{\mathbb V}$ per tenerne conto) abbiamo $\epsilon\op{\mathbb V}(1+\epsilon\op{\mathbb U}^{(1)})=\epsilon\op{\mathbb V}+\epsilon^2\op{\mathbb V}\op{\mathbb U}^{(1)}$ ma il termine $\epsilon^2\op{\mathbb V}\op{\mathbb U}^{(1)}$ è già del secondo ordine ed è quindi da trascurare.
Approssimando al primo ordine la \eqref{eq:evoluzione-temporale-perturbazione-dipendende-tempo} si può ridurre dunque alla forma
\begin{equation}
	i\hbar\drp{}{t}\op{\mathbb U^{(1)}}=-\op{\mathbb V}.
	\label{eq:evoluzione-temporale-perturbazione-dipendende-tempo-primo-ordine}
\end{equation}
Saremmo tentati a questo punto di assumere che anche $\op{\mathbb U}(t,t_0)$ si possa scrivere, alla luce di questi risultati, come l'esponenziale
\begin{equation}
	\exp\bigg(\!\!-\frac{i}{\hbar}\int_{t_0}^t\op{\mathbb V(t')}\,\dd t'\bigg),
\end{equation}
assimilando dunque anche il caso dipendente dal tempo a quello indipendente con $\op U_0$, con $\op{\mathbb V}$ al posto dell'hamiltoniano $\op H^{(0)}$: se infatti $\op{\mathbb V}$ fosse indipendente dal tempo tale integrale sarebbe proprio $(t-t_0)\op{\mathbb V}$.
Un'analisi più approfondita, considerando anche i termini di ordine successivo, mostra invece che non è questo il caso.

\paragraph{Serie di Dyson}
Possiamo integrare la \eqref{eq:evoluzione-temporale-perturbazione-dipendende-tempo} ottenendo l'equazione, del tutto equivalente,
\begin{equation}
	\op{\mathbb U}(t,t_0)=1-\frac{i}{\hbar}\int_{t_0}^t\op{\mathbb V}(t',t_0)\op{\mathbb U}(t',t_0)\,\dd t'.
	\label{eq:evoluzione-temporale-perturbazione-dipendente-tempo-integrata}
\end{equation}
Nell'integrale appare ancora $\op{\mathbb U}$, per cui dovremmo escogitare un modo per approssimare la funzione.
Possiamo ricavare invece una soluzione formale nel modo seguente.
Riprendiamo la \eqref{eq:evoluzione-temporale-perturbazione-dipendende-tempo} agli ordini successivi al primo: trascurando i termini $o(\epsilon^2)$ troviamo
\begin{equation}
	\drp{}{t}(1+\epsilon\op{\mathbb U}^{(1)}+\epsilon^2\op{\mathbb U}^{(2)})=-\frac{i}{\hbar}\epsilon\op{\mathbb V}(1+\epsilon\op{\mathbb U}^{(1)})
\end{equation}
ma per i risultati già ottenuti agli ordini di approssimazione precedenti questa si riduce a
\begin{equation}
	\drp{}{t}(\epsilon^2\op{\mathbb U}^{(2)})=-\frac{i}{\hbar}\epsilon^2\op{\mathbb V}\op{\mathbb U}^{(1)}
	\qqq
	\drp{}{t}\op{\mathbb U}^{(2)}=-\frac{i}{\hbar}\op{\mathbb V}\op{\mathbb U}^{(1)}.
	\label{eq:evoluzione-temporale-perturbazione-dipendente-tempo-secondo-ordine}
\end{equation}
La formula ottenuta è della stessa forma della \eqref{eq:evoluzione-temporale-perturbazione-dipendende-tempo-primo-ordine}, con gli ordini opportunamente scalati.
Mostriamo infatti per induzione che vale l'equazione
\begin{equation}
	\drp{}{t}\op{\mathbb U}^{(n)}=-\frac{i}{\hbar}\op{\mathbb V}\op{\mathbb U}^{(n-1)}.
	\label{eq:evoluzione-temporale-perturbazione-dipendente-tempo-ordine-generico}
\end{equation}
Sicuramente vale per $n=0,1,2$ come già visto.
Supponendo che sia vera per $n$, per $n+1$ risulta
\begin{equation}
	\drp{}{t}\sum_{k=0}^{n+1}\epsilon^k\op{\mathbb U}^{(k)}=-\frac{i}{\hbar}\epsilon\op{\mathbb V}\sum_{k=0}^{n+1}\epsilon^k\op{\mathbb U}^{(k)}
\end{equation}
ma ancora, sfruttando la \eqref{eq:evoluzione-temporale-perturbazione-dipendente-tempo-ordine-generico} per i termini da $k=0$ a $k=n$ e trascurando i termini $o(\epsilon^{n+1})$ si ottiene proprio (dividendo per $\epsilon^{n+1}$ i due membri)
\begin{equation}
	\drp{}{t}\op{\mathbb U}^{(n+1)}=-\frac{i}{\hbar}\op{\mathbb V}\op{\mathbb U}^{(n)}
\end{equation}
che prova l'ipotesi di induzione.

Per ottenere tutti i termini dello sviluppo in serie di $\op{\mathbb U}(t,t_0)$ è sufficiente allora integrare la \eqref{eq:evoluzione-temporale-perturbazione-dipendente-tempo-ordine-generico} per ogni $n$.
Nell'integrale dell'ordine $n$-esimo appare l'integranda $\op{\mathbb V}(t')\op{\mathbb U}^{(n-1)}(t',t_0)$, che possiamo riscrivere sostituendo a $\op{\mathbb U}^{(n-1)}$ il risultato ottenuto integrando la \eqref{eq:evoluzione-temporale-perturbazione-dipendente-tempo-ordine-generico} dal tempo $t_0$ a $t'$: in tale integrale si avrà dunque il fattore $\op{\mathbb V}(t'')\op{\mathbb U}^{(n-2)}(t'',t_0)$ (per $t_0<t''<t'$) in cui possiamo sostituire a $\op{\mathbb U}^{(n-2)}$ un altro integrale, in modo analogo, procedendo in questo modo (per ciascun termine dello sviluppo) fino ad arrivare all'ordine zero della serie.
Troviamo cos\`i per $\op{\mathbb U}(t,t_0)$ l'espressione formale
\begin{multline}
	\op{\mathbb U}(t,t_0)=
	1
	-\frac{i}{\hbar}\int_{t_0}^t\op{\mathbb V}(t',t_0)\,\dd t'
	+\bigg(\!\!-\frac{i}{\hbar}\bigg)^2\int_{t_0}^t\int_{t_0}^{t'}\op{\mathbb V}(t',t_0)\op{\mathbb V}(t'',t_0)\,\dd t''\,\dd t'+\\
	+\bigg(\!\!-\frac{i}{\hbar}\bigg)^n\int_{t_0}^t\cdots\int_{t_0}^{t^{(n)}}\op{\mathbb V}(t',t_0)\cdots\op{\mathbb V}(t^{(n)},t_0)\,\dd t^{(n)}\cdots\dd t''\,\dd t'
	\label{eq:serie-dyson}
\end{multline}
detta \emph{serie di Dyson}.
È essenziale in quest'ultima equazione l'ordinamento temporale dei vari operatori, con $t>t'>\cdots>t^{(n)}>t_0$.

Per semplificare questa espressione, introduciamo un operatore $\mathcal T$ che, applicato ad un prodotto di operatori, ne restituisce il prodotto ordinato nel tempo: ad esempio, con due operatori $\op A(t)$ e $\op B(t)$ si ha
\begin{equation}
	\mathcal T\big(\op A(t_1)\op B(t_2)\big)=
	\begin{cases*}
		\op A(t_1)\op B(t_2)	&se $t_1<t_2$\\
		\op B(t_2)\op A(t_1)	&se $t_1>t_2$
	\end{cases*}
	\label{eq:operatore-prodotto-ordinato-tempo}
\end{equation}
Si generalizza facilmente al caso di un numero qualsiasi di operatori nel prodotto.
\footnote{
	Siccome questo operatore ``mette a posto'' da solo l'ordine della somma, possiamo considerare a tutti gli effetti come se gli operatori su cui è applicato commutassero.
}
Consideriamo ora i due integrali
\begin{equation}
	\begin{aligned}
		I_n&=\int_{t_0}^t\int_{t_0}^t\cdots\int_{t_0}^t\mathcal T\big(\op{\mathbb V}(t')\op{\mathbb V}(t'')\cdots\op{\mathbb V}(t^{(n)})\big)\,\dd t^{(n)}\cdots\dd t''\,\dd t'\text{ e}\\
		S_n&=\int_{t_0}^t\int_{t_0}^{t'}\cdots\int_{t_0}^{t^{(n)}}\mathcal T\big(\op{\mathbb V}(t')\op{\mathbb V}(t'')\cdots\op{\mathbb V}(t^{(n)})\big)\,\dd t^{(n)}\cdots\dd t''\,\dd t':
	\end{aligned}
\end{equation}
possiamo suddividere l'insieme di integrazione, che è qui $[t_0,t]^n$, nei vari sottoinsiemi $t^{(1)}>t^{(2)}>\cdots>t^{(n)}$, $t^{(2)}>t^{(1)}>\cdots>t^{(n)}$ e cos\`i via, trovandoci con $n!$ di questi sottoinsiemi (ognuno con un diverso ordinamento dei $t^{(k)}$).
\footnote{
	Un esempio di tutto ciò appare indispensabile.
	Prendiamo il secondo ``ordine'', cioè $I_2$, che è un integrale sul quadrato $[t_0,t]\times[t_0,t]$.
	Possiamo suddividerlo nei due triangoli in cui, rispettivamente, $t^{(1)}>t^{(2)}$ e $t^{(1)}<t^{(2)}$.
	I due integrali coincidono, perch\'e l'integranda (qui $\mathcal T$ applicato al prodotto dei $\op{\mathbb V}$, che sono tutti la stessa funzione) è simmetrica rispetto alle sue variabili, nel senso che le possiamo liberamente scambiare grazie all'operatore $\mathcal T$ di ordinamento, quindi abbiamo $I_2=2S_2$.
	Analogamente, possiamo suddividere il cubo $[t_0,t]\times[t_0,t]\times[t_0,t]$ in sei tetraedri $\{t^{(1)}>t^{(2)}>t^{(3)}\}$, $\{t^{(1)}>t^{(3)}>t^{(2)}\}$, $\{t^{(2)}>t^{(1)}>t^{(3)}\}$, $\{t^{(2)}>t^{(3)}>t^{(1)}\}$, $\{t^{(3)}>t^{(1)}>t^{(2)}\}$ e $\{t^{(3)}>t^{(2)}>t^{(1)}\}$.
	In ciascun sottointervallo si ottiene sempre $S_3$, per cui $I_3=6S_3=3!S_3$.
	Generalizzando possiamo vedere che $I_n=n!S_n$: ogni insieme $[t_0,t]^n$ si suddivide in questo modo in un numero di sottoinsiemi pari al numero delle permutazioni possibili dei $\{t',t'',\dots,t^{(n)}\}$, ossia dei primi $n$ numeri naturali, che è $n!$.
}
In virtù della simmetria in tutte le variabili della funzione integranda $\mathcal T\big(\op{\mathbb V}(t')\cdots\op{\mathbb V}(t^{(n)})\big)$, data dall'operatore $\mathcal T$, in ciascuno di questi sottoinsiemi l'integrale è lo stesso e vale $S_n$, perciò otteniamo
\begin{equation}
	\begin{split}
		\op{\mathbb U}(t,t_0)
		&=\sum_{n=0}^{+\infty}\bigg[\frac1{n!}\bigg(\!\!-\frac{i}{\hbar}\bigg)^n\int_{t_0}^t\cdots\int_{t_0}^t\mathcal T\big(\op{\mathbb V}(t')\cdots\op{\mathbb V}(t^{(n)})\big)\,\dd t^{(n)}\cdots\dd t'\bigg]=\\
		&=\sum_{n=0}^{+\infty}\bigg[\frac1{n!}\bigg(\!\!-\frac{i}{\hbar}\bigg)^n\mathcal T\int_{t_0}^t\op{\mathbb V}(t')\,\dd t'\cdots\int_{t_0}^t\op{\mathbb V}(t^{(n)})\,\dd t^{(n)}\bigg]=\\
		&=\mathcal T\sum_{n=0}^{+\infty}\frac1{n!}\bigg(\!\!-\frac{i}{\hbar}\bigg)^n\bigg(\int_{t_0}^t\op{\mathbb V}(t')\,\dd t'\bigg)^n=\\
		&=\mathcal T\exp\bigg(-\frac{i}{\hbar}\int_{t_0}^t\op{\mathbb V}(t')\,\dd t'\bigg)
	\end{split}
	\label{eq:serie-dyson-alt}
\end{equation}
come maniera più compatta per scrivere la serie di Dyson.

Ora, ricordando che $\op{\mathbb V}(t)$ è stato definito come $\adj{\op U}_0(t,t_0)\op V(t)\op U_0(t,t_0)$, tornando alla probabilità di transizione abbiamo se $\ket{E_f^{(0)}}\ne\ket{E_i^{(0)}}$ che
\begin{equation}
	\begin{split}
        P_{fi}&=\abs[\bigg]{\bra{E_f^{(0)}}1-\frac{i}{\hbar}\int_{t_0}^t\op{\mathbb V}(t')\,\dd t'\ket{E_i^{(0)}}}^2=\\
        &=\frac1{\hbar^2}\abs[\bigg]{\bra{E_f^{(0)}}\int_{t_0}^t\op{\mathbb V}(t')\,\dd t'\ket{E_i^{(0)}}}^2=\\
        &=\frac1{\hbar^2}\abs[\bigg]{\int_{t_0}^t\bra{E_f^{(0)}}\adj{\op U}_0(t',t_0)\op V(t')\op U_0(t',t_0)\ket{E_i^{(0)}}\,\dd t'}^2=\\
        &=\frac1{\hbar^2}\abs[\bigg]{\int_{t_0}^te^{-\frac{i}{\hbar}(E_f^{(0)}-E_i^{(0)})(t'-t_0)}\bra{E_f^{(0)}}\op V(t')\ket{E_i^{(0)}}\,\dd t'}^2
	\end{split}
	\label{eq:probabilita-transizione-autostati-dipendente-tempo}
\end{equation}
poich\'e gli stati $\ket{E_i^{(0)}}$ e $\ket{E_f^{(0)}}$ sono stazionari.

\section{Perturbazioni periodiche nel tempo}
Studiamo ora il sistema di un atomo su cui incide una radiazione elettromagnetica monocromatica.
In meccanica classica la funzione hamiltoniana per una particella di carica $q$ soggetta ai potenziali $\Phi$ e $\vec A$ è
\begin{equation}
	\mathcal H=\frac1{2m}\bigg(\vec p-\frac{q}{c}\vec A\bigg)^2+q\Phi
	\label{eq:hamiltoniana-potenziale-elettromagnetico}
\end{equation}
che trasportiamo in versione quantistica, come al solito, sostituendo opportunamente gli operatori alle varie osservabili.
Nello sviluppare il quadrato, troviamo il prodotto scalare tra $\vec p$ e $\vec A$, i cui operatori non commutano: per risolvere il problema, notiamo che nella rappresentazione di Schr\"odinger
\begin{equation}
	[\op p_i,\op A_i]f=-i\hbar\bigg[\drp{}{x_i},A_i\bigg]f=\drp{}{x_i}(A_if)-A_i\drp{f}{x_i}=-i\hbar\drp{A_i}{x_i}f=-i\hbar f\div\vec A
	\label{eq:commutatore-impulso-potenziale-vettore}
\end{equation}
dunque ci basta scegliere il gauge in cui $\div\vec A=0$ affinch\'e i due commutino.
In questo caso l'hamiltoniano della particella nel campo elettromagnetico è
\begin{equation}
	\op H=\frac{\op{\vec p}^2}{2m}-\frac{q}{mc}\scalar{\op{\vec A}}{\op{\vec p}}+\frac{q^2}{2mc^2}\op{\vec A}^2-q\op{\Phi}
	\label{eq:hamiltoniano-campo-elettromagnetico}
\end{equation}
e come ulteriore approssimazione, se già il campo elettromagnetico può essere considerato come perturbazione del sistema significa che $\scalar{\op{\vec A}}{\op{\vec p}}$ è piccolo, quindi a maggior ragione $\op{\vec A}^2$ sarà trascurabile, quindi possiamo ignorare (al primo ordine di approssimazione) quest'ultimo termine.
Abbiamo dunque un hamiltoniano $\op H=\op H^{(0)}+\op V'(t)$ con
\footnote{Il potenziale scalare $\Phi$ è già incorporato in $\op H^{(0)}$.}
\begin{equation}
	\op V(t')=\frac{q}{mc}\scalar{\op{\vec p}}{\op{\vec A}}(t).
\end{equation}
Con esso la probabilità di transizione tra due autostati $\ket{i^{(0)}}$ e $\ket{f^{(0)}}$ di $\op H^{(0)}$, con energia rispettivamente $E_i^{(0)}$ e $E_f^{(0)}$, è data da
\begin{equation}
    P_{fi}=\frac1{\hbar^2}\abs[\bigg]{\int_{t_0}^t\frac{q}{mc}\bra{f^{(0)}}\scalar{\op{\vec p}}{\op{\vec A}}(t')\ket{i^{(0)}}e^{\frac{i}{\hbar}(E_f^{(0)}-E_i^{(0)})(t'-t_0)}\,\dd t'}^2=\\
\end{equation}
e se il campo elettromagnetico è un'onda monocromatica di frequenza, detto $\vec u$ il vettore di polarizzazione, allora
\begin{equation}
	\vec A(t)=\frac{A}2\Big[\vec u e^{i(\scalar{\vec k}{\vec x}-\omega t)}-\vec u^*e^{-i(\scalar{\vec k}{\vec x}-\omega t)}\Big].
	\label{eq:campo-elettromagnetico-monocromatico}
\end{equation}
Definiamo a questo punto
\begin{equation}
	\omega_{fi}\defeq\frac{E_f^{(0)}-E_i^{(0)}}{\hbar}
	\label{eq:frequenza-transizione}
\end{equation}
ottenendo
\begin{equation}
    P_{fi}=\bigg(\frac{q}{mc\hbar}\bigg)^2\abs[\bigg]{\int_{t_0}^t
		\bra{f^{(0)}}\scalar{\op{\vec p}}{\vec u}e^{i\scalar{\vec k}{\op{\vec x}}}\ket{i^{(0)}}e^{i(\omega_{fi}-\omega)(t'-t_0)}
		\bra{f^{(0)}}\scalar{\op{\vec p}}{\vec u}e^{-i\scalar{\vec k}{\op{\vec x}}}\ket{i^{(0)}}e^{i(\omega_{fi}+\omega)(t'-t_0)}
	\,\dd t'}^2
\end{equation}
e definendo le quantità
\begin{equation}
	F\defeq\frac{qA}{2mc\hbar}\bra{f^{(0)}}\scalar{\op{\vec p}}{\vec u}e^{i\scalar{\vec k}{\op{\vec x}}}\ket{i^{(0)}}
	\qeq
	G\defeq\frac{qA}{2mc\hbar}\bra{f^{(0)}}\scalar{\op{\vec p}}{\vec u^*}e^{-i\scalar{\vec k}{\op{\vec x}}}\ket{i^{(0)}}
\end{equation}
e integrando otteniamo l'espressione
\begin{equation}
    P_{fi}=\abs[\bigg]{F\frac{e^{i(\omega_{fi}-\omega)(t-t_0)}-1}{\omega_{fi}-\omega}+G\frac{e^{i(\omega_{fi}+\omega)(t-t_0)}-1}{\omega_{fi}+\omega}}^2.
\end{equation}
Ci aspettiamo che solo uno dei due termini contribuisca in gran parte al risultato, ossia solo il termine con $F$ o quello con $G$ siano ``grandi''.
In tal modo possiamo trascurare l'altro e, con esso, i termini di interferenza tra i due.
Se ad esempio $E_f^{(0)}>E_i^{(0)}$, vale a dire $\omega_{fi}>0$, può essere che $\omega_{fi}-\omega\approx 0$ per cui possiamo tenere in prima approssimazione solo il termine con $F$.
L'interpretazione fisica è la seguente: se i fotoni che portano il campo $\vec A$ hanno esattamente l'energia che separa i due livelli energetici $E_i^{(0)}$ e $E_f^{(0)}$, allora si ha proprio $\omega=\omega_{fi}$: un fotone viene assorbito e l'elettrone effettua una transizione dallo stato $\ket{i^{(0)}}$ a $\ket{f^{(0)}}$.
Al contrario, il caso in cui $\omega_{fi}=-\omega$ è quello in cui il fotone è emesso dall'atomo quando quest'ultimo perde energia, nel momento in cui un elettrone effettua una transizione da $\ket{i^{(0)}}$ a $\ket{f^{(0)}}$ questa volta con $E_f^{(0)}<E_i^{(0)}$, ossia in uno stato con energia minore.
Si ha dunque un fenomeno di \emph{assorbimento indotto} o \emph{emissione indotta}, in quanto l'elettrone non cambia stato spontaneamente.

Consideriamo dunque solo il primo termine, con $F$, in cui si ha
\begin{equation}
	\begin{split}
        \abs[\bigg]{\frac{e^{i(\omega_{fi}-\omega)(t-t_0)}-1}{\omega_{fi}-\omega}}^2
        &=\abs[\big]{e^{\frac{i}2(\omega_{fi}-\omega)(t-t_0)}}^2\abs[\bigg]{\frac{e^{\frac{i}2(\omega_{fi}-\omega)(t-t_0)}-e^{-\frac{i}2(\omega_{fi}-\omega)(t-t_0)}}{\omega_{fi}-\omega}}^2=\\
		&=\bigg(\frac2{\omega_{fi}-\omega}\bigg)^2\sin^2\frac{(\omega_{fi}-\omega)(t-t_0)}2.
	\end{split}
\end{equation}
Chiamiamo $\Omega\defeq\omega_{fi}-\omega$ e $T\defeq t-t_0$: se $T\gg\frac1{\Omega}$, o più formalmente per $T\to+\infty$, si ha
\begin{equation}
	\lim_{T\to+\infty}\frac1{T}\bigg(\frac{\sin\frac{\Omega T}2}{\frac{\Omega}2}\bigg)^2=2\pi T\delta(\Omega)
\end{equation}
che rappresenta la probabilità di transizioni per intervallo di tempo.
Più in generale, per una perturbazione periodica della forma
\begin{equation}
	\op V'(t)=\op B e^{-i\omega t}+\op B^* e^{i\omega t}
	\label{eq:perturbazione-periodica-generica}
\end{equation}
si ottiene l'espressione nota come \emph{regola d'oro di Fermi}:
\begin{equation}
	\lim_{T\to+\infty}\frac{P_{fi}}{T}=\frac{2\pi}{\hbar}\abs{\bra{f^{(0)}}\op B\ket{i^{(0)}}}^2\delta(E_f^{(0)}-E_i^{(0)}-\hbar\omega)
	\label{eq:regola-oro-fermi}
\end{equation}
che vale sia per il fenomeno di emissione che di assorbimento.

\section{Approssimazione di dipolo elettrico}
Continuiamo lo studio dell'atomo colpito da un'onda elettromagnetica monocromatica nel caso in cui la sua lunghezza d'onda sia comparabile con le dimensioni atomiche, come il raggio di Bohr.
In questo senso possiamo assumere che il vettore d'onda $\vec k$ non vari apprezzabilmente all'interno del sistema fisico considerato, per cui possiamo usare l'approssimazione
\begin{equation}
	e^{\pm i\scalar{\vec k}{\vec x}}\approx 1
\end{equation}
trascurando cos\`i la dipendenza spaziale dell'onda incidente.
Questa approssimazione è detta \emph{di dipolo elettrico}, per ragioni che vedremo in seguito.
Nel gauge scelto, il campo elettrico portato dall'onda è dato da
\begin{equation}
	\vec E(\vec x,t)=-\frac1{c}\drp{\vec A}{t}(\vec x,t)=-\frac1{c}\drp{}{t}\bigg[\frac{A}2\big(\vec ue^{-i\omega t}+\vec u^*e^{i\omega t}\big)\bigg]=
	\frac{iA\omega}{2c}\big(\vec ue^{-i\omega t}-\vec u^*e^{i\omega t}\big).
	\label{eq:campo-elettrico-approssimazione-dipolo}
\end{equation}
Prendiamo ora il caso di un elettrone in un atomo, con $q=e$: il termine $F$ definito precedentemente diventa, con questa approssimazione,
\begin{equation}
	F=-\frac{eA}{2mc\hbar}\bra{f^{(0)}}\scalar{\op{\vec p}}{\vec u}\ket{i^{(0)}}
\end{equation}
Dall'equazione per l'evoluzione temporale della posizione $\drp{}{t}\op{\vec x}=\frac{i}{\hbar}[\op H_0,\op{\vec x}]=\frac1{m}\op{\vec p}$ abbiamo inoltre
\begin{multline}
	F=-\frac{eA}{2\hbar c}\bra{f^{(0)}}\frac{i}{\hbar}[\op H_0,\vec x]\ket{i^{(0)}}=
	-\frac{ieA}{2\hbar^2c}\bra{f^{(0)}}\op H_0\scalar{\op{\vec x}}{\vec u}-\scalar{\op{\vec x}}{\vec u}\op H_0\ket{i^{(0)}}=\\=
	-\frac{ieA}{2\hbar^2c}(E_f^{(0)}-E_i^{(0)})\bra{f^{(0)}}\scalar{\op{\vec x}}{\vec u}\ket{i^{(0)}}
\end{multline}
dato che i due stati sono autostati di $\op H_0$.
Il vettore $-e\vec x$ è il vettore di dipolo elettrico, che qui ritroviamo come operatore e indichiamo più brevemente con $\op{\vec d}$, per cui
\begin{equation}
	F=\frac{iA\omega_{fi}}{2m\hbar}\bra{f^{(0)}}\scalar{\op{\vec d}}{\vec u}\ket{i^{(0)}}.
\end{equation}

Cos\`i come per l'effetto Stark, anche la perturbazione data da $\bra{f^{(0)}}\scalar{\op{\vec d}}{\vec u}\ket{i^{(0)}}$ ha importanti conseguenze dalle regole di selezione, perch\'e come vedremo alcune transizioni tra stati, al livello dell'approssimazione di dipolo, sono proibite a causa di certe simmetrie.
Innanzitutto $\op{\vec d}$ è un operatore dispari, poich\'e lo è $\op{\vec x}$, di conseguenza (come già visto nell'effetto Stark) siccome gli stati $\ket{f^{(0)}}$ e $\ket{i^{(0)}}$ hanno una parità definita nell'atomo di idrogeno, le transizioni tra stati della medesima parità sono proibite in quanto
\begin{equation}
	\bra{f^{(0)}}\scalar{\op{\vec d}}{\vec u}\ket{i^{(0)}}=
	\bra{f^{(0)}}\op I^2\scalar{\op{\vec d}}{\vec u}\op I^2\ket{i^{(0)}}=
	\bra{f^{(0)}}\op I\scalar{\op{\vec d}}{\vec u}\op I\ket{i^{(0)}}=
	-\bra{f^{(0)}}\scalar{\op{\vec d}}{\vec u}\ket{i^{(0)}}.
\end{equation}
Il termine successivo nell'approssimazione (il termine di \emph{quadrupolo elettrico}) è invece pari, perciò in tal caso sono proibite le transizioni tra stati di parità opposta: la transizione di dipolo e quella di quadrupolo sono quindi esclusive.
Troviamo anche delle regole di selezione con il momento angolare: poich\'e è un vettore, $\op{\vec d}$ ruota con la solita equazione $[\op L_i,\op d_j]=i\hbar\epsilon_{ijk}\op d_k$.
In particolare, risultano
\begin{equation}
	\begin{aligned}
		[\op L_3,\op d_1+i\op d_2]&=i\hbar(\op d_2-i\op d_i)=\hbar(\op d_1+i\op d_2)\\
		[\op L_3,\op d_1-i\op d_2]&=i\hbar(\op d_2+i\op d_1)=-\hbar(\op d_1-i\op d_2)\\
		[\op L_3,\op d_3]&=0
	\end{aligned}
	\label{eq:operatore-dipolo-scala-momento-angolare}
\end{equation}
quindi $\op d_+\defeq \op d_1+i\op d_2$ e $\op d_-\defeq\op d_1-i\op d_2$ si comportano proprio come gli operatori a scala della terza componente momento angolare.
Indicando gli stati con il loro numero quantico magnetico $m$, dalla terza di queste tre equazioni risulta
\begin{equation}
	0=\bra{m'}[\op L_3,\op d_3]\ket{m}=\bra{m'}\op L_3\op d_3-\op d_3\op L_3\ket{m}=\hbar(m'-m)\bra{m'}\op d_3\ket{m}
\end{equation}
perciò deve essere $\bra{m'}\op d_3\ket{m}=0$ se $m\ne m'$: ciò indica che se l'onda è polarizzata linearmente lungo l'asse di $\op L_3$ (l'asse $z$) allora sono proibite transizioni tra stati con diverso $m$.
Se l'onda è invece polarizzata circolarmente il vettore $\vec u$ di polarizzazione ha la forma $\vec u_\pm=\vec u_1\pm i\vec u_2$, perciò $\scalar{\op{\vec d}}{\vec u_\pm}=\op d_\pm$:
\begin{equation}
	\begin{aligned}
		\bra{m'}\op d_+\ket{m}&=\frac1{\hbar}\bra{m'}[\op L_3,\op d_+]\ket{m}=(m'-m-1)\bra{m'}\op d_+\ket{m}\\
		\bra{m'}\op d_-\ket{m}&=-\frac1{\hbar}\bra{m'}[\op L_3,\op d_-]\ket{m}=-(m'-m+1)\bra{m'}\op d_-\ket{m}
	\end{aligned}
\end{equation}
quindi la polarizzazione circolare proibisce transizioni tra stati con $m'\ne m\pm 1$.
Il quadrato del momento angolare fornisce ulteriori regole di selezione: sappiamo che $\op{\vec d}$, essendo proporzionale a $\op{\vec x}$, è a sua volta una combinazione lineare delle tre armoniche sferiche con $l=1$ (come abbiamo già visto studiando l'effetto Stark), perciò $\op{\vec d}\ket{n,l,m}$ è uno stato al cui numero $l$ è aggiunto 1.
Dalla regola di composizione dei momenti angolari segue quindi che il numero quantico $l$ del nuovo stato assume valori tra $\abs{l-1},l$ e $l+1$, perciò sono ammesse solo le transizioni tra stati con $\Delta l=0,\pm1$.
In particolare la transizione da $l=0$ a $l'=0$ è sempre proibita, dato che $\op d\ket{n,0,m}$ ha $l=1$; questa regola di selezione vale in realtà non solo nell'approssimazione di dipolo, ma più in generale.
Scegliendo infatti l'asse $\vec e_3$ come direzione di propagazione, ossia $\vec k=k\vec e_3$, il vettore $\vec u$ di polarizzazione è nel piano a esso ortogonale quindi $\scalar{\vec p}{\vec u}=p_1u_1+p_2u_2$.
Se ora $l=l'=0$ risulta anche qui che $\bra{n',0,0}(\op p_1u1+\op p_2u_2)e^{ik\op x_3}\ket{n,0,0}=0$ dato che, come già visto, $\op x_3$ porta $\ket{n,0,0}$ in uno stato con $l>0$ che è cos\'i ortogonale a $\ket{n',0,0}$.
Nel caso specifico dell'atomo di idrogeno, la parità di uno stato è data dal momento angolare, ed è $(-1)^l$, perciò le transizioni con $\Delta l=0$ risultano proibita in quanto tra stati con la stessa parità.
A causa della degenerazione dei livelli energetici, però, queste regole di selezione rimangono nascoste; diventano evidenti se invece all'atomo è applicato un campo elettrico esterno, in modo da rimuovere tali degenerazioni.

