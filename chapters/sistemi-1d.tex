\chapter{Sistemi unidimensionali}
In queto capitolo affronteremo una serie di esempi ``accademici'' di sistemi in una dimensione, cercando di risolvere dove possibile l'equazione di Schr\"odinger per la funzione d'onda e lo spettro di energia, o di trovare la migliore approssimazione sfruttando le tecniche acquisite.

\section{Particella libera}
Contrariamente al caso classico, in cui questo era il sistema più semplice da studiare, nel caso quantistico troviamo qualche complicazione.
L'hamiltoniano del sistema è semplicemente
\begin{equation}
	\op H=\frac1{2m}\op p^2.
	\label{eq:H-particella-libera}
\end{equation}
Notiamo immediatamente che $[\op p,\op H]=0$ essendo $\op H$ funzione unicamente di $\op p$, dunque le due osservabili sono compatibili.
L'impulso dunque è una quantità conservata: a questo corrisponde la simmetria (evidente) del sistema per traslazioni spaziali, che sono generate da $\op p$.
Preso un autostato $\ket{p}$ dell'impulso (di autovalore $p$), troviamo $\op H\ket{p}=\frac1{2m}\op p^2\ket{p}=\frac1{2m}p^2\ket{p}$.
D'altro canto, $\frac1{2m}p^2$ è l'autovalore dell'hamiltoniano dunque è l'energia $E$ del sistema; se fissiamo dunque questo valore, troviamo due autostati di $\op p$ aventi questa energia, che sono $\ket{p}$ e $\ket{-p}$.
Ciò significa che l'hamiltoniano è degenere, ossia un suo autostato $\ket{E}$ si scrive come $\alpha\ket{p}+\beta\ket{-p}$.

Nella base degli autostati della posizione, la funzione d'onda è soluzione dell'equazione di Schr\"odinger
\begin{equation}
	\frac{\hbar^2}{2m}\psi''(x)=E\psi(x)
	\label{eq:schrodinger-particella-libera}
\end{equation}
da cui
\begin{equation}
	\psi(x)=Ae^{-\frac{i}{\hbar}\sqrt{2mE}x}+Be^{\frac{i}{\hbar}\sqrt{2mE}x}
	\label{eq:wf-particella-libera}
\end{equation}
per qualche $A,B\in\C$.
L'energia $E$ è necessariamente positiva in quanto, come visto prima, è uguale a $\frac{p^2}{2m}$.
Alternativamente, se $E$ fosse negativa allora $\sqrt{E}$ avrebbe anche una parte immaginaria, che moltiplicata per $\pm i$ negli esponenti porterebbe a un'espressione della forma $e^{\pm\lambda x}$: in tal caso la funzione d'onda divergerebbe per $\abs{x}\to+\infty$ e non sarebbe accettabile (fisicamente) come soluzione.
Oltre a questo, $E$ può assumere qualsiasi valore reale positivo; in ogni caso, per nessun valore risulta $\psi\in L^2(\R)$, dato che la soluzione è oscillante.
Questo fatto non ci deve turbare, perch\'e sappiamo che l'impulso di uno stato non può essere conosciuto con assoluta precisione, mentre all'inizio del problema abbiamo preso proprio un autostato di $\op p$.
Dopotutto, se il sistema fosse nell'autostato $\ket{\pm p}$, avrebbe un'indeterminazione \emph{nulla} sull'impulso e di conseguenza, per il principio di Heisenberg, l'indeterminazione sulla posizione dovrà essere \emph{infinita} (ed è questo il caso) affinch\'e il prodotto $\Delta q\Delta p$ possa essere finito.

\section{Potenziale lineare}
Consideriamo il sistema formato da una particella soggetta al potenziale $V(x)=-ax$.
L'hamiltoniano del sistema è l'operatore
\begin{equation}
	\op H=\frac1{2m}\op p^2-a\op q.
	\label{eq:H-potenziale-lineare}
\end{equation}
Possiamo ricavare qualitativamente alcune informazioni sul sistema guardando al potenziale:
\begin{itemize}
	\item dato che $V\to-\infty$ per $x\to+\infty$, non ammette un minimo, perciò lo spettro di $\op H$ non potrà essere discreto e non possono esistere, di conseguenza, stati legati;
	\item d'altro canto $V\to+\infty$ per $x\to-\infty$ dunque $\op H$ non sarà degenere.
\end{itemize}
Nella base della posizione l'equazione di Schr\"odinger è
\begin{equation}
	-\frac{\hbar^2}{2m}\psi''(x)-(ax+E)\psi(x)=0
	\label{eq:schrodinger-posizione-potenziale-lineare}
\end{equation}
ossia
\begin{equation}
	\psi''(x)+\frac{2m}{\hbar^2}(E+ax)\psi(x)=0.
\end{equation}
Osserviamo che $\frac{2ma}{\hbar^2}$ ha le dimensioni di una $\textup{lunghezza}^{-3}$, e che possiamo raggruppare $a$ nell'equazione nel termine $(E+ax)\psi(x)$.
La variabile $x+\frac{E}{a}$ ha dunque le dimenzioni di una lunghezza: nell'equazione di Schr\"odinger operiamo dunque il cambio di variabile
\begin{equation}
	\xi=\bigg(\frac{2ma}{\hbar^2}\bigg)^{\frac13}\bigg(x-\frac{E}{a}\bigg).
\end{equation}
Per la derivata, abbiamo
\begin{equation}
	\drv{}{x}=\drv{}{\xi}\drv{\xi}{x}=\bigg(\frac{2ma}{\hbar^2}\bigg)^{\frac13}\drv{}{\xi}
\end{equation}
ottenendo la nuova equazione
\begin{equation}
	\begin{gathered}
		\bigg(\frac{2ma}{\hbar^2}\bigg)^{\frac23}\psi''(\xi)+\frac{2ma}{\hbar^2}\bigg(\frac{2ma}{\hbar^2}\bigg)^{-\frac13}\xi\psi(\xi)=0\\
		\bigg(\frac{2ma}{\hbar^2}\bigg)^{\frac23}\psi''(\xi)+\bigg(\frac{2ma}{\hbar^2}\bigg)^{\frac23}\xi\psi(\xi)=0\\
		\psi''(\xi)+\xi\psi(\xi)=0.
	\end{gathered}
	\label{eq:soluzione-potenziale-lineare}
\end{equation}
Cambiamo ancora variabile con $\zeta=-\xi$ (per cui si ha $\ddrv{}{\xi}=\ddrv{}{\zeta}$) per ottenere l'\emph{equazione di Airy}
\begin{equation}
	\psi''(\zeta)+\zeta\psi(\zeta)=0
	\label{eq:airy-potenziale-lineare}
\end{equation}
le cui due soluzioni indipendenti sono le omonime \emph{funzioni di Airy} del primo e del secondo tipo, denominate rispettivamente $\Ai\zeta$ e $\Bi\zeta$.
Sono funzioni particolari, non esprimibili solamente in termini di funzioni elementari.
In particolare, hanno un comportamento oscillatorio per $\zeta<0$ ed esponenziale per $\zeta>0$.
Dato che $\Bi\to+\infty$ esponenzialmente per $\zeta\to+\infty$, ci interessiamo d'ora in poi solo della funzione del primo tipo, $\Ai$: la funzione d'onda soluzione di \eqref{eq:airy-potenziale-lineare} ha dunque la forma $\psi(\zeta)=c\Ai(\zeta)$, o tornando nella variabile $\xi$ precedente $\psi(\xi)=c\Ai(-\xi)$, per qualche $c\in\C$.
La funzione è mostrata nella figura \ref{fig:potenziale-lineare}.
\begin{figure}
	\tikzsetnextfilename{potenziale-lineare}
	\centering
	\begin{tikzpicture}
		\begin{axis}[
				standard,
				enlargelimits,
				height=.5\linewidth, width=\linewidth,
				xlabel=$\xi$,
				xmin=-5, xmax=8.5,
				ymin=-0.5, ymax=0.6,
				xtick={-4,-2,0,2,4,6}, ytick={-0.5,0.5},
				yticklabels={$-\frac12$,$\frac12$}
			]
			\addplot[thick,samples=1000,densely dotted,domain=-5:8] function {airy(-x)};
			\addplot[thick,samples=1000,domain=-5:8] function {(airy(-x))**2};
			\legend{$\psi(\xi)$,$\abs{\psi(\xi)}^2$}
		\end{axis}
	\end{tikzpicture}
	\caption{Soluzione dell'equazione di Schr\"odinger per il potenziale lineare, nella variabile $\xi=\big(\frac{2ma}{\hbar^2}\big)^{1/3}\big(x-\frac{E}{a}\big)$. Il punto $\xi=0$ corrisponde al punto di inversione, in cui $x=\frac{E}{a}$, dove $E$ è l'energia del sistema e $a$ è la costante in $V(x)=-ax$.}
	\label{fig:potenziale-lineare}
\end{figure}


Cerchiamo ora il comportamento asintotico di $\psi$ per valori molto grandi di $\xi$: ipotizzando che $\psi(\xi)\sim\exp(-\gamma{\xi}^s)$, con $\gamma,s>0$, troviamo
\begin{gather*}
	\psi'(\xi)=-(\sgn\xi)\gamma s\abs{\xi}^{s-1}e^{-\gamma\abs{\xi}^s}\\
	\psi''(\xi)=\big[s^2\gamma^2\abs{\xi}^{2s-2}-(\sgn\xi)\gamma s(s-1)\abs{\xi}^{s-2}\big]e^{-\gamma\abs{\xi}^s}.
\end{gather*}
Sostituendole nella \eqref{eq:soluzione-potenziale-lineare} otteniamo che deve essere $\abs{\xi}^{2s-2}\sim\abs{\xi}$ e $s^2\gamma^2=1$ affinch\'e la soluzione sia accettabile, perciò troviamo $s=\frac32$ e $\gamma=\frac23$.
La funzione d'onda approssimata per grandi valori di $\xi$ è dunque
\begin{equation}
	\psi(\xi)\sim ce^{-\frac23\abs{\xi}^{3/2}}
\end{equation}
con $c$ da determinare normalizzando.
Otteniamo una soluzione più ``fine'' ipotizzando che $\psi(\xi)\sim c\exp(-\frac23\abs{\xi}^{3/2})\abs{\xi}^\beta$ per la quale si ottiene $\beta=-\frac14$.

Sebbene sia più ``naturale'' lavorare nella base della posizione, in questo caso risulta più comodo usare la base dell'impulso, perch\'e non appaiono potenze maggiori di $\op q$ nell'hamiltoniano: si ottiene dunque un'equazione differenziale del primo, e non del secondo, ordine.
L'equazione di Schr\"odinger (chiamiamo $\tilde{\psi}$ la funzione d'onda nello spazio degli impulsi per evitare confusioni) in questa base è dunque
\begin{equation}
	\frac1{2m}p^2\tilde{\psi}(p)-ia\hbar\tilde{\psi}'(p)=E\tilde{\psi}(p)
	\label{eq:schrodinger-impulso-potenziale-lineare}
\end{equation}
da cui
\begin{equation}
	\tilde{\psi}'(p)=\frac{i}{a\hbar}\bigg(E-\frac{p^2}{2m}\bigg)\tilde{\psi}(p)
\end{equation}
che ha come soluzione la funzione d'onda
\begin{equation}
	\tilde{\psi}(p)=Ae^{\frac{i}{a\hbar}\big(Ep-\frac{p^3}{6m}\big)}.
\end{equation}

Torniamo dunque allo spazio della posizione con la trasformata di Fourier:
\begin{equation}
	\begin{split}
		\psi(x)&=(\four{\tilde{\psi}})(x)=\frac{A}{\sqrt{2\pi\hbar}}\int_{-\infty}^{+\infty}e^{-\frac{i}{\hbar}px}\tilde{\psi}(p)\,\dd p=\\
		&=\frac{A}{\sqrt{2\pi\hbar}}\int_{-\infty}^{+\infty}\exp\bigg[\frac{i}{\hbar}p\bigg(x+\frac{E}{a}\bigg)+\frac{ip^3}{6am\hbar}\bigg]\,\dd p=\\
		&=\frac{A}{\sqrt{2\pi\hbar}}\int_{-\infty}^{+\infty}\exp\bigg[\frac{i}{\hbar}p\xi\bigg(\frac{2am}{\hbar^2}\bigg)^{-\frac13}+\frac{ip^3}{6am\hbar}\bigg]\,\dd p=\\ &=\tilde{A}\int_{-\infty}^{+\infty}\exp\bigg(ip'\xi+\frac{ip'^3}3\bigg)\,\dd p'
	\end{split}
\end{equation}
ponendo $p'=\frac{p}{\hbar}\big(\frac{2am}{\hbar^2}\big)^{-\frac13}$.
Il risultato è un'espressione integrale proprio della funzione $\Ai(\xi)$, dunque (dopo una normalizzazione per determinare il valore di $\tilde{A}$) si ottiene $\psi(\xi)$.

\section{Buca di potenziale}
Studiamo un sistema composto da una particella in una buca di potenziale di altezza $V_0$, larghezza $2a$ e centrata nell'origine, ossia soggetta al potenziale
\begin{equation}
	V(x)=
	\begin{cases}
		0	&x\in[-a,a]\\
		V_0	&x\in(-\infty,-a)\cup(a,+\infty)
	\end{cases}.
	\label{eq:buca-potenziale-finita}
\end{equation}
La discontinuità di $V$ è finita, dunque le funzioni d'onda dovranno essere di classe $\cont{1}(\R)$.
Inoltre $V$ è pari dunque dovranno essere di parità definita (autostati dell'operatore di inversione spaziale).
Supponiamo che l'energia $E$ del sistema sia minore di $V_0$, e risolviamo l'equazione di Schr\"odinger
\begin{equation}
	\begin{cases}
		\psi''(x)=-\frac{2mE}{\hbar^2}\psi(x)		&x\in(-a,a)\\
		\psi''(x)=\frac{2mE}{\hbar^2}(V_0-E)\psi(x)	&x\in(-\infty,-a)\cup(a,+\infty)
	\end{cases}
	\label{eq:schrodinger-buca-finita}
\end{equation}
da cui la soluzione generale
\begin{equation}
	\begin{cases}
		\psi(x)=A_2e^{-k_2x}+B_2e^{k_2x}	&x\in(-\infty,-a)\\
		\psi(x)=A_1e^{-ik_1x}+B_1e^{ik_1x}	&x\in(-a,a)\\
		\psi(x)=A_3e^{-k_2x}+B_3e^{k_2x}	&x\in(a,+\infty)
	\end{cases}
	\label{eq:soluzione-generale-schrodinger-buca-finita}
\end{equation}
definendo $k_1\defeq\sqrt{\frac{2mE}{\hbar^2}}$ e $k_2\defeq\sqrt{\frac{2m(V_0-E)}{\hbar^2}}$.
Possiamo riscrivere le soluzioni nei termini delle autofunzioni dell'operatore di inversione come
\begin{equation}
	\begin{cases}
		\psi(x)=A_2e^{-k_2x}+B_2e^{k_2x}	&x\in(-\infty,-a)\\
		\psi(x)=A\cos(k_1x)+B\sin(k_1x)		&x\in(-a,a)\\
		\psi(x)=A_3e^{-k_2x}+B_3e^{k_2x}	&x\in(a,+\infty)
	\end{cases}
	\label{eq:schrodinger-buca-potenziale-finita-autofunzioni-inversione}
\end{equation}

A questo punto dobbiamo raccordare le soluzioni in modo che $\psi\in\cont{1}(\R)$, e questo ci darà i valori ammessi di $E$.
Innanzitutto dovrà essere $A_2=B_3=0$ affinch\'e $\psi\in L^2(\R)$.
La funzione d'onda dello stato fondamentale non ha nodi, perciò deve essere pari: nella regione $(-a,a)$ allora risulta $B=0$ da cui $\psi(x)=A\cos(k_1x)$, mentre all'esterno si ha $B_2=A_3$ sempre per la parità.
Uguagliando le espressioni di $\psi$ in $x=-a$ ricaviamo $A\cos(k_1a)=B_2e^{k_2a}$; otteniamo il medesimo risultato in $x=a$ per la parità della funzione.
Ripetiamo il procedimento per la derivata (che sarà dispari) ottenendo $-k_2B_2e^{-k_2a}=-k_1A\sin(k_1a)$.
Dividiamo le due espressioni trovate ottenendo $k_2\tan(k_1a)=k_1$: dato che $a\ne 0$ nella regione interessata, moltiplichiamo per $a$ e definiamo $\gamma=k_1a$ e $\eta=k_2a$.
Otteniamo il sistema
\begin{equation}
	\begin{cases}
		\gamma\tan\gamma=\eta\\
		\gamma^2+\eta^2=\frac{2mV_0a^2}{\hbar^2}
	\end{cases}
	\label{eq:sistema-soluzioni-buca-potenziale-finita}
\end{equation}
Per trovare le soluzioni dobbiamo ricorrere alla via grafica, mostrata in figura \ref{fig:soluzione-grafica-buca-potenziale}.
\begin{figure}
	\tikzsetnextfilename{soluzione-grafica-buca-potenziale}
	\centering
	\begin{tikzpicture}
		\begin{axis}[
				standard,
				enlargelimits,
				xlabel=$\gamma$,
				xmin=0, xmax=5,
				xtick={1.5213,3.1416,4.6629}, xticklabels={$\frac{\pi}2$,$\pi$,$\frac{3\pi}2$},
				ylabel=$\eta$,
				ymin=0, ymax=4,
				ytick=\empty
			]
			\addplot[thick,densely dashed,domain=0:2] function {sqrt(4-x**2)};
			\addplot[thick, domain=0:pi/2-0.01] function {x*tan(x)};
			\addplot[thick, domain=pi:3*pi/2-0.01] function {x*tan(x)};
			\legend{$\gamma^2+\eta^2=\frac{2mV_0}{\hbar^2}$,$\eta=\gamma\tan\gamma$}
		\end{axis}
	\end{tikzpicture}
	\caption{Soluzione grafica per il sistema di equazioni \eqref{eq:sistema-soluzioni-buca-potenziale-finita}.}
	\label{fig:soluzione-grafica-buca-potenziale}
\end{figure}

Il numero di intersezioni dei due grafici, che dà il numero di soluzioni $(\gamma,\eta)$ del sistema, è il numero di possibili valori di $k_1$ e $k_2$ per cui la $\psi$ è una funzione accettabile per il sistema, che tornando alle loro definizioni è poi il numero di valori di $E$.
Notiamo che, qualunque siano i valori di $V_0$ e $a$, il sistema ammette sempre una soluzione: è lo stato fondamentale.
Inoltre, come accade spesso in meccanica quantistica, anche se il sistema parte con un'energia minore del livello del potenziale esiste comunque una probabilità non nulla di individuare la particella ``fuori dalla buca''.

Guardiamo infine al caso di una buca di potenziale infinita: questa volta non è più garantita la continuità di $\psi'$, ma in ogni caso (affinch\'e sia ben definita la probabilità associata alla $\abs{\psi}^2$) dovrà essere $\psi\in\cont{}(\R)$.
Innanzitutto nella figura \ref{fig:soluzione-grafica-buca-potenziale} vediamo subito che il numero di soluzioni diventa infinito.
Inoltre, il coefficiente $k_2$ degli esponenziali tende a $+\infty$, dunque le ``code'' della funzione d'onda, che decadono esponenzialmente a 0, sono del tipo $e^{-k_2\abs{x}}$ cioè diventano sempre più ``basse'': allora $\psi=0$ per $x\notin(-a,a)$.
In una buca di potenziale infinita perciò si ha una barriera impenetrabile di potenziale, cioè stavolta la particella è \emph{davvero} confinata nella regione $(-a,a)$.

\section{Potenziale delta attrattivo}
Dopo una buca di potenziale ``standard'', vediamone la versione infinitamente alta e stretta, modellizzata dall'equazione del potenziale
\begin{equation}
	V(x)=-\lambda\delta(x).
	\label{eq:potenziale-delta-attrattivo}
\end{equation}
Possiamo risolvere questo problema in tre metodi diversi, che ora presentiamo.
\paragraph{Limite della buca di altezza finita}
Riprendendo direttamente l'esempio precedente, consideriamo una buca centrata nell'origine, di altezza $V_0$ e larghezza $a$, e prendiamone il limite per $a\to 0$, $V_0\to+\infty$ tali che $aV_0\to\lambda$, un valore finito.
Prendiamo un potenziale della forma
\begin{equation}
	V(x)=
	\begin{cases}
		-V_0	&x\in\big[-\frac{a}2,\frac{a}2\big]\\
		0		&x\in\big(-\infty,-\frac{a}2\big)\cup\big(\frac{a}2,+\infty\big)
	\end{cases}
\end{equation}
Il potenziale è sempre pari, dunque la funzione d'onda dello stato fondamentale è anch'essa pari.
Scegliamo arbitrariamente il suo valore nell'origine ponendo $\psi(0)=1$, e in modo simile alle equazioni \eqref{eq:soluzione-generale-schrodinger-buca-finita} otteniamo\footnote{La scelta di porre $\psi(0)=1$ non lede la generalità del problema, dato che la funzione d'onda è poi da normalizzare.}
\begin{equation}
	\psi(x)=
	\begin{cases}
		\cos\big(k\frac{a}2\big)e^{\alpha(x+\frac{a}2)}		&x<-\frac{a}2\\
		\cos\big(kx\big)									&\abs{x}\le\frac{a}2\\
		\cos\big(k\frac{a}2\big)e^{-\alpha(x-\frac{a}2)}	&x>\frac{a}2
	\end{cases}
	\label{eq:soluzione-generale-delta-attrattivo}
\end{equation}
dove $k=\sqrt{\frac{2m(E+V_0)}{\hbar^2}}$ e $\alpha=\sqrt{\frac{-2mE}{\hbar^2}}$, con un'energia iniziale del sistema $-V_0<E<0$.
Per come è scritta, la $\psi$ è già continua; imponendo anche $\psi\in\cont{1}(\R)$ otteniamo
\begin{equation}
	k\sin\frac{ka}2=\alpha\cos\frac{ka}2\qqq k\tan\frac{ka}2=\alpha
	\label{eq:condizione-C1-delta-attrattivo}
\end{equation}
e portando al limite $V_0\to+\infty$ si ottiene $ka\to 0$, dunque possiamo approssimare
\begin{equation}
	\alpha=k\tan\frac{ka}2\approx k^2\frac{a}2,
\end{equation}
da cui troviamo l'energia dello stato fondamentale
\begin{equation}
	-\frac{2mE}{\hbar^2}=\frac{m^2\lambda^2}{\hbar^4}\qqq E=-\frac{m\lambda^2}{2\hbar^2}.
	\label{eq:E-fondamentale-delta-attrattivo}
\end{equation}
Con $a\to 0$ si ha inoltre la funzione d'onda $\psi(x)=e^{-\alpha\abs{x}}$.
La sua derivata non è continua in $x=0$, ma non è un problema dato che il potenziale ha una discontinuità infinita in quel punto.
Risulta, in particolare, $\psi'(x)=-\alpha e^{-\alpha\abs{x}}\sgn x$, che integrata in $[-\epsilon,\epsilon]$ dà
\begin{equation}
	\psi'(\epsilon)-\psi'(-\epsilon)=\int_{-\epsilon}^{\epsilon}-\alpha e^{-\alpha\abs{x}}\sgn x\,\dd x=-2\alpha=-\frac{2m\lambda}{\hbar^2}\psi(0)=-\frac{2m\lambda}{\hbar^2}
	\label{eq:discontinuita-derivata-wf-delta-attrattivo}
\end{equation}
che misura la discontinuità di $\psi'$ nell'origine.

\paragraph{Equazione di Schr\"odinger}
Scriviamo direttamente l'equazione di Schr\"odinger del sistema, con energia $E$, che è
\begin{equation}
	-\frac{\hbar^2}{2m}\psi''(x)-\lambda\delta(x)\psi(x)=E\psi(x)\qqq \psi''(x)=-\frac{2m}{\hbar^2}\big[E+\lambda\delta(x)\big]\psi(x).
	\label{eq:schrodinger-delta-attrattivo}
\end{equation}
Integrando i due membri dell'equazione nell'intervallo $[-\epsilon,\epsilon]$ abbiamo
\begin{equation}
	\begin{split}
		\psi'(\epsilon)-\psi'(-\epsilon)&=\int_{-\epsilon}^{\epsilon}-\frac{2m}{\hbar^2}\big[E+\lambda\delta(x)\big]\psi(x)\,\dd x=\\
		&=-\frac{2mE}{\hbar^2}\int_{-\epsilon}^{\epsilon}\psi(x)\,\dd x-\frac{2m\lambda}{\hbar^2}\int_{-\epsilon}^{\epsilon}\delta(x)\psi(x)\,\dd x=\\
		&=-\frac{2mE}{\hbar^2}\int_{-\epsilon}^{\epsilon}\psi(x)\,\dd x-\frac{2m\lambda}{\hbar^2}\psi(0)=\\
		&=-\frac{2m\lambda}{\hbar^2}\psi(0)
	\end{split}
\end{equation}
nel limite, all'ultimo passaggio, per $\epsilon\to 0^+$, e dato che $\psi$ è continua il primo addendo è nullo.
Escludendo l'origine, l'equazione di Schr\"odinger si scrive più semplicemente come $\psi''(x)=-\frac{2mE}{\hbar^2}\psi(x)$, che ha come soluzione $\psi(x)=\psi(0)\exp(-\alpha\abs{x})$ dove $\alpha$ è definita come prima.
Imponendo la discontinuità di $\psi'$ nell'origine, appena trovata integrando la \eqref{eq:schrodinger-delta-attrattivo}, troviamo lo stesso valore per $\alpha$ del paragrafo precedente; si ha un solo valore di energia ammissibile.

\paragraph{Spazio degli impulsi}
Nella notazione di Dirac la \eqref{eq:schrodinger-delta-attrattivo} si scrive, in uno stato legato $\ket{E}$, come
\begin{equation}
	\bigg(\frac1{2m}\op p^2+V(\op q)\bigg)\ket{E}=E\ket{E}\qqq\bigg(\frac1{2m}\op p^2-E\bigg)\ket{E}=-V(\op q)\ket{E}.
	\label{eq:schrodinger-delta-attrattivo-dirac}
\end{equation}
Anzich\'e rappresentarla nella base della posizione, lavoriamo nella base degli impulsi: moltiplicando per $\bra{p}$ otteniamo
\begin{equation}
	\frac1{2m}\bra{p}\op p^2\ket{E}-E\braket{p}{E}=-\bra{p}V(\op q)\ket{E}
\end{equation}
da cui (chiamiamo ancora $\psi$ la funzione d'onda nello spazio degli impulsi)
\begin{equation}
	\frac{p^2}{2m}\psi(p)-E\psi(p)=-\int_{-\infty}^{+\infty}\bra{p}V(\op q)\ket{p'}\braket{p'}{E}\,\dd p'=-\int_{-\infty}^{+\infty}\bra{p}V(\op q)\ket{p'}\psi(p')\,\dd p'.
\end{equation}
Calcoliamo l'elemento di matrice $\bra{p}V(\op q)\ket{p'}$: l'espressione del potenziale nella base della posizione è troppo comoda per non essere usata, dunque introduciamo a destra del potenziale una risoluzione dell'identità con gli autostati $\ket{x}$ della posizione, ottenendo
\begin{multline}
	\bra{p}V(\op q)\ket{p'}=\int_{-\infty}^{+\infty}\bra{p}V(\op q)\ket{x}\braket{x}{p'}\,\dd x=\int_{-\infty}^{+\infty}V(x)\braket{p}{x}\braket{x}{p'}\,\dd x=\\
	=\frac1{2\pi\hbar}\int_{-\infty}^{+\infty}V(x)\exp\bigg[\frac{i}{\hbar}(p'-p)x\bigg]\,\dd x=-\frac{\lambda}{2\pi\hbar}\int_{-\infty}^{+\infty}\delta(x)\exp\bigg[\frac{i}{\hbar}(p'-p)x\bigg]\,\dd x=-\frac{\lambda}{2\pi\hbar}.
\end{multline}
Sostituendo questo nell'equazione precedente, troviamo dunque
\begin{equation}
	\bigg(\frac{p^2}{2m}-E\bigg)\psi(p)=\frac{\lambda}{2\pi\hbar}\int_{-\infty}^{+\infty}\psi(p')\,\dd p'=\frac{\lambda}{\sqrt{2\pi\hbar}}\psi_0
\end{equation}
dove $\psi_0$ è il valore della funzione d'onda \emph{delle coordinate} nel punto $x=0$.
Troviamo infine
\begin{equation}
	\psi(p)=\frac{\lambda\psi_0}{\sqrt{2\pi\hbar}}\frac1{\frac{p^2}{2m}-E}.
	\label{eq:wf-impulsi-delta-attrattivo}
\end{equation}
Integrando in $p$ (su tutto $\R$) e dividendo per $\sqrt{2\pi\hbar}$ i due membri si può trovare la relazione che lega $E$ a $\lambda$, con
\begin{equation}
	\frac1{\sqrt{2\pi\hbar}}\int_{-\infty}^{+\infty}\psi(p)\,\dd p=\psi_0\frac1{2\pi\hbar}\int_{-\infty}^{+\infty}\frac1{\frac{p^2}{2m}-E}\,\dd p
\end{equation}
da cui, considerando $\psi$ normalizzata, e ricordando che $\psi_0=1$ (dai paragrafi precedenti) affinch\'e sia normalizzata anche la funzione d'onda nella base della posizione, si ottiene l'equazione
\begin{equation}
	\int_{-\infty}^{+\infty}\frac1{p^2+2mE}\,\dd p=\frac{\pi\hbar}{m\lambda}.
\end{equation}

