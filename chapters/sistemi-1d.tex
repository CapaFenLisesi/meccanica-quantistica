\chapter{Sistemi unidimensionali}
In queto capitolo affronteremo una serie di esempi ``accademici'' di sistemi in una dimensione, cercando di risolvere dove possibile l'equazione di Schr\"odinger per la funzione d'onda e lo spettro di energia, o di trovare la migliore approssimazione sfruttando le tecniche acquisite.

\section{Metodo costruttivo}
Verifichiamo innanzitutto quante soluzioni accettabili (cioè a quadrato sommabile) possiamo ricavare dall'equazione di Schr\"odinger.
Prendiamo un valore $E\in\R$ appartenente allo spettro discreto dell'hamiltoniano, e prendiamo l'equazione nella rappresentazione nella posizione
\begin{equation}
	\psi''(x)=-\frac{2m}{\hbar^2}\big[E-V(x)\big]\psi(x)
	\label{eq:schroedinger-coordinate}
\end{equation}
Supponiamo, per assurdo, che esistano due soluzioni $\psi_1,\psi_2$: moltiplichiamo l'equazione di $\psi_1$ per $\psi_2$ e viceversa, ottenendo
\begin{equation}
	\begin{gathered}
		\psi_1''(x)\psi_2(x)+\frac{2m}{\hbar^2}\big[E-V(x)\big]\psi_1(x)\psi_2(x)=0;\\
		\psi_2''(x)\psi_1(x)+\frac{2m}{\hbar^2}\big[E-V(x)\big]\psi_2(x)\psi_1(x)=0.
	\end{gathered}
\end{equation}
Sottraendole otteniamo $\psi_1''(x)\psi_2(x)-\psi_2''(x)\psi_1(x)=0$, vale a dire\footnote{Trascuriamo d'ora in poi, per alleggerire la notazione, la dipendenza $(x)$ delle funzioni.}
\begin{equation}
	0=\psi_1''\psi_2-\psi_2''\psi_1+\psi_1'\psi_2'-\psi_2'\psi_1'=\drv{}{x}(\psi_1'\psi_2-\psi_2'\psi_1)
\end{equation}
da cui $\psi_1'\psi_2-\psi_1\psi_2'=k$ ($k\in\R$).
Otteniamo il valore di $k$ valutando il primo membro: per $x\to+\infty$, le due funzioni $\psi_1$ e $\psi_2$ devono essere a quadrato sommabile, perciò devono tendere a zero: allora $\psi_1'\psi_2=\psi_1\psi_2'$.
Dividiamo per $\psi_1\psi_2$ e integriamo ricavando
\begin{equation}
	\int\frac{\psi_1'}{\psi_1}\,\dd x=\int\frac{\psi_2'}{\psi_2}\,\dd x\qqq\log\psi_1=\log\psi_2+k
\end{equation}
cioè $\psi_1=e^k\psi_2$: le due soluzioni sono linearmente dipendenti.
Esiste dunque un \emph{unica} soluzione accettabile dell'equazione di Schr\"odinger per ciascun valore di $E$ dello spettro discreto dell'hamiltoniano.

\subsection{Potenziale limitato inferiormente}
Cerchiamo ora di estrarre delle informazioni qualitative sulla funzione d'onda (nella rappresentazione della posizione) di un sistema unidimensionale composto da una singola particella dalle caratteristiche dell'hamiltoniano.
Supponiamo di avere una funzione del potenziale limitata inferiormente: dato che $\frac1{2m}p^2$ non è mai negativo, si ha sempre $H(x,p)\ge V(x)\ge\min_{x\in\R}V(x)$.
Fissata un'energia $E$ iniziale del sistema, esistono dei punti, detti \emph{punti di inversione}, in cui $V(x)=E$: in tali punti la particella ha velocità nulla, dato che l'energia cinetica è $T=E-V(x)$; dato che $T$ non può essere negativa, in una visione classica del sistema la particella resta confinata tra questi punti di inversione: fissata dunque un'energia e un punto $x\in\R$ iniziale del moto, determiniamo univocamente la regione in cui potremo trovare la particella.
Chiamiamo queste regioni \emph{classicamente accessibili}.
Abbiamo visto però, nell'oscillatore armonico, che la funzione d'onda dello stato fondamentale (una tra tutte) non è mai nulla in nessun punto dell'asse reale.
Se, classicamente, la particella si troverebbe soltanto nell'intervallo limitato dai punti $\pm\frac{\hbar\omega}2$, troviamo comunque una probabilità non nulla di poterla misurare \emph{in qualsiasi posizione}.
Non sappiamo bene cosa aspettarci dal sistema nelle regioni classicamente \emph{non accessibili}: una velocità negativa?
Non possiamo conoscere con precisione la velocità, se vogliamo mantenere le informazioni sulla posizione.

Studiamo il comportamento della funzione d'onda nei due tipi di regione.
Tra i tipi di soluzioni, ossia di autofunzioni dell'hamiltoniano, distinguiamo quelle:
\begin{itemize}
	\item proprie, ossia in $L^2(\R)$;
	\item improprie, che non sono in $L^2(\R)$ ma rimangono limitate per ogni $x\in\R$;
	\item non accettabili, se divergono.
\end{itemize}
Nelle regioni classicamente accessibili, abbiamo $E-V(x)\ge 0$, dunque dalla \eqref{eq:schroedinger-coordinate} vediamo che la derivata seconda $\psi''$ della funzione d'onda è di segno opposto a	$\psi$: si ha dunque un fenomeno di oscillazione.
Nella regione classicamente non accessibile, invece, $\psi''$ e $\psi$ hanno lo stesso segno.

Prendiamo ora un potenziale tale che $\lim_{\abs{x}\to+\infty}V(x)=+\infty$, e prendiamo un punto $x_0$ di inversione tale che la regione per $x>x_0$ sia classicamente non accessibile e per $x<x_0$ (almeno in un intorno) sia classicamente accessibile.
Distinguiamo due casi, se $E$ è maggiore o minore del minimo del potenziale.
Nel primo caso, se la $\psi$ non è identicamente nulla per $x>x_0$, prendiamo un punto $x'$ per cui $\psi(x')\ne 0$.
Possiamo assumere $\psi(x')>0$, perch\'e per la linearità dell'equazione \eqref{eq:schroedinger-coordinate} se $\phi$ è una soluzione lo è anche $k\phi$ per ogni $k\in\C$.
In tale punto, dunque, abbiamo $\psi''(x')>0$, quindi la derivata prima $\psi'$ è crescente: ciò significa che $\psi\to+\infty$ per $x\to+\infty$, ma allora la soluzione non è accettabile.

Se invece $E>V_\textup{min}$, supponiamo ancora che esista un punto $x'>x_0$ tale che $\psi(x')>0$ e $\psi'(x')>0$: risulta che $\psi$ diminuisce per valori crescenti di $x$ fino a incontrare lo zero (che è necessariamente del primo ordine in quanto $\psi''$ ha lo stesso segno di $\psi$, quindi è positiva).
A quel punto, $\psi$ sarà negativa perciò anche $\psi''<0$ e la funzione decrescerà sempre di più, cioè $\psi\to-\infty$, e ancora non è accettabile.
Variando con continuità $\psi'(x')$, possiamo ``regolarla'' in modo che $\psi$ tenda proprio a zero per $x\to+\infty$.
Per valori di $x$ decrescenti, però, la derivata cresce fino a che $x$ incontra il punto di inversione, nel quale $\psi''$ cambia segno.
Maggiore è l'energia $E$, maggiore è la quantità $E-V(x)$ di conseguenza la funzione si ``piega'' di più nella regione classicamente accessibile, dunque al crescere dell'energia le oscillazioni crescono e diventano più fitte.
Proseguendo per $x$ sempre minore, si arriverà all'ultimo punto di inversione a sinistra, dopo il quale resta solo una regione classicamente non acessibile fino a $-\infty$: si ripete qui il ragionamento fatto finora, per cui non sempre la funzione tende a zero per $x\to-\infty$; come prima, se nella regione non classicamente accessibile la funzione ``tocca'' l'asse delle ascisse, cioè si annulla, allora sarà poi negativa e continuerà a decrescere fino a $-\infty$.
D'altro canto, potrebbero esistere valori di $E$ per cui la funzione ``piega troppo'' e torna a crescere (mantenendo sempre una concavità dello stesso segno) divergendo a $+\infty$.
Esistono dunque solo valori particolari di $E$ per i quali la funzione d'onda tende a zero all'infinito, per i quali dunque la soluzione è acettabile come funzione d'onda del sistema.

Prendiamo dunque una soluzione $\psi_0$ accettabile, e confrontiamola con un'altra (che chiamiamo $\psi_1$) relativa ad un'energia maggiore: dato che le due devono essere ortogonali (sono autofunzioni di autovalori differenti), $\psi_1$ dovrà necessariamente avere almeno uno zero in più di $\psi_0$, affinch\'e il prodotto delle due sia un po' negativo e un po' positivo.
In generale, se indichiamo con $\psi_n$ la funzione d'onda dell'$n$-esimo stato eccitato (da $n=1$, ordinati in base alla loro energia), allora $\psi_n$ ha $n-1$ nodi, e se $n_1<n_2$ allora tra due zeri di $\psi_{n_1}$ si trova uno zero di $\psi_{n_2}$.\footnote{Il numero di zeri, e la loro distribuzione nello spazio, di una soluzione di un'equazione differenziale del secondo ordine (come quella di Schr\"odinger) è il soggetto della teoria di Sturm-Liouville.}

\subsection{Potenziale da interazione}
Consideriamo una forma ``standard'' di potenziale dovuto ad un'interazione, tale che $\lim_{x\to-\infty}V(x)=+\infty$ e $\lim_{x\to+\infty}V(x)=0^-$ e sia limitato inferiormente, come in figura \ref{fig:esempio-potenziale-interazione}.
\begin{figure}
	\tikzsetnextfilename{esempio-potenziale-interazione}
	\centering
	\begin{tikzpicture}
		\begin{axis}[
				standard,
				xlabel=$x$, ylabel=$V(x)$,
				xmin=0.8, xmax=2,
				ymin=-2, ymax=4,
				xtick=\empty, ytick=\empty
			]
			\addplot[thick,smooth,domain=0.8:2] function {2*((1/x)**12-2*(1/x)**6)}; % Potenziale di Lennard-Jones di altezza 2 e minimo in x=1
		\end{axis}
	\end{tikzpicture}
	\caption{Esempio di un potenziale dovuto a un'interazione tra due corpi.}
	\label{fig:esempio-potenziale-interazione}
\end{figure}

\begin{itemize}
	\item Se $V_\textup{min}\le E\le0$, si hanno due punti di inversione e la particella oscilla tra di essi; i valori di $E$ ammissibili saranno discreti, per quanto detto precedentemente.
		Per $x\to+\infty$, il potenziale è trascurabile quindi la funzione d'onda si comporterà come un'onda piana, ossia la particella si potrà considerare libera, con un'energia negativa.
		Possono dunque esistere delle autofunzioni, che caratterizzano degli stati legati: è interessante in particolare studiare la funzione nell'intorno del minimo del potenziale.
	\item Se $E>0$, l'equazione di Schr\"odinger ammette soluzioni accettabili per ogni valore reale positivo di $E$, cioè l'hamiltoniano ha uno spettro continuo.
		Come già in precedenza, se avessimo due soluzioni $\psi_1,\psi_2$ dell'equazione, allora giungiamo ancora a $\psi_1'\psi_2-\psi_1\psi_2'=k$ come prima e possiamo valutare il primo membro per $x\to-\infty$, in cui si deve annullare affinch\'e le soluzioni siano accettabili; di conseguenza troviamo ancora che esiste un'unica soluzione della \eqref{eq:schroedinger-coordinate} per ogni valore di $E$, ossia l'hamiltoniano non è degenere.
\end{itemize}

\subsection{Potenziale limitato}
Consideriamo infine un potenziale che rimane limitato anche all'infinito: ammette un minimo, ma converge a due valori (non necessariamente uguali) $V_1$ e $V_2$ per $x$ che tende a $-\infty$ o $+\infty$, come nella figura \ref{fig:esempio-potenziale-limitato}
\begin{figure}
	\tikzsetnextfilename{esempio-potenziale-limitato}
	\centering
	\begin{tikzpicture}
		\begin{axis}[
				standard,
				xlabel=$x$, ylabel=$V(x)$,
				xmin=-8, xmax=8,
				ymin=-1, ymax=2,
				xtick=\empty,
				ytick={1},
				yticklabels={$V_1$},
				extra y ticks={2}, % la keyword extra permette di specificare uno stile differente per alcuni ticks
				extra y tick labels={$V_2$},
				extra y tick style={yticklabel style={anchor=west, xshift=1ex}}
			]
			\draw[thick] (axis cs:-8,1.9) to[out=350,in=180] (axis cs:-1,-1) to[out=0,in=190] (axis cs:8,0.9);
			\draw[dashed] (axis cs:0,1) -- (axis cs:8,1);
			\draw[dashed] (axis cs:-8,2) -- (axis cs:0,2);
		\end{axis}
	\end{tikzpicture}
	\caption{Esempio di un potenziale limitato.}
	\label{fig:esempio-potenziale-limitato}
\end{figure}

Supponiamo per esempio che $V_1<V_2$: allora abbiamo quattro casi in base all'energia $E$ del sistema:
\begin{itemize}
	\item se $E\in(-\infty,V_\textup{min})$ non esistono soluzioni accettabili;
	\item se $E\in[V_\textup{min},V_1)$ esistono degli stati legati e lo spettro dell'hamiltoniano è discreto;
	\item se $E\in[V_1,V_2]$ si ha uno spettro continuo, ma l'hamiltoniano ancora non è degenere;
	\item se $E\in(V_2,+\infty)$ lo spettro è continuo e la funzione d'onda è oscillante, dunque è accettabile ma non è un'autofunzione propria, inoltre l'hamiltoniano è degenere perch\'e non possiamo più calcolare il solito $\psi_1'\psi_2-\psi_1\psi_2'=k$ in un punto in cui siamo certi che le funzioni d'onda si annullino.
\end{itemize}

\subsection{Regolarità della funzione d'onda}
Per concludere le nostre osservazioni sulle soluzioni dell'equazioni di Schr\"odinger, è fondamentale studiare la regolarità delle funzioni d'onda: spesso il potenziale è definito ``a pezzi'' o è discontinuo, e si avrà una funzione d'onda definita anch'essa in pezzi che devono essere opportunamente raccordati.
Questo permette di stabilire il valore di alcune costanti arbitrarie.
Oltre a trovarsi nello spazio $L^2(\R)$ o in analoghi multidimensionali, la funzione d'onda deve essere sempre limitata: non avrebbe senso fisico una densità di probabilità infinita in un punto.

Solitamente l'equazione da risolvere è del tipo $\psi''(x)=\frac{2m}{\hbar^2}[V(x)-E]\psi(x)$: supponiamo di trovarci nel caso di un potenziale che varia bruscamente (non necessariamente con continuità) in un breve intervallo $[0,\epsilon]$, rimanendo però limitato.
Integrando l'equazione in tale intervallo si ha
\begin{equation}
	\psi'(\epsilon)-\psi'(0)=\lim_{\epsilon\to 0}\int_0^\epsilon\frac{2m}{\hbar^2}[V(x)-E]\psi(x)\,\dd x.
\end{equation}
Se dunque $V$ è limitato in $[0,\epsilon]$, la funzione integrale (come funzione di $\epsilon$) è continua perciò il limite tende a zero, ossia $\lim_{\epsilon\to 0}\psi'(\epsilon)=\psi'(0)$: quindi la derivata della funzione d'onda è continua, e lo è di conseguenza anche la funzione d'onda stessa.\footnote{Un altro motivo per non poter accettare un'eventuale discontinuità della funzione d'onda è il seguente: se vedessimo una discontinuità finita, ossia un ``salto'', come il limite di un tratto di funzione sempre più ripido fino a diventare verticale, allora l'impulso del sistema, che è proporzionale alla derivata della funzione, risulterebbe infinito, cosa che non avrebbe senso.}
Notare che il potenziale può anche non essere continuo: l'ipotesi fondamentale è che sia limitato.
Queste conclusioni infatti non valgono nel caso di discontinuità infinite, come vedremo più avanti.

\section{Operatore di inversione spaziale}
Introduciamo un importante operatore che ci servirà spesso negli esempi che incontreremo, l'operatore di \emph{inversione spaziale}, detto anche operatore di \emph{parità}.
Chiamiamo $\op I$ l'operatore (in una dimensione) che, nella rappresentazione di Schr\"odinger della posizione, agisce sulla funzione d'onda $\psi(q)=\braket{q}{\psi}$ dello stato $\ket{\psi}$ come
\begin{equation}
	\bra{q}\op I\ket{\psi}=\psi(-q).
	\label{eq:inversione-spaziale-posizione}
\end{equation}
La sua rappresentazione nella base dell'impulso la otteniamo con il cambiamento di base usando la trasformata di Fourier: detta $\tilde{\psi}(p)=\braket{p}{\psi}$, risulta
\begin{equation}
	\begin{split}
		\bra{p}\op I\ket{\psi}&=\four(\op I\psi)(p)=\frac1{\sqrt{2\pi\hbar}}\int_{-\infty}^{+\infty}e^{-\frac{i}{\hbar}pq}(\op I\psi)(q)\,\dd q=\\
		&=\frac1{\sqrt{2\pi\hbar}}\int_{-\infty}^{+\infty}e^{-\frac{i}{\hbar}pq}\psi(-q)\,\dd q=\\
		&=\frac1{\sqrt{2\pi\hbar}}\int_{-\infty}^{+\infty}e^{-\frac{i}{\hbar}(-p)(-q)}\psi(-q)\,\dd q=\\
		&=\frac1{\sqrt{2\pi\hbar}}\int_{-\infty}^{+\infty}e^{-\frac{i}{\hbar}(-p)q'}\psi(q')\,\dd q'=\\
		&=\tilde{\psi}(-p)
	\end{split}
	\label{eq:inversione-spaziale-impulso}
\end{equation}
con il cambiamento di variabile $q'=-q$.
Dunque l'operatore $\op I$ cambia segno sia alla posizione che all'impulso.
Tornando nella notazione di Dirac, senza passare nella rappresentazione di Schr\"odinger, abbiamo dunque $\op I\ket{q}=\ket{-q}$ e $\op I\ket{p}=\ket{-p}$.

Ovviamente l'operatore è anche il suo stesso inverso, ossia $\op I^2=1$, inoltre
\begin{multline}
	\bra{A}\adj{\op I}\ket{B}=\bra{B}\op I\ket{A}^*=\bigg[\int_{-\infty}^{+\infty}\psi_B^*(q)\psi_A(-q)\,\dd q\bigg]^*=\int_{-\infty}^{+\infty}\psi_B(q)\psi_A^*(-q)\,\dd q=\\
	=\int_{-\infty}^{+\infty}\psi_B(-q')\psi_A^*(q')\,\dd q'=\bra{A}\op I\ket{B}
	\label{eq:operatore-inversione-spaziale-hermitiano}
\end{multline}
dunque $\op I$ è hermitiano, ma allora $1=\op I^2=\adj{\op I}\op I=\op I\adj{\op I}$ per cui è anche unitario.
Dato che è sia hermitiano che unitario, i suoi autovalori sono di modulo unitario e reali, quindi lo spettro è $\{-1,1\}$.
Le sue autofunzioni sono tali che $f(-x)=f(x)$ oppure $f(-x)=-f(x)$, ossia sono le funzioni pari o dispari.\footnote{Dato che lo spettro è discreto, abbiamo autovalori \emph{propri} a cui corrispondono autofunzioni \emph{proprie} quindi, ad esempio, la delta di Dirac è esclusa.}
Ogni funzione $g$ dopotutto si può scrivere come
\begin{equation}
	g(x)=\frac{g(x)+g(-x)}2+\frac{g(x)-g(-x)}2
	\label{eq:decomposizione-funzione-pari-dispari}
\end{equation}
e il primo addendo è pari mentre il secondo è dispari.

Completiamo il quadro con le relazioni tra $\op I$ e gli operatori di posizione e impulso: anzich\'e calcolare i commutatori direttamente su una funzione di $L^2(\R)$, calcoliamo
\begin{equation}
	\bra{q}\op I\op q\op I\ket{\psi}=\bra{-q}\op q\op I\ket{\psi}=-q\bra{-q}\op I\ket{\psi}=-q\braket{q}{\psi}
\end{equation}
da cui $\op I\op q\op I=-\op q$.
Moltiplicando a destra per $\op I$ troviamo $-\op q\op I=\op I\op q\op I^2=\op I\op q$, dunque i due operatori non commutano.
Ripetiamo i calcoli per $\op q^2$: con lo stesso metodo di prima abbiamo
\begin{equation}
	\bra{q}\op I\op q^2\op I\ket{\psi}=\bra{-q}\op q^2\op I\ket{\psi}=(-q)^2\bra{-q}\op I\ket{\psi}=q^2\braket{q}{\psi}
\end{equation}
perciò $\op I\op q^2\op I=\op q^2$.
Ancora moltiplicando a destra per $\op I$ troviamo $\op q^2\op I=\op I\op q^2\op I^2=\op I\op q^2$ quindi i due commutano.

Per l'impulso troviamo un risultato analogo, dato che l'azione di $\op I$ sugli autostati di $\op p$ è la stessa che su quelli di $\op q$.
Abbiamo dunque
\begin{equation}
	\begin{gathered}
		[\op q,\op I]=-2\op I\op q,\hspace{1cm}[\op q^2,\op I]=0;\\
		[\op p,\op I]=-2\op I\op p,\hspace{1cm}[\op p^2,\op I]=0.
	\end{gathered}
	\label{eq:commutatori-inversione-posizione-impulso}
\end{equation}
\paragraph{Parità del potenziale}
È evidente che, oltre a $\op q^2$ (o $\op p^2$), l'operatore di inversione commuta anche con una qualsiasi funzione pari di $\op q$ (o di $\op p$).\footnote{Si intende ovviamente una funzione \emph{soltanto} di $\op q$ o soltanto di $\op p$, altrimenti il risultato in generale non è valido.}
In questo caso, per hamiltoniani della forma $T(\op p)+V(\op q)$ dove $T$ e $V$ sono funzioni pari rispettivamente di $\op p$ e di $\op q$, si ha che $[\op H,\op I]=0$: se $\op H$ non è degenere, i due operatori hanno allora gli stessi autostati; dato inoltre che $\op I$ ha autovalori $\pm 1$ si ha che le autofunzioni di $\op H$ devono essere a parità definita (pari o dispari).
In particolare, dato che lo stato fondamentale non ha zeri, dorvà per forza rientrare tra le funzioni pari.
Analogamente, lo stato successivo (diciamo per $n=1$) ha un solo zero che è semplice, quindi l'autofunzione è dispari, e cos\`i via.
Inoltre, le funzioni pari e dispari sono autofunzioni di $\op I$, che è hermitiano, corrispondenti ad autostati differenti; allora una funzione dispari e una pari sono ortogonali in $L^2(\R)$.

Un esempio di questo fatto lo troviamo nell'oscillatore armonico unidimensionale: dato inoltre che l'hamiltoniano è somma di un multiplo di $\op p^2$ e un multiplo di $\op q^2$, abbiamo $[\op H,\op I]=0$.
L'hamiltoniano inoltre non è degenere, perciò ciascun autostato è direttamente anche un autostato dell'operatore di inversione, senza possibilità di avere delle combinazioni lineari.
Di conseguenza ogni autostato di $\op H$ ha una parità definita: dopotutto sappiamo che le autofunzioni sono le funzioni di Hermite, ed esse stesse hanno una parità definita, ossia $h_n(-x)=(-1)^nh_n(x)$; in altre parole, $\op I\ket{n}=(-1)^n\ket{n}$.

\section{Particella libera}
Contrariamente al caso classico, in cui questo era il sistema più semplice da studiare, nel caso quantistico troviamo qualche complicazione.
L'hamiltoniano del sistema è semplicemente
\begin{equation}
	\op H=\frac1{2m}\op p^2.
	\label{eq:H-particella-libera}
\end{equation}
Notiamo immediatamente che $[\op p,\op H]=0$ essendo $\op H$ funzione unicamente di $\op p$, dunque le due osservabili sono compatibili.
L'impulso dunque è una quantità conservata: a questo corrisponde la simmetria (evidente) del sistema per traslazioni spaziali, che sono generate da $\op p$.
Preso un autostato $\ket{p}$ dell'impulso (di autovalore $p$), troviamo $\op H\ket{p}=\frac1{2m}\op p^2\ket{p}=\frac1{2m}p^2\ket{p}$.
D'altro canto, $\frac1{2m}p^2$ è l'autovalore dell'hamiltoniano dunque è l'energia $E$ del sistema; se fissiamo dunque questo valore, troviamo due autostati di $\op p$ aventi questa energia, che sono $\ket{p}$ e $\ket{-p}$.
Ciò significa che l'hamiltoniano è degenere, ossia un suo autostato $\ket{E}$ si scrive come $\alpha\ket{p}+\beta\ket{-p}$.

Nella base degli autostati della posizione, la funzione d'onda è soluzione dell'equazione di Schr\"odinger
\begin{equation}
	\frac{\hbar^2}{2m}\psi''(x)=E\psi(x)
	\label{eq:schrodinger-particella-libera}
\end{equation}
da cui
\begin{equation}
	\psi(x)=Ae^{-\frac{i}{\hbar}\sqrt{2mE}x}+Be^{\frac{i}{\hbar}\sqrt{2mE}x}
	\label{eq:wf-particella-libera}
\end{equation}
per qualche $A,B\in\C$.
L'energia $E$ è necessariamente positiva in quanto, come visto prima, è uguale a $\frac{p^2}{2m}$.
Alternativamente, se $E$ fosse negativa allora $\sqrt{E}$ avrebbe anche una parte immaginaria, che moltiplicata per $\pm i$ negli esponenti porterebbe a un'espressione della forma $e^{\pm\lambda x}$: in tal caso la funzione d'onda divergerebbe per $\abs{x}\to+\infty$ e non sarebbe accettabile (fisicamente) come soluzione.
Oltre a questo, $E$ può assumere qualsiasi valore reale positivo; in ogni caso, per nessun valore risulta $\psi\in L^2(\R)$, dato che la soluzione è oscillante.
Questo fatto non ci deve turbare, perch\'e sappiamo che l'impulso di uno stato non può essere conosciuto con assoluta precisione, mentre all'inizio del problema abbiamo preso proprio un autostato di $\op p$.
Dopotutto, se il sistema fosse nell'autostato $\ket{\pm p}$, avrebbe un'indeterminazione \emph{nulla} sull'impulso e di conseguenza, per il principio di Heisenberg, l'indeterminazione sulla posizione dovrà essere \emph{infinita} (ed è questo il caso) affinch\'e il prodotto $\Delta q\Delta p$ possa essere finito.

\section{Potenziale lineare}
Consideriamo il sistema formato da una particella soggetta al potenziale $V(x)=-ax$.
L'hamiltoniano del sistema è l'operatore
\begin{equation}
	\op H=\frac1{2m}\op p^2-a\op q.
	\label{eq:H-potenziale-lineare}
\end{equation}
Possiamo ricavare qualitativamente alcune informazioni sul sistema guardando al potenziale:
\begin{itemize}
	\item dato che $V\to-\infty$ per $x\to+\infty$, non ammette un minimo, perciò lo spettro di $\op H$ non potrà essere discreto e non possono esistere, di conseguenza, stati legati;
	\item d'altro canto $V\to+\infty$ per $x\to-\infty$ dunque $\op H$ non sarà degenere.
\end{itemize}
Nella base della posizione l'equazione di Schr\"odinger è
\begin{equation}
	-\frac{\hbar^2}{2m}\psi''(x)-(ax+E)\psi(x)=0
	\label{eq:schrodinger-posizione-potenziale-lineare}
\end{equation}
ossia
\begin{equation}
	\psi''(x)+\frac{2m}{\hbar^2}(E+ax)\psi(x)=0.
\end{equation}
Osserviamo che $\frac{2ma}{\hbar^2}$ ha le dimensioni di una $\textup{lunghezza}^{-3}$, e che possiamo raggruppare $a$ nell'equazione nel termine $(E+ax)\psi(x)$.
La variabile $x+\frac{E}{a}$ ha dunque le dimenzioni di una lunghezza: nell'equazione di Schr\"odinger operiamo dunque il cambio di variabile
\begin{equation}
	\xi=\bigg(\frac{2ma}{\hbar^2}\bigg)^{\frac13}\bigg(x-\frac{E}{a}\bigg).
\end{equation}
Per la derivata, abbiamo
\begin{equation}
	\drv{}{x}=\drv{}{\xi}\drv{\xi}{x}=\bigg(\frac{2ma}{\hbar^2}\bigg)^{\frac13}\drv{}{\xi}
\end{equation}
ottenendo la nuova equazione
\begin{equation}
	\begin{gathered}
		\bigg(\frac{2ma}{\hbar^2}\bigg)^{\frac23}\psi''(\xi)+\frac{2ma}{\hbar^2}\bigg(\frac{2ma}{\hbar^2}\bigg)^{-\frac13}\xi\psi(\xi)=0\\
		\bigg(\frac{2ma}{\hbar^2}\bigg)^{\frac23}\psi''(\xi)+\bigg(\frac{2ma}{\hbar^2}\bigg)^{\frac23}\xi\psi(\xi)=0\\
		\psi''(\xi)+\xi\psi(\xi)=0.
	\end{gathered}
	\label{eq:soluzione-potenziale-lineare}
\end{equation}
Cambiamo ancora variabile con $\zeta=-\xi$ (per cui si ha $\ddrv{}{\xi}=\ddrv{}{\zeta}$) per ottenere l'\emph{equazione di Airy}
\begin{equation}
	\psi''(\zeta)+\zeta\psi(\zeta)=0
	\label{eq:airy-potenziale-lineare}
\end{equation}
le cui due soluzioni indipendenti sono le omonime \emph{funzioni di Airy} del primo e del secondo tipo, denominate rispettivamente $\Ai\zeta$ e $\Bi\zeta$.
Sono funzioni particolari, non esprimibili solamente in termini di funzioni elementari.
In particolare, hanno un comportamento oscillatorio per $\zeta<0$ ed esponenziale per $\zeta>0$.
Dato che $\Bi\to+\infty$ esponenzialmente per $\zeta\to+\infty$, ci interessiamo d'ora in poi solo della funzione del primo tipo, $\Ai$: la funzione d'onda soluzione di \eqref{eq:airy-potenziale-lineare} ha dunque la forma $\psi(\zeta)=c\Ai(\zeta)$, o tornando nella variabile $\xi$ precedente $\psi(\xi)=c\Ai(-\xi)$, per qualche $c\in\C$.
La funzione è mostrata nella figura \ref{fig:potenziale-lineare}.
\begin{figure}
	\tikzsetnextfilename{potenziale-lineare}
	\centering
	\begin{tikzpicture}
		\begin{axis}[
				standard,
				enlargelimits,
				height=.5\linewidth, width=\linewidth,
				xlabel=$\xi$,
				xmin=-5, xmax=8.5,
				ymin=-0.5, ymax=0.6,
				xtick={-4,-2,0,2,4,6}, ytick={-0.5,0.5},
				yticklabels={$-\frac12$,$\frac12$}
			]
			\addplot[thick,samples=1000,densely dotted,domain=-5:8] function {airy(-x)};
			\addplot[thick,samples=1000,domain=-5:8] function {(airy(-x))**2};
			\legend{$\psi(\xi)$,$\abs{\psi(\xi)}^2$}
		\end{axis}
	\end{tikzpicture}
	\caption{Soluzione dell'equazione di Schr\"odinger per il potenziale lineare, nella variabile $\xi=\big(\frac{2ma}{\hbar^2}\big)^{1/3}\big(x-\frac{E}{a}\big)$. Il punto $\xi=0$ corrisponde al punto di inversione, in cui $x=\frac{E}{a}$, dove $E$ è l'energia del sistema e $a$ è la costante in $V(x)=-ax$.}
	\label{fig:potenziale-lineare}
\end{figure}


Cerchiamo ora il comportamento asintotico di $\psi$ per valori molto grandi di $\xi$: ipotizzando che $\psi(\xi)\sim\exp(-\gamma{\xi}^s)$, con $\gamma,s>0$, troviamo
\begin{gather*}
	\psi'(\xi)=-(\sgn\xi)\gamma s\abs{\xi}^{s-1}e^{-\gamma\abs{\xi}^s}\\
	\psi''(\xi)=\big[s^2\gamma^2\abs{\xi}^{2s-2}-(\sgn\xi)\gamma s(s-1)\abs{\xi}^{s-2}\big]e^{-\gamma\abs{\xi}^s}.
\end{gather*}
Sostituendole nella \eqref{eq:soluzione-potenziale-lineare} otteniamo che deve essere $\abs{\xi}^{2s-2}\sim\abs{\xi}$ e $s^2\gamma^2=1$ affinch\'e la soluzione sia accettabile, perciò troviamo $s=\frac32$ e $\gamma=\frac23$.
La funzione d'onda approssimata per grandi valori di $\xi$ è dunque
\begin{equation}
	\psi(\xi)\sim ce^{-\frac23\abs{\xi}^{3/2}}
\end{equation}
con $c$ da determinare normalizzando.
Otteniamo una soluzione più ``fine'' ipotizzando che $\psi(\xi)\sim c\exp(-\frac23\abs{\xi}^{3/2})\abs{\xi}^\beta$ per la quale si ottiene $\beta=-\frac14$.

Sebbene sia più ``naturale'' lavorare nella base della posizione, in questo caso risulta più comodo usare la base dell'impulso, perch\'e non appaiono potenze maggiori di $\op q$ nell'hamiltoniano: si ottiene dunque un'equazione differenziale del primo, e non del secondo, ordine.
L'equazione di Schr\"odinger (chiamiamo $\tilde{\psi}$ la funzione d'onda nello spazio degli impulsi per evitare confusioni) in questa base è dunque
\begin{equation}
	\frac1{2m}p^2\tilde{\psi}(p)-ia\hbar\tilde{\psi}'(p)=E\tilde{\psi}(p)
	\label{eq:schrodinger-impulso-potenziale-lineare}
\end{equation}
da cui
\begin{equation}
	\tilde{\psi}'(p)=\frac{i}{a\hbar}\bigg(E-\frac{p^2}{2m}\bigg)\tilde{\psi}(p)
\end{equation}
che ha come soluzione la funzione d'onda
\begin{equation}
	\tilde{\psi}(p)=Ae^{\frac{i}{a\hbar}\big(Ep-\frac{p^3}{6m}\big)}.
\end{equation}

Torniamo dunque allo spazio della posizione con la trasformata di Fourier:
\begin{equation}
	\begin{split}
		\psi(x)&=(\four{\tilde{\psi}})(x)=\frac{A}{\sqrt{2\pi\hbar}}\int_{-\infty}^{+\infty}e^{-\frac{i}{\hbar}px}\tilde{\psi}(p)\,\dd p=\\
		&=\frac{A}{\sqrt{2\pi\hbar}}\int_{-\infty}^{+\infty}\exp\bigg[\frac{i}{\hbar}p\bigg(x+\frac{E}{a}\bigg)+\frac{ip^3}{6am\hbar}\bigg]\,\dd p=\\
		&=\frac{A}{\sqrt{2\pi\hbar}}\int_{-\infty}^{+\infty}\exp\bigg[\frac{i}{\hbar}p\xi\bigg(\frac{2am}{\hbar^2}\bigg)^{-\frac13}+\frac{ip^3}{6am\hbar}\bigg]\,\dd p=\\ &=\tilde{A}\int_{-\infty}^{+\infty}\exp\bigg(ip'\xi+\frac{ip'^3}3\bigg)\,\dd p'
	\end{split}
\end{equation}
ponendo $p'=\frac{p}{\hbar}\big(\frac{2am}{\hbar^2}\big)^{-\frac13}$.
Il risultato è un'espressione integrale proprio della funzione $\Ai(\xi)$, dunque (dopo una normalizzazione per determinare il valore di $\tilde{A}$) si ottiene $\psi(\xi)$.

\section{Buca di potenziale}
Studiamo un sistema composto da una particella in una buca di potenziale di altezza $V_0$, larghezza $2a$ e centrata nell'origine, ossia soggetta al potenziale
\begin{equation}
	V(x)=
	\begin{cases}
		0	&x\in[-a,a]\\
		V_0	&x\in(-\infty,-a)\cup(a,+\infty)
	\end{cases}.
	\label{eq:buca-potenziale-finita}
\end{equation}
La discontinuità di $V$ è finita, dunque le funzioni d'onda dovranno essere di classe $\cont{1}(\R)$.
Inoltre $V$ è pari dunque dovranno essere di parità definita (autostati dell'operatore di inversione spaziale).
Supponiamo che l'energia $E$ del sistema sia minore di $V_0$, e risolviamo l'equazione di Schr\"odinger
\begin{equation}
	\begin{cases}
		\psi''(x)=-\frac{2mE}{\hbar^2}\psi(x)		&x\in(-a,a)\\
		\psi''(x)=\frac{2mE}{\hbar^2}(V_0-E)\psi(x)	&x\in(-\infty,-a)\cup(a,+\infty)
	\end{cases}
	\label{eq:schrodinger-buca-finita}
\end{equation}
da cui la soluzione generale
\begin{equation}
	\begin{cases}
		\psi(x)=A_2e^{-k_2x}+B_2e^{k_2x}	&x\in(-\infty,-a)\\
		\psi(x)=A_1e^{-ik_1x}+B_1e^{ik_1x}	&x\in(-a,a)\\
		\psi(x)=A_3e^{-k_2x}+B_3e^{k_2x}	&x\in(a,+\infty)
	\end{cases}
	\label{eq:soluzione-generale-schrodinger-buca-finita}
\end{equation}
definendo $k_1\defeq\sqrt{\frac{2mE}{\hbar^2}}$ e $k_2\defeq\sqrt{\frac{2m(V_0-E)}{\hbar^2}}$.
Possiamo riscrivere le soluzioni nei termini delle autofunzioni dell'operatore di inversione come
\begin{equation}
	\begin{cases}
		\psi(x)=A_2e^{-k_2x}+B_2e^{k_2x}	&x\in(-\infty,-a)\\
		\psi(x)=A\cos(k_1x)+B\sin(k_1x)		&x\in(-a,a)\\
		\psi(x)=A_3e^{-k_2x}+B_3e^{k_2x}	&x\in(a,+\infty)
	\end{cases}
	\label{eq:schrodinger-buca-potenziale-finita-autofunzioni-inversione}
\end{equation}

A questo punto dobbiamo raccordare le soluzioni in modo che $\psi\in\cont{1}(\R)$, e questo ci darà i valori ammessi di $E$.
Innanzitutto dovrà essere $A_2=B_3=0$ affinch\'e $\psi\in L^2(\R)$.
La funzione d'onda dello stato fondamentale non ha nodi, perciò deve essere pari: nella regione $(-a,a)$ allora risulta $B=0$ da cui $\psi(x)=A\cos(k_1x)$, mentre all'esterno si ha $B_2=A_3$ sempre per la parità.
Uguagliando le espressioni di $\psi$ in $x=-a$ ricaviamo $A\cos(k_1a)=B_2e^{k_2a}$; otteniamo il medesimo risultato in $x=a$ per la parità della funzione.
Ripetiamo il procedimento per la derivata (che sarà dispari) ottenendo $-k_2B_2e^{-k_2a}=-k_1A\sin(k_1a)$.
Dividiamo le due espressioni trovate ottenendo $k_2\tan(k_1a)=k_1$: dato che $a\ne 0$ nella regione interessata, moltiplichiamo per $a$ e definiamo $\gamma=k_1a$ e $\eta=k_2a$.
Otteniamo il sistema
\begin{equation}
	\begin{cases}
		\gamma\tan\gamma=\eta\\
		\gamma^2+\eta^2=\frac{2mV_0a^2}{\hbar^2}
	\end{cases}
	\label{eq:sistema-soluzioni-buca-potenziale-finita}
\end{equation}
Per trovare le soluzioni dobbiamo ricorrere alla via grafica, mostrata in figura \ref{fig:soluzione-grafica-buca-potenziale}.
\begin{figure}
	\tikzsetnextfilename{soluzione-grafica-buca-potenziale}
	\centering
	\begin{tikzpicture}
		\begin{axis}[
				standard,
				enlargelimits,
				xlabel=$\gamma$,
				xmin=0, xmax=5,
				xtick={1.5213,3.1416,4.6629}, xticklabels={$\frac{\pi}2$,$\pi$,$\frac{3\pi}2$},
				ylabel=$\eta$,
				ymin=0, ymax=4,
				ytick=\empty
			]
			\addplot[thick,densely dashed,domain=0:2] function {sqrt(4-x**2)};
			\addplot[thick, domain=0:pi/2-0.01] function {x*tan(x)};
			\addplot[thick, domain=pi:3*pi/2-0.01] function {x*tan(x)};
			\legend{$\gamma^2+\eta^2=\frac{2mV_0}{\hbar^2}$,$\eta=\gamma\tan\gamma$}
		\end{axis}
	\end{tikzpicture}
	\caption{Soluzione grafica per il sistema di equazioni \eqref{eq:sistema-soluzioni-buca-potenziale-finita}.}
	\label{fig:soluzione-grafica-buca-potenziale}
\end{figure}

Il numero di intersezioni dei due grafici, che dà il numero di soluzioni $(\gamma,\eta)$ del sistema, è il numero di possibili valori di $k_1$ e $k_2$ per cui la $\psi$ è una funzione accettabile per il sistema, che tornando alle loro definizioni è poi il numero di valori di $E$.
Notiamo che, qualunque siano i valori di $V_0$ e $a$, il sistema ammette sempre una soluzione: è lo stato fondamentale.
Inoltre, come accade spesso in meccanica quantistica, anche se il sistema parte con un'energia minore del livello del potenziale esiste comunque una probabilità non nulla di individuare la particella ``fuori dalla buca''.

Guardiamo infine al caso di una buca di potenziale infinita: questa volta non è più garantita la continuità di $\psi'$, ma in ogni caso (affinch\'e sia ben definita la probabilità associata alla $\abs{\psi}^2$) dovrà essere $\psi\in\cont{}(\R)$.
Innanzitutto nella figura \ref{fig:soluzione-grafica-buca-potenziale} vediamo subito che il numero di soluzioni diventa infinito.
Inoltre, il coefficiente $k_2$ degli esponenziali tende a $+\infty$, dunque le ``code'' della funzione d'onda, che decadono esponenzialmente a 0, sono del tipo $e^{-k_2\abs{x}}$ cioè diventano sempre più ``basse'': allora $\psi=0$ per $x\notin(-a,a)$.
In una buca di potenziale infinita perciò si ha una barriera impenetrabile di potenziale, cioè stavolta la particella è \emph{davvero} confinata nella regione $(-a,a)$.

\paragraph{Buca di potenziale traslata}
Affrontiamo ora un sistema analogo al precedente, ma non più centrato nell'origine, prendendo il potenziale
\begin{equation}
	V(x)=
	\begin{cases}
		0	&x\in[0,a]\\
		V_0	&x\in(-\infty,0)\cup(a,+\infty)
	\end{cases}
	\label{eq:buca-potenziale-traslata}
\end{equation}
Questa volta non è pari, e $\op H$ non commuta con l'operatore di inversione spaziale.
In ogni caso, per $E<V_0$ la soluzione è una funzione d'onda analoga al caso precedente, del tipo
\begin{equation}
	\psi(x)=A\cos(kx)+B\sin(kx),\qquad k\defeq\frac{\sqrt{2m(V_0-E)}}{\hbar}
\end{equation}
per $x\in(0,a)$.
Nel limite $V_0\to+\infty$ abbiamo ancora $\psi=0$ nei punti di discontinuità del potenziale, quindi $\psi(0)=\psi(a)=0$: questo porta ad $A=0$ e $B\sin(ka)=0$, ossia $ka=n\pi$ con $n\in\N$.
Elevando al quadrato quest'ultima relazione otteniamo gli autovalori di energia, ossia
\begin{equation}
	\frac{2mE}{\hbar^2}a^2=n^2\pi^2\qqq E=\frac{n^2\pi^2\hbar^2}{2ma^2}.
\end{equation}

\paragraph{Particella in una scatola}
Come digressione sulla buca di potenziale, consideriamo una ``particella in una scatola'', ossia un sistema fisico composto da una particella singola confinata in una regione di spazio rettangolare; possiamo vederlo alternativamente come una buca di potenziale di altezza infinita, nelle tre dimensioni spaziali.
Il concetto fondamentale è che la particella ha una probabilità di essere trovata fuori dalla regione assegnata pari a zero.
Nel suo moto all'interno, qualunque esso sia, essa esercita una certa forza sulle barriere: possiamo chiederci quanto vale la pressione che ne risulta.
Detto $\tau$ il volume della scatola, se $E$ è l'energia del sistema la pressione è data da $P=-\drp{E}{\tau}$.
Possiamo calcolarla con l'aiuto del seguente teorema.
\begin{teorema} \label{t:dipendenza-hamiltoniano-parametro}
	Se l'hamiltoniano dipende da un parametro $\lambda$, allora il valore di aspettazione di $\drp{}{\lambda}\op H(\lambda)$ calcolato in un autostato di $\op H$ con autovalore $E$ è
	\begin{equation}
		\bra{E}\drp{\op H}{\lambda}\ket{E}=\drp{E}{\lambda}=\drp{}{\lambda}\bra{E}\op H\ket{E}.
	\end{equation}
\end{teorema}
\begin{proof}
	Sappiamo che $(\op H-E)\ket{E}=0$, perciò derivando rispetto a $\lambda$ otteniamo
	\begin{equation}
		0=\drp{}{\lambda}\big[(\op H-E)\ket{E}\big]=\drp{}{\lambda}(\op H-E)\ket{E}-(\op H-E)\drp{}{\lambda}\ket{E}
	\end{equation}
	e applicando $\bra{E}$ troviamo
	\begin{equation}
		0=\bra{E}\drp{\op H}{\lambda}-\drp{E}{\lambda}\ket{E}+\bra{E}(\op H-E)\drp{}{\lambda}\ket{E}
	\end{equation}
	ma $\op H-E$ è hermitiano dunque $\bra{E}(\op H-E)=0$, e troviamo dunque
	\begin{equation}
		\bra{E}\drp{\op H}{\lambda}\ket{E}=\drp{E}{\lambda}\braket{E}{E}=\drp{E}{\lambda}.\qedhere
	\end{equation}
\end{proof}
Tornando alla particella nella scatola, l'hamiltoniano del sistema si può scomporre nelle tre direzioni, in tre addendi $\op H_i$, che commutano, e si possono diagonalizzare contemporaneamente.
Detti $a_1$, $a_2$ e $a_3$ le lunghezze dei tre lati della scatola si ha dunque
\begin{equation}
	E(a_1,a_2,a_3)=\frac{\pi^2\hbar^2}{2m}\sum_{i=1}^3\frac{n_i^2}{a_i^2}
\end{equation}
Dato che $\tau=a_1a_2a_3$, la pressione esercitata sulle pareti, con la formula data precedentemente, è dunque
\begin{multline}
	P=-\drp{E}{\tau}=-\sum_{i=1}^3\drp{E}{a_i}\drp{a_i}{\tau}=
	\frac{\pi^2\hbar^2}{m}\sum_{i=1}^3\frac{n_i^2}{a_i^3}\bigg(\drp{\tau}{a_i}\bigg)^{-1}=\\=
	\frac{\pi^2\hbar^2}{m}\bigg(\frac{n_1^2}{a_1^3a_2a_3}+\frac{n_2^2}{a_1a_2^3a_3}+\frac{n_3^2}{a_1a_2a_3^3}\bigg)=
	\frac{\pi^2\hbar^2}{m}\frac{(n_1a_2a_3)^2+(n_2a_1a_3)^2+(n_3a_1a_2)^2}{\tau^3}.
\end{multline}

\section{Potenziale delta attrattivo}
Dopo una buca di potenziale ``standard'', vediamone la versione infinitamente alta e stretta, modellizzata dall'equazione del potenziale
\begin{equation}
	V(x)=-\lambda\delta(x).
	\label{eq:potenziale-delta-attrattivo}
\end{equation}
Possiamo risolvere questo problema in tre metodi diversi, che ora presentiamo.
\paragraph{Limite della buca di altezza finita}
Riprendendo direttamente l'esempio precedente, consideriamo una buca centrata nell'origine, di altezza $V_0$ e larghezza $a$, e prendiamone il limite per $a\to 0$, $V_0\to+\infty$ tali che $aV_0\to\lambda$, un valore finito.
Prendiamo un potenziale della forma
\begin{equation}
	V(x)=
	\begin{cases}
		-V_0	&x\in\big[-\frac{a}2,\frac{a}2\big]\\
		0		&x\in\big(-\infty,-\frac{a}2\big)\cup\big(\frac{a}2,+\infty\big)
	\end{cases}
\end{equation}
Il potenziale è sempre pari, dunque la funzione d'onda dello stato fondamentale è anch'essa pari.
Scegliamo arbitrariamente il suo valore nell'origine ponendo $\psi(0)=1$, e in modo simile alle equazioni \eqref{eq:soluzione-generale-schrodinger-buca-finita} otteniamo\footnote{La scelta di porre $\psi(0)=1$ non lede la generalità del problema, dato che la funzione d'onda è poi da normalizzare.}
\begin{equation}
	\psi(x)=
	\begin{cases}
		\cos\big(k\frac{a}2\big)e^{\alpha(x+\frac{a}2)}		&x<-\frac{a}2\\
		\cos\big(kx\big)									&\abs{x}\le\frac{a}2\\
		\cos\big(k\frac{a}2\big)e^{-\alpha(x-\frac{a}2)}	&x>\frac{a}2
	\end{cases}
	\label{eq:soluzione-generale-delta-attrattivo}
\end{equation}
dove $k=\sqrt{\frac{2m(E+V_0)}{\hbar^2}}$ e $\alpha=\sqrt{\frac{-2mE}{\hbar^2}}$, con un'energia iniziale del sistema $-V_0<E<0$.
Per come è scritta, la $\psi$ è già continua; imponendo anche $\psi\in\cont{1}(\R)$ otteniamo
\begin{equation}
	k\sin\frac{ka}2=\alpha\cos\frac{ka}2\qqq k\tan\frac{ka}2=\alpha
	\label{eq:condizione-C1-delta-attrattivo}
\end{equation}
e portando al limite $V_0\to+\infty$ si ottiene $ka\to 0$, dunque possiamo approssimare
\begin{equation}
	\alpha=k\tan\frac{ka}2\approx k^2\frac{a}2,
\end{equation}
da cui troviamo l'energia dello stato fondamentale
\begin{equation}
	-\frac{2mE}{\hbar^2}=\frac{m^2\lambda^2}{\hbar^4}\qqq E=-\frac{m\lambda^2}{2\hbar^2}.
	\label{eq:E-fondamentale-delta-attrattivo}
\end{equation}
Con $a\to 0$ si ha inoltre la funzione d'onda $\psi(x)=e^{-\alpha\abs{x}}$.
La sua derivata non è continua in $x=0$, ma non è un problema dato che il potenziale ha una discontinuità infinita in quel punto.
Risulta, in particolare, $\psi'(x)=-\alpha e^{-\alpha\abs{x}}\sgn x$, che integrata in $[-\epsilon,\epsilon]$ dà
\begin{equation}
	\psi'(\epsilon)-\psi'(-\epsilon)=\int_{-\epsilon}^{\epsilon}-\alpha e^{-\alpha\abs{x}}\sgn x\,\dd x=-2\alpha=-\frac{2m\lambda}{\hbar^2}\psi(0)=-\frac{2m\lambda}{\hbar^2}
	\label{eq:discontinuita-derivata-wf-delta-attrattivo}
\end{equation}
che misura la discontinuità di $\psi'$ nell'origine.

\paragraph{Equazione di Schr\"odinger}
Scriviamo direttamente l'equazione di Schr\"odinger del sistema, con energia $E$, che è
\begin{equation}
	-\frac{\hbar^2}{2m}\psi''(x)-\lambda\delta(x)\psi(x)=E\psi(x)\qqq \psi''(x)=-\frac{2m}{\hbar^2}\big[E+\lambda\delta(x)\big]\psi(x).
	\label{eq:schrodinger-delta-attrattivo}
\end{equation}
Integrando i due membri dell'equazione nell'intervallo $[-\epsilon,\epsilon]$ abbiamo
\begin{equation}
	\begin{split}
		\psi'(\epsilon)-\psi'(-\epsilon)&=\int_{-\epsilon}^{\epsilon}-\frac{2m}{\hbar^2}\big[E+\lambda\delta(x)\big]\psi(x)\,\dd x=\\
		&=-\frac{2mE}{\hbar^2}\int_{-\epsilon}^{\epsilon}\psi(x)\,\dd x-\frac{2m\lambda}{\hbar^2}\int_{-\epsilon}^{\epsilon}\delta(x)\psi(x)\,\dd x=\\
		&=-\frac{2mE}{\hbar^2}\int_{-\epsilon}^{\epsilon}\psi(x)\,\dd x-\frac{2m\lambda}{\hbar^2}\psi(0)=\\
		&=-\frac{2m\lambda}{\hbar^2}\psi(0)
	\end{split}
\end{equation}
nel limite, all'ultimo passaggio, per $\epsilon\to 0^+$, e dato che $\psi$ è continua il primo addendo è nullo.
Escludendo l'origine, l'equazione di Schr\"odinger si scrive più semplicemente come $\psi''(x)=-\frac{2mE}{\hbar^2}\psi(x)$, che ha come soluzione $\psi(x)=\psi(0)\exp(-\alpha\abs{x})$ dove $\alpha$ è definita come prima.
Imponendo la discontinuità di $\psi'$ nell'origine, appena trovata integrando la \eqref{eq:schrodinger-delta-attrattivo}, troviamo lo stesso valore per $\alpha$ del paragrafo precedente; si ha un solo valore di energia ammissibile.

\paragraph{Spazio degli impulsi}
Nella notazione di Dirac la \eqref{eq:schrodinger-delta-attrattivo} si scrive, in uno stato legato $\ket{E}$, come
\begin{equation}
	\bigg(\frac1{2m}\op p^2+V(\op q)\bigg)\ket{E}=E\ket{E}\qqq\bigg(\frac1{2m}\op p^2-E\bigg)\ket{E}=-V(\op q)\ket{E}.
	\label{eq:schrodinger-delta-attrattivo-dirac}
\end{equation}
Anzich\'e rappresentarla nella base della posizione, lavoriamo nella base degli impulsi: moltiplicando per $\bra{p}$ otteniamo
\begin{equation}
	\frac1{2m}\bra{p}\op p^2\ket{E}-E\braket{p}{E}=-\bra{p}V(\op q)\ket{E}
\end{equation}
da cui (chiamiamo ancora $\psi$ la funzione d'onda nello spazio degli impulsi)
\begin{equation}
	\frac{p^2}{2m}\psi(p)-E\psi(p)=-\int_{-\infty}^{+\infty}\bra{p}V(\op q)\ket{p'}\braket{p'}{E}\,\dd p'=-\int_{-\infty}^{+\infty}\bra{p}V(\op q)\ket{p'}\psi(p')\,\dd p'.
\end{equation}
Calcoliamo l'elemento di matrice $\bra{p}V(\op q)\ket{p'}$: l'espressione del potenziale nella base della posizione è troppo comoda per non essere usata, dunque introduciamo a destra del potenziale una risoluzione dell'identità con gli autostati $\ket{x}$ della posizione, ottenendo
\begin{multline}
	\bra{p}V(\op q)\ket{p'}=\int_{-\infty}^{+\infty}\bra{p}V(\op q)\ket{x}\braket{x}{p'}\,\dd x=\int_{-\infty}^{+\infty}V(x)\braket{p}{x}\braket{x}{p'}\,\dd x=\\
	=\frac1{2\pi\hbar}\int_{-\infty}^{+\infty}V(x)\exp\bigg[\frac{i}{\hbar}(p'-p)x\bigg]\,\dd x=-\frac{\lambda}{2\pi\hbar}\int_{-\infty}^{+\infty}\delta(x)\exp\bigg[\frac{i}{\hbar}(p'-p)x\bigg]\,\dd x=-\frac{\lambda}{2\pi\hbar}.
\end{multline}
Sostituendo questo nell'equazione precedente, troviamo dunque
\begin{equation}
	\bigg(\frac{p^2}{2m}-E\bigg)\psi(p)=\frac{\lambda}{2\pi\hbar}\int_{-\infty}^{+\infty}\psi(p')\,\dd p'=\frac{\lambda}{\sqrt{2\pi\hbar}}\psi_0
\end{equation}
dove $\psi_0$ è il valore della funzione d'onda \emph{delle coordinate} nel punto $x=0$.
Troviamo infine
\begin{equation}
	\psi(p)=\frac{\lambda\psi_0}{\sqrt{2\pi\hbar}}\frac1{\frac{p^2}{2m}-E}.
	\label{eq:wf-impulsi-delta-attrattivo}
\end{equation}
Integrando in $p$ (su tutto $\R$) e dividendo per $\sqrt{2\pi\hbar}$ i due membri si può trovare la relazione che lega $E$ a $\lambda$, con
\begin{equation}
	\frac1{\sqrt{2\pi\hbar}}\int_{-\infty}^{+\infty}\psi(p)\,\dd p=\psi_0\frac1{2\pi\hbar}\int_{-\infty}^{+\infty}\frac1{\frac{p^2}{2m}-E}\,\dd p
\end{equation}
da cui, considerando $\psi$ normalizzata, e ricordando che $\psi_0=1$ (dai paragrafi precedenti) affinch\'e sia normalizzata anche la funzione d'onda nella base della posizione, si ottiene l'equazione
\begin{equation}
	\int_{-\infty}^{+\infty}\frac1{p^2+2mE}\,\dd p=\frac{\pi\hbar}{m\lambda}.
\end{equation}

\section{Coefficiente di trasmissione}
Prendiamo un sistema con un potenziale monotono crescente da $-\infty$ in cui è nullo a $+\infty$ in cui assume un certo valore $V_0>0$, e immaginiamo un fascio di particelle incidente ``da sinistra'', ossia da $-\infty$; esse si propagano fino al \emph{gradino di potenziale}: classicamente, esse possono superarlo e proseguire a destra solo se la loro energia $E$ inizial è maggiore dell'altezza $V_0$, altrimenti tornano indietro.
Nella meccanica quantistica, come ci si può ormai aspettare, anche con energie minori di $V_0$ la probabilità di trovare la particella oltre il gradino non sarà nulla: possiamo ricavare un coefficiente di trasmissione per quantificare questa probabilità.

Un procedimento analogo si usa per modellizzare un urto tra particelle (o tra fasci di esse, se vogliamo trascurare il tempo): il potenziale è nullo sull'asse reale tranne che in una certa regione $S$ in cui non sappiamo il suo comportamento.
Essendo $V(x)=0$ per $x\notin S$, in tale regione la particella sarà libera e la sua funzione d'onda sarà un'onda piana della forma
\begin{equation}
	\psi(x)=
	\begin{cases}
		A'e^{ikx}+Ae^{-ikx}	&x\to-\infty\\
		Ce^{ikx}+C'e^{-ikx}	&x\to+\infty
	\end{cases}
\end{equation}
e le due soluzioni sono indipendenti.
Ora assegnamo alcune delle costanti: studiando in particolare un sistema composto da un fascio di particelle incidente da sinistra, avremo sicuramente $A'=1$, mentre poich\'e da destra non indice alcuna particella avremo analogamente $C'=0$.
Con queste scelte la funzione d'onda può essere vista come la sovrapposizione di
\begin{itemize}
	\item un'onda incidente da sinistra $e^{ikx}$;
	\item un'onda trasmessa $Ce^{ikx}$ oltre alla regione $S$, verso destra;
	\item un'onda riflessa $A'e^{-ikx}$ dalla regione $S$ verso sinistra.
\end{itemize}
Essi soddisfano, come per delle reali onde incidenti su una superficie, l'equazione $\abs{A}^2+\abs{C}^2=1$.
Calcoliamo infatti la densità di corrente di probabilità dei due termini: nella regione a sinistra di $S$ abbiamo
\begin{gather*}
	\psi^*(x)\drv{}{x}\psi(x)=ik(e^{ikx}-Ae^{-ikx})(e^{-ikx}+A^*e^{ikx})=ik(1-Ae^{-2ikx}+A^*e^{2ikx}-\abs{A}^2)\\
	\psi(x)\drv{}{x}\psi^*(x)=-ik(e^{ikx}+Ae^{-ikx})(e^{-ikx}-A^*e^{ikx})=-ik(1+Ae^{-2ikx}-A^*e^{2ikx}-\abs{A}^2).
\end{gather*}
Nella regione a destra invece
\begin{gather*}
	\psi^*(x)\drv{}{x}\psi(x)=ikCe^{ikx}C^*e^{-ikx}=ik\abs{C}^2\\
	\psi(x)\drv{}{x}\psi^*(x)=-ikCe^{ikx}C^*e^{-ikx}=-ik\abs{C}^2
\end{gather*}
di conseguenza nelle regioni a sinistra e a destra di $S$ si ha rispettivamente
\begin{equation}
	\begin{aligned}
		J_-(x)&=-\frac{i\hbar}{2m}ik(1-Ae^{-2ikx}+A^*e^{2ikx}-\abs{A}^2+1+Ae^{-2ikx}-A^*e^{2ikx}-\abs{A}^2)=\frac{\hbar k}{m}(1-\abs{A}^2)\\
		J_+(x)&=-\frac{i\hbar}{2m}(ik\abs{C}^2+ik\abs{C}^2)=\frac{\hbar k}{m}\abs{C}^2.
	\end{aligned}
\end{equation}
Notiamo inoltre che in questo caso, poich\'e la funzione d'onda è una sovrapposizione di autostati dell'impulso, ossia lo stato $\ket{\psi}$ a cui è associata è della forma $\ket{\psi}=\ket{k}\braket{k}{\psi}+\ket{-k}\braket{-k}{\psi}$, la densità corrente di probabilità è la somma delle densità di ciascun autostato, ossia $J_-$ è la somma della densità di corrente dell'onda $e^{ikx}$ (che vale $\frac{\hbar k}{m}$) e di quella dell'onda $Ae^{-ikx}$ (che è $\frac{\hbar k}{m}\abs{A}^2$).

Chiamiamo dunque $j_i$ la densità di corrente di probabilità dell'onda incidente, $j_r$ di quella riflessa e $j_t$ di quella trasmessa; con ciò troviamo $J_-=j_i+j_r$ e $J_+=j_t$.
Definiamo dunque il \emph{coefficiente di trasmissione} $T$ come il rapporto tra la densità di corrente di probabilità trasmessa e incidente, e quello di \emph{riflessione} come il rapporto tra riflessa e incidente:
\begin{equation}
	T\defeq\frac{j_t}{j_i},\hspace{1cm} R\defeq\frac{j_r}{j_i}
	\label{eq:coefficienti-trasmissione-riflessione}
\end{equation}
e in questo caso ovviamente $T=\abs{A}^2$ e $R=\abs{C}^2$.
Notiamo che vale, come in ottica, la relazione $T+R=1$.

\section{Effetto tunnel}
Applichiamo i risultati della sezione precedente ad un problema preciso, con un particella soggetta al potenziale $V(x)=V_0\chi_{[0,a]}(x)$, che forma una sorta di barriera.
Se l'energia $E$ è maggiore dell'altezza della barriera, chiaramente la particella può superarla, ma questo può succedere anche se $E<V_0$, dando luogo a un effetto noto come \emph{effetto tunnel} quantistico.
Troveremo una probabilità non nulla di misurare la particella anche per $x>a$.
\begin{figure}
	\tikzsetnextfilename{tunneling}
	\centering
	\begin{tikzpicture}
		\begin{axis}[
				standard,
				width=.8\linewidth,
				height=.4\linewidth,
				xlabel=$x$, ylabel=$V(x)$,
				xmin=-6, xmax=6,
				ymin=-1, ymax=3,
				xtick=\empty, ytick=\empty
			]
			\addplot[thick] coordinates {(-6,0) (0,0) (0,3) (2,3) (2,0) (6,0)};
			\addplot[dashed,domain=-6:0] function {1.5};
			\addplot[dashed,domain=0:2] function {1.5*exp(-x)};
			\addplot[dashed,domain=2:6] function {1.5*exp(-2)};
			\legend{$V(x)$,$\abs{\psi(x)}^2$}
		\end{axis}
	\end{tikzpicture}
	\caption{Un sistema con potenziale $V=V_0\chi_{[0,a]}$: la particella incide nella ``barriera'' da destra come un'onda piana, essendo una particella libera. Anche con un'energia minore di $V_0$ può però penetrare la barriera, in cui la densità di probabilità decresce esponenzialmente con la profondità (è una regione classicamente non accessibile). La probabilità di trovare la particella oltre la barriera, seppur molto piccola, è comunque non nulla.}
	\label{fig:tunneling}
\end{figure}

Posti $k\defeq\Big(\frac{2mE}{\hbar^2}\Big)^{1/2}$ e $\lambda\defeq\Big(\frac{2m(V_0-E)}{\hbar^2}\Big)^{1/2}$, l'equazione di Schr\"odinger ha come soluzione la funzione d'onda (negli autostati della posizione)
\begin{equation}
	\psi(x)=
	\begin{cases}
		e^{ikx}+Ae^{-ikx}				&x<0\\
		Be^{\lambda x}+B'e^{-\lambda x}	&0\ge x\ge a\\
		Ce^{ik(x-a)}					&x>a
	\end{cases}
	\label{eq:tunnel-wf}
\end{equation}
Notiamo che nell'espressione di $\psi$ per $x>a$ abbiamo scritto $e^{ik(x-a)}$ anzich\'e $e^{ikx}$: questo non modifica la soluzione, in quanto la modifica consiste soltanto nel moltpilicare $\psi$ per un fattore di fase $e^{-ika}$, per cui ciò che cambia è solo la costante $C$, che comunque è arbitraria.
Imponendo la continuità della funzione e della sua derivata nei punti $x=0$ e $x=a$ abbiamo rispettivamente
\begin{equation}
	\begin{cases}
		1+A=B+B'\\
		ik(1-A)=\lambda(B-B')
	\end{cases}
	\qeq
	\begin{cases}
		Be^{\lambda a}+B'e^{-\lambda a}=C\\
		\lambda(Be^{\lambda a}-B'e^{-\lambda a})=ikC
	\end{cases}
\end{equation}

Troviamo dunque l'andamento dei coefficienti $\abs{A}^2$ e $\abs{C}^2$ in funzione del rapporto $E/V_0$.
Scrivendo $e^x=\cosh x+\sinh x$, risulta
\begin{equation}
	\begin{gathered}
		C=B\big[\cosh(\lambda a)+\sinh(\lambda a)\big]+B'\big[\cosh(\lambda a)-\sinh(\lambda a)\big]=(1+A)\cosh(\lambda a)+\frac{ik}{\hbar}(1-A)\sinh(\lambda a)\\
		C=-\frac{i\lambda}{k}\bigg[\frac{ik(1-A)}{\lambda}\cosh(\lambda a)+(1+A)\sinh(\lambda a)\bigg]=(1-A)\cosh(\lambda a)-\frac{i\lambda}{k}(1+A)\sinh(\lambda a)
	\end{gathered}
\end{equation}
e sottraendole abbiamo
\begin{equation}
	2A\cosh(\lambda a)+i\bigg(\frac{k}{\lambda}+\frac{\lambda}{k}\bigg)\sinh(\lambda a)-i\bigg(\frac{k}{\lambda}-\frac{\lambda}{k}\bigg)A\sinh(\lambda a)=0
\end{equation}
da cui i valori
\begin{equation}
	A=\frac{-i\Big(\frac{\lambda}{k}-\frac{k}{\lambda}\Big)\sinh(\lambda a)}{2\cosh(\lambda a)-i\Big(\frac{k}{\lambda}-\frac{\lambda}{k}\Big)\sinh(\lambda a)}
	\hspace{1cm}
	C=\frac{2}{2\cosh(\lambda a)-i\Big(\frac{k}{\lambda}-\frac{\lambda}{k}\Big)\sinh(\lambda a)}.
\end{equation}
Calcolando i moduli al quadrato, abbiamo i coefficienti di trasmissione e riflessione
\begin{equation}
	\abs{A}^2=\frac{V_0^2\sinh^2(\lambda a)}{4E(V_0-E)+V_0^2\sinh^2(\lambda a)}\qeq\abs{C}^2=\frac{4E(V_0-E)}{4E(V_0-E)+V_0^2\sinh^2(\lambda a)}.
	\label{eq:coefficienti-trasmissione-riflessione-tunneling}
\end{equation}

