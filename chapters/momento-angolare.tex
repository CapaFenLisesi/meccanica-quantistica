\chapter{Momento angolare}
L'operatore associato al momento angolare è, come in meccanica classica, il generatore delle rotazioni, che soddisfa le relazioni di commutazione
\begin{equation}
	[\op L_i,\op L_j]=i\hbar\epsilon_{ijk}\op L_k.
	\label{eq:commutazione-momento-angolare}
\end{equation}
A meno di fattori moltiplicativi, le costanti di struttura (la parte rilevante è il simbolo di Levi-Civita $\epsilon_{ijk}$) sono le stesse dell'algebra di Lie $\mathfrak{su}(2)$.
Non a caso, $\mathfrak{su}(2)$ è isomorfa a $\mathfrak{so}(3)$ che è l'algebra del gruppo $SO(3)$ delle rotazioni tridimensionali; in altre parole $\mathfrak{su}(2)$ genera $SO(3)$, allo stesso modo in cui il momento angolare genera le rotazioni in $\R^3$, e ciò giustifica la \eqref{eq:commutazione-momento-angolare}.
Una quantità che commuta con il momento angolare lungo una data direzione è dunque simmetrica per le rotazioni attorno a tale asse: gli scalari sono degli importanti esempi di queste quantità.
D'ora in poi, elimineremo sistematicamente la costante $\hbar$ dalle equazioni (a meno di trovarci in casi ambigui che la richiedano esplicitamente), sottindendendo che il momento angolare è misurato \emph{in unità di $\hbar$}.
In pratica, dividiamo la \eqref{eq:commutazione-momento-angolare} per $\hbar^2$ ottenendo la forma più semplice $[\op L_i,\op L_j]=i\epsilon_{ijk}\op L_k$.
Insieme alle tre componenti cartesiane del momento angolare abbiamo anche il quadrato del suo modulo, ossia $\op L^2=\op L_i\op L_i$: essendo uno scalare, ci aspettiamo che sia invariante per rotazioni ossia che commuti con tutte le componenti del momento angolare.
Una verifica diretta mostra infatti che, per $i=1,2,3$,
\begin{equation}
	[\op L^2,\op L_i]=[\op L_k\op L_k,\op L_i]=\op L_k[\op L_k,\op L_i]+[\op L_k,\op L_i]\op L_k=i\epsilon_{kil}(\op L_k\op L_l+\op L_l\op L_k)=i(\epsilon_{kil}+\epsilon_{lik})\op L_k\op L_l=0
\end{equation}
per l'antisimmetria di $\epsilon_{kil}$.
In ogni caso le tre componenti $\op L_i$ non sono tra loro compatibili (dalla \eqref{eq:commutazione-momento-angolare}) a meno del caso banale in cui siano tutte nulle.
Nella costruzione di un sistema completo di osservabili compatibili, è convenzione scegliere insieme a $\op L^2$ anche la componente $z$, ossia $\op L_3$, da diagonalizzare simultaneamente.

\section{Autovalori del momento angolare}
Supponiamo di aver trovato un sistema completo di osservabili compatibili che contenga $\op L^2$ e $\op L^3$.
Chiamiamo $m$ gli autovalori di $\op L_3$ e $\mu^2$ quelli di $\op L^2$, e trascurando lo altre osservabili del sistema indichiamo gli autostati simultanei con $\ket{m,\mu^2}$, che sono quindi tali per cui\footnote{Gli autovalori di $\op L^2$ sono evidentemente positivi, perciò li indichiamo direttamente con il quadrato $\mu^2$.}
\begin{equation}
	\op L^2\ket{m,\mu^2}=\mu^2\ket{m,\mu^2}\qeq\op L_3\ket{m,\mu^2}=m\ket{m,\mu^2}.
\end{equation}
Come già per l'oscillatore armonico, introduciamo gli operatori a scala $\op L_+\defeq\op L_1+i\op L_2$ e $\op L_-\defeq\op L_1-i\op L_2$.
Dato che gli $\op L_i$ sono hermitiani (poich\'e rappresentano delle osservabili) si vede subito che $\adj{\op L_+}=\op L_-$.
Essi sono in relazione con la componente scelta del momento angolare con i commutatori
\begin{equation}
	[\op L_3,\op L_+]=[\op L_3,\op L_1]+i[\op L_3,\op L_2]=i\op L_2-i^2\op L_1=\op L_+
\end{equation}
da cui
\begin{equation}
	[\op L_3,\op L_-]=[\op L_3,\adj{\op L_+}]=-\adj{[\op L_3,\op L_+]}=-\adj{\op L_+}=-\op L_-.
\end{equation}
Vediamo dunque come questi due operatori agiscono sugli autostati:
\begin{multline}
	\op L_3\op L_+\ket{m,\mu^2}=(\op L_3\op L_+ - \op L_+\op L_3 + \op L_+\op L_3)\ket{m,\mu^2}=[\op L_3,\op L_+]\ket{m,\mu^2}+\op L_+\op L_3\ket{m,\mu^2}=\\=
	\op L_+\ket{m,\mu^2}+m\op L_+\ket{m,\mu^2}=(m+1)\op L_+\ket{m,\mu^2}
\end{multline}
e analogamente $\op L_3\op L_-\ket{m,\mu^2}=(m-1)\op L_-\ket{m,\mu^2}$.
Gli operatori $\op L_+$ e $\op L_-$ dunque portano un autostato di $\op L_3$ in un altro autostato con autovalore, rispettivamente, aumentato o diminuito di 1.
In altri termini, possiamo scriverlo come $\op L_+\ket{m,\mu^2}=\ket{m+1,\mu^2}$ e $\op L_-\ket{m,\mu^2}=\ket{m-1,\mu^2}$.

Dobbiamo quindi chiederci se la sequenza $\{\dots,m-2,m-1,m,m+1,\dots\}$ sia limitata o meno.
Dato che $\op L_3^2=\op L^2-\op L_1^2+\op L_2^2$, sicuramente per ciascuno stato il valore di aspettazione di $\op L_3^2$ deve essere minore di quello di $\op L^2$, dato che ovviamente $\op L_1^2+\op L_2^2$ è un operatore definito positivo.
In particolare, dunque, risulta $m^2\le\mu^2$, ossia $\abs{m}\le\abs{\mu}$: gli autovalori sono allora necessariamente limitati; esiste un certo stato che chiamiamo $\ket{j_+,\mu^2}$ tale per cui $\op L_+\ket{j_+,\mu^2}=0$ e analogamente uno stato $\ket{j_-,\mu^2}$ per cui $\op L_-\ket{j_-,\mu^2}=0$, dove $j_+$ e $j_-$ sono il massimo e il minimo autovalore, rispettivamente, di $\op L_3$.
Per questi due stati si ha comunque $\op L^2\ket{j_\pm,\mu^2}=\mu^2\ket{j_\pm,\mu^2}$.
Calcolando la norma di $\op L_+\ket{j_+,\mu^2}$ troviamo
\begin{equation}
	\begin{split}
		0&=\bra{j_+,\mu^2}\adj{\op L_+}\op L_+\ket{j_+,\mu^2}=\\
		&=\bra{j_+,\mu^2}\op L_-\op L_+\ket{j_+,\mu^2}=\\
		&=\bra{j_+,\mu^2}(\op L_1-i\op L_2)(\op L_1+i\op L_2)\ket{j_+,\mu^2}=\\
		&=\bra{j_+,\mu^2}(\op L_1^2+\op L_2^2+i[\op L_1,\op L_2])\ket{j_+,\mu^2}=\\
		&=\bra{j_+,\mu^2}(\op L^2-\op L_3^2-\op L_3])\ket{j_+,\mu^2}=\\
		&=(\mu^2-j_+^2-j_+)\braket{j_+,\mu^2}{j_+,\mu^2}
	\end{split}
\end{equation}
da cui otteniamo $\mu^2=j_+(j_++1)$.
Dalla norma di $\op L_-\ket{j_-,\mu^2}$, anch'essa nulla, troviamo invece $\mu^2=j_-(j_--1)$.
Assegnato un valore a $j_-$, uguagliando le due espressioni abbiamo $j_j(j_++1)=j_-(j_--1)$ da cui $j_+=-j_-$ oppure $j_+=j_--1$.
La seconda delle due però non è accettabile dato che per costruzione $j_+\ge j_-$, perciò rimane $j_+=-j_-$.
Chiamiamo questo valore semplicemente con $j$.
Abbiamo quindi ottenuto lo spettro di $\op L_3$ che è l'insieme $\{-j,-j+1,\dots,j-1,j\}$, che ha $2j+1$ elementi.
Questa cardinalità è chiaramente un numero intero, perciò $j$ deve essere a sua volta intero oppure semiintero (positivo o nullo).
Una volta che $\op L_3$ assume uno di questi valori, poi, $\op L^2$ ha come autovalore $j(j+1)$: tale numero non è un quadrato perfetto, per il fatto che anche quando $j=m$ (il massimo autovalore di $\op L_3$) non si ha comunque $\mu^2=m^2$, perch\'e da questo seguirebbe che $\op L_1$ e $\op L_2$ avrebbero solo autovalori nulli, ossia il momento angolare lungo i due assi restanti sarebbe nullo; sappiamo che questo non è possibile, perch\'e non possiamo determinare con precisione assoluta contemporaneamente due componenti del momento angolare.
L'unica eccezione a questo si ha nel caso banale in cui $j=0$, in cui $\op{\vec L}=0$.

In questa rappresentazione del momento angolare, inoltre, $\op L^2$ è un multiplo dell'identità (detto anche \emph{operatore di Casimir}): è degenere dato che ad ogni suo autostato corrispondono $2j+1$ autostati linearmente indipendenti (di $\op L_3$), come si poteva anche capire dal teorema \ref{t:degenerazione} oppure dal lemma di Schur.
La rappresentazione è inoltre determinata completamente determinata dal numero $j$.\footnote{Ogni rappresentazione del gruppo delle rotazioni è univocamente determinata da un numero intero o semiintero positivo o nullo. Tutte le rappresentazioni irriducibili si ottengono in questo modo, mentre quelle riducibili si ricavano come somma diretta o prodotto di queste.}
Ad esempio la rappresentazione con $j=\frac12$ è data dalle matrici di Pauli: si ha infatti $\sigma_3=\frac12\begin{psmallmatrix}1&0\\0&-1\end{psmallmatrix}$ e $\frac14(\sigma_1^2+\sigma_2^2+\sigma_3^2)=\op L^2=\frac34=j(j+1)$.

\section{Momento angolare orbitale}
Possiamo definire l'operatore del momento angolare \emph{orbitale} di una particella attorno all'origine, in coordinate cartesiane, utilizzando la definizione classica e sostituendo gli operatori corrispondenti: otteniamo
\begin{equation}
	\op L_k=\epsilon_{ijk}\op x_i\op p_j
	\label{eq:momento-angolare-orbitale}
\end{equation}
in dimensione 3 (riprendiamo qui l'uso di $\hbar$ nelle equazioni).
Si può eventualmente generalizzare a un numero differente di dimensioni seguendo le regole di commutazione dell'algebra $\mathfrak{so}(n)$.
Nella rappresentazione di Schr\"odinger della posizione troviamo che $\op L_i$ è dato da
\begin{equation}
	L_i=\epsilon_{ijk}x_j\bigg(-i\hbar\drp{}{x_k}\bigg)=-i\hbar\epsilon_{ijk}x_j\drp{}{x_k}.
\end{equation}
Possiamo ottenere però una rappresentazione più vantaggiosa usando le coordinate polari, più adatte a descrivere ad esempio sistemi a simmetria centrale, cioè invarianti per rotazioni.
Usando la terna di coordinate $(r,\phi,\theta)\in[0,+\infty)\times[0,2\pi]\times[0,\pi]$ troviamo che $\op L_3$ è rappresentato dall'operatore
\begin{equation}
	-i\hbar\drp{}{\phi}.
\end{equation}
La quantizzazione del momento angolare è dovuta, tra il resto, al fatto che la variabile angolare $\phi$ è in un intervallo limitato, dunque la parte in $\phi$ delle autofunzioni deve risultare periodica di $2\pi$, cos\`i come una funzione definita su un intervallo limitato si può analizzare in serie di Fourier, e ammette dei modi normali di oscillazione con un insieme discreto (numerabile) di frequenze.
Le autofunzioni di questo operatore sono funzioni $f(r,\phi,\theta)$ tali che
\begin{equation}
	-i\hbar\drp{f}{\phi}(r,\phi,\theta)=m\hbar f(r,\phi,\theta)\qqq f(r,\phi,\theta)=g(r,\theta)e^{im\phi}.
\end{equation}
Imponendo la periodicità della funzione in $\phi$ otteniamo che deve soddisfare
\begin{equation}
	g(r,\theta)e^{im(\phi+2\pi)}=g(r,\theta)e^{im\phi}e^{2im\pi}=g(r,\theta)e^{im\phi}
\end{equation}
da cui ricaviamo che $m$ deve essere intero.

Rimane da rappresentare anche $\op L^2$: dall'identità, valida in una generica dimensione $d$,
\begin{equation}
	\op{\vec L}^2=\op{\vec x}^2\op{\vec p}^2-(\scalar{\op{\vec x}}{\op{\vec p}})^2+i\hbar(d-2)\scalar{\op{\vec x}}{\op{\vec p}}
	\label{eq:relazione-L2-x-p}
\end{equation}
possiamo anche ottenere una comoda rappresentazione di $\op{\vec p}^2$ in coordinate polari: è sufficiente moltiplicare a sinistra per l'operatore $\op{\vec x}^{-2}$ (l'inverso del quadrato della norma).
In pratica, la norma al quadrato dell'operatore di posizione corrisponde al quadrato della distanza dall'origine, cioè $r^2$, che sarà l'autovalore di questo operatore in una rappresentazione in cui è diagonale.

Un'operatore hamiltoniano della forma $\frac1{2m}\op{\vec p}^2+V(\norm{\op{\vec x}}^2)$ è invariante per rotazioni e commuta dunque con $\op{\vec L}^2$ e $\op L_3$.
In generale sarà un operatore degenere: possiamo etichettare gli autostati con tre numeri $E,l,m$ tali che, oltre all'equazione di Schr\"odinger, abbiamo
\begin{equation}
	L^2\psi_{E,l,m}(\vec x)=l(l+1)\hbar^2\psi_{E,l,m}(\vec x)\qeq L_3\psi_{E,l,m}(\vec x)=m\hbar\psi_{E,l,m}(\vec x).
	\label{eq:autofunzioni-momento-angolare}
\end{equation}
Vediamo dunque come rappresentare opportunamente $\op{\vec p}^2$ in $L^2(\R^3)$ con l'identità \eqref{eq:relazione-L2-x-p}: nella rappresentazione delle coordinate $\op{\vec x}^2$ è diagonale si traduce nella moltiplicazione per $r^2$ (analogamente per $\op{\vec x}^{-2}$), mentre
\begin{equation}
	\scalar{\op{\vec x}}{\op{\vec p}}=-i\hbar x_i\drp{}{x_i}=-i\hbar r\frac{x_i}{r}\drp{}{x_i}=r\drp{r}{x_i}\drp{}{x_i}=r\drp{}{r}.
\end{equation}
Analogamente alla coppia di coordinate cartesiane, possiamo definire un operatore $\op r\defeq\sqrt{\op x^2+\op y^2+\op z^2}$ (in senso generalizzato: abbiamo gli stessi problemi di $x$ e $p$), che risulta autoaggiunto, e un momento coniugato $\op p_r\defeq-i\hbar\drp{}{r}$.
Quest'ultimo però non è autoaggiunto: se per verificarlo nel casi dei $\op p_i$ cartesiani era sufficiente integrare per parti dato che i termini al contorno (per $\abs{x}\to+\infty$) erano nulli, in questo caso il contorno del dominio di $r$ è $0$ e $+\infty$, e non c'è alcun motivo di supporre che le funzioni d'onda siano nulle nell'origine; oltretutto, la misura in coordinate polari non è il semplice prodotto dei $\dd x_i$ ma contiene altri fattori.
Con queste definizioni abbiamo dunque
\begin{equation}
	(\scalar{\op{\vec x}}{\op{\vec p}})^2=-\hbar^2r\drp{}{r}r\drp{}{r}=-\hbar^2\bigg(r\drp{}{r}+r^2\ddrp{}{r}\bigg)=\op r^2\op p_r^2-i\hbar \op r\op p_r
\end{equation}
e il quadrato dell'impulso si scrive finalmente come
\begin{equation}
	\op{\vec p}^2=-\hbar^2\ddrp{}{r}-\hbar^2(d-1)\frac1{r}\drp{}{r}+\frac{L^2}{r^2}=
	-\hbar^2\bigg[\ddrp{}{r}+(d-1)\frac1{r}\drp{}{r}\bigg]+\frac{L^2}{r^2}=
	-\hbar^2\lap
\end{equation}
dato che
\begin{equation}
	\lap=\ddrp{}{r}+(d-1)\frac1{r}\drp{}{r}-\frac{L^2}{\hbar^2r^2}=
	\frac1{r^{d-1}}\drp{}{r}r^{d-1}\drp{}{r}-\frac{L^2}{\hbar^2r^2}
	\label{eq:laplaciano-momento-angolare}
\end{equation}
è l'operatore laplaciano, nella nostra rappresentazione in coordinate sferiche in dimensione $d$.

Notiamo che nelle equazioni agli autovalori \eqref{eq:autofunzioni-momento-angolare} per le funzioni $\psi_{E,l,m}$ non sono coinvolti il potenziale e l'energia del sistema, ma solo i numeri $l$ e $m$.
Per cercare le autofunzioni scegliamo innanzitutto un potenziale nullo, ottenendo l'equazione $-\hbar^2\lap\psi=2mE\psi$.
Ponendo $k^2\defeq\frac{2mE}{\hbar^2}$ possiamo semplificarla nella forma $\lap\psi=-k^2\psi$.
Effettuiamo dunque una separazione delle variabili fattorizzando la funzione d'onda come $R(r)Y(\phi,\theta)$: nella prima delle \eqref{eq:autofunzioni-momento-angolare} troviamo cos\`i $L^2R(r)Y(\phi,\theta)=\hbar^2l(l+1)R(r)Y(\phi,\theta)$ in cui semplifichiamo $R(r)$ trovando
\begin{equation}
	L^2Y(\phi,\theta)=\hbar^2l(l+1)Y(\phi,\theta)
\end{equation}
e con un ragionamento analogo, detto $g(l)$ l'autovalore di $L^2$ in dimensione generica $d$, l'equazione $\lap\psi=-k^2\psi$ si riscrive con la \eqref{eq:laplaciano-momento-angolare} come
\begin{equation}
	Y(\phi,\theta)\bigg[\frac1{r^{d-1}}\drp{}{r}r^{d-1}\drp{}{r}+\frac{g(l)}{r^2}\bigg]R(r)=-k^2R(r)Y(\phi,\theta)
\end{equation}
e possiamo semplificare il fattore $Y(\phi,\theta)$ dai due membri.
Ora possiamo porre anche $E=0$, ottenendo per la funzione d'onda l'equazione di Laplace $\lap\psi=0$: essa è tale che se $\psi(\vec x)$ è soluzione, lo è anche $\psi(\lambda\vec x)$ per ogni $\lambda\in\R$, ossia le soluzioni sono funzioni omogenee.
Cercheremo le soluzioni a questa equazione, quindi, nell'insieme dei polinomi omogenei.
