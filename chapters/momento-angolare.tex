\chapter{Momento angolare}
L'operatore associato al momento angolare è, come in meccanica classica, il generatore delle rotazioni, che soddisfa le relazioni di commutazione
\begin{equation}
	[\op L_i,\op L_j]=i\hbar\epsilon_{ijk}\op L_k.
	\label{eq:commutazione-momento-angolare}
\end{equation}
A meno di fattori moltiplicativi, le costanti di struttura (la parte rilevante è il simbolo di Levi-Civita $\epsilon_{ijk}$) sono le stesse dell'algebra di Lie $\mathfrak{su}(2)$.
Non a caso, $\mathfrak{su}(2)$ è isomorfa a $\mathfrak{so}(3)$ che è l'algebra del gruppo $SO(3)$ delle rotazioni tridimensionali; in altre parole $\mathfrak{su}(2)$ genera $SO(3)$, allo stesso modo in cui il momento angolare genera le rotazioni in $\R^3$, e ciò giustifica la \eqref{eq:commutazione-momento-angolare}.
Una quantità che commuta con il momento angolare lungo una data direzione è dunque simmetrica per le rotazioni attorno a tale asse: gli scalari sono degli importanti esempi di queste quantità.
D'ora in poi, elimineremo sistematicamente la costante $\hbar$ dalle equazioni (a meno di trovarci in casi ambigui che la richiedano esplicitamente), sottindendendo che il momento angolare è misurato \emph{in unità di $\hbar$}.
In pratica, dividiamo la \eqref{eq:commutazione-momento-angolare} per $\hbar^2$ ottenendo la forma più semplice $[\op L_i,\op L_j]=i\epsilon_{ijk}\op L_k$.
Insieme alle tre componenti cartesiane del momento angolare abbiamo anche il quadrato del suo modulo, ossia $\op L^2=\op L_i\op L_i$: essendo uno scalare, ci aspettiamo che sia invariante per rotazioni ossia che commuti con tutte le componenti del momento angolare.
Una verifica diretta mostra infatti che, per $i=1,2,3$,
\begin{equation}
	[\op L^2,\op L_i]=[\op L_k\op L_k,\op L_i]=\op L_k[\op L_k,\op L_i]+[\op L_k,\op L_i]\op L_k=i\epsilon_{kil}(\op L_k\op L_l+\op L_l\op L_k)=i(\epsilon_{kil}+\epsilon_{lik})\op L_k\op L_l=0
\end{equation}
per l'antisimmetria di $\epsilon_{kil}$.
In ogni caso le tre componenti $\op L_i$ non sono tra loro compatibili (dalla \eqref{eq:commutazione-momento-angolare}) a meno del caso banale in cui siano tutte nulle.
Nella costruzione di un sistema completo di osservabili compatibili, è convenzione scegliere insieme a $\op L^2$ anche la componente $z$, ossia $\op L_3$, da diagonalizzare simultaneamente.

\section{Autovalori del momento angolare}
Supponiamo di aver trovato un sistema completo di osservabili compatibili che contenga $\op L^2$ e $\op L^3$.
Chiamiamo $m$ gli autovalori di $\op L_3$ e $\mu^2$ quelli di $\op L^2$, e trascurando lo altre osservabili del sistema indichiamo gli autostati simultanei con $\ket{m,\mu^2}$, che sono quindi tali per cui\footnote{Gli autovalori di $\op L^2$ sono evidentemente positivi, perciò li indichiamo direttamente con il quadrato $\mu^2$.}
\begin{equation}
	\op L^2\ket{m,\mu^2}=\mu^2\ket{m,\mu^2}\qeq\op L_3\ket{m,\mu^2}=m\ket{m,\mu^2}.
\end{equation}
Come già per l'oscillatore armonico, introduciamo gli operatori a scala $\op L_+\defeq\op L_1+i\op L_2$ e $\op L_-\defeq\op L_1-i\op L_2$.
Dato che gli $\op L_i$ sono hermitiani (poich\'e rappresentano delle osservabili) si vede subito che $\adj{\op L_+}=\op L_-$.
Essi sono in relazione con la componente scelta del momento angolare con i commutatori
\begin{equation}
	[\op L_3,\op L_+]=[\op L_3,\op L_1]+i[\op L_3,\op L_2]=i\op L_2-i^2\op L_1=\op L_+
\end{equation}
da cui
\begin{equation}
	[\op L_3,\op L_-]=[\op L_3,\adj{\op L_+}]=-\adj{[\op L_3,\op L_+]}=-\adj{\op L_+}=-\op L_-.
\end{equation}
Vediamo dunque come questi due operatori agiscono sugli autostati:
\begin{multline}
	\op L_3\op L_+\ket{m,\mu^2}=(\op L_3\op L_+ - \op L_+\op L_3 + \op L_+\op L_3)\ket{m,\mu^2}=[\op L_3,\op L_+]\ket{m,\mu^2}+\op L_+\op L_3\ket{m,\mu^2}=\\=
	\op L_+\ket{m,\mu^2}+m\op L_+\ket{m,\mu^2}=(m+1)\op L_+\ket{m,\mu^2}
\end{multline}
e analogamente $\op L_3\op L_-\ket{m,\mu^2}=(m-1)\op L_-\ket{m,\mu^2}$.
Gli operatori $\op L_+$ e $\op L_-$ dunque portano un autostato di $\op L_3$ in un altro autostato con autovalore, rispettivamente, aumentato o diminuito di 1.
In altri termini, possiamo scriverlo come $\op L_+\ket{m,\mu^2}=\ket{m+1,\mu^2}$ e $\op L_-\ket{m,\mu^2}=\ket{m-1,\mu^2}$.

Dobbiamo quindi chiederci se la sequenza $\{\dots,m-2,m-1,m,m+1,\dots\}$ sia limitata o meno.
Dato che $\op L_3^2=\op L^2-\op L_1^2+\op L_2^2$, sicuramente per ciascuno stato il valore di aspettazione di $\op L_3^2$ deve essere minore di quello di $\op L^2$, dato che ovviamente $\op L_1^2+\op L_2^2$ è un operatore definito positivo.
In particolare, dunque, risulta $m^2\le\mu^2$, ossia $\abs{m}\le\abs{\mu}$: gli autovalori sono allora necessariamente limitati; esiste un certo stato che chiamiamo $\ket{j_+,\mu^2}$ tale per cui $\op L_+\ket{j_+,\mu^2}=0$ e analogamente uno stato $\ket{j_-,\mu^2}$ per cui $\op L_-\ket{j_-,\mu^2}=0$, dove $j_+$ e $j_-$ sono il massimo e il minimo autovalore, rispettivamente, di $\op L_3$.
Per questi due stati si ha comunque $\op L^2\ket{j_\pm,\mu^2}=\mu^2\ket{j_\pm,\mu^2}$.
Calcolando la norma di $\op L_+\ket{j_+,\mu^2}$ troviamo
\begin{equation}
	\begin{split}
		0&=\bra{j_+,\mu^2}\adj{\op L_+}\op L_+\ket{j_+,\mu^2}=\\
		&=\bra{j_+,\mu^2}\op L_-\op L_+\ket{j_+,\mu^2}=\\
		&=\bra{j_+,\mu^2}(\op L_1-i\op L_2)(\op L_1+i\op L_2)\ket{j_+,\mu^2}=\\
		&=\bra{j_+,\mu^2}(\op L_1^2+\op L_2^2+i[\op L_1,\op L_2])\ket{j_+,\mu^2}=\\
		&=\bra{j_+,\mu^2}(\op L^2-\op L_3^2-\op L_3])\ket{j_+,\mu^2}=\\
		&=(\mu^2-j_+^2-j_+)\braket{j_+,\mu^2}{j_+,\mu^2}
	\end{split}
\end{equation}
da cui otteniamo $\mu^2=j_+(j_++1)$.
Dalla norma di $\op L_-\ket{j_-,\mu^2}$, anch'essa nulla, troviamo invece $\mu^2=j_-(j_--1)$.
Assegnato un valore a $j_-$, uguagliando le due espressioni abbiamo $j_j(j_++1)=j_-(j_--1)$ da cui $j_+=-j_-$ oppure $j_+=j_--1$.
La seconda delle due però non è accettabile dato che per costruzione $j_+\ge j_-$, perciò rimane $j_+=-j_-$.
Chiamiamo questo valore semplicemente con $j$.
Abbiamo quindi ottenuto lo spettro di $\op L_3$ che è l'insieme $\{-j,-j+1,\dots,j-1,j\}$, che ha $2j+1$ elementi.
Questa cardinalità è chiaramente un numero intero, perciò $j$ deve essere a sua volta intero oppure semiintero (positivo o nullo).
Una volta che $\op L_3$ assume uno di questi valori, poi, $\op L^2$ ha come autovalore $j(j+1)$: tale numero non è un quadrato perfetto, per il fatto che anche quando $j=m$ (il massimo autovalore di $\op L_3$) non si ha comunque $\mu^2=m^2$, perch\'e da questo seguirebbe che $\op L_1$ e $\op L_2$ avrebbero solo autovalori nulli, ossia il momento angolare lungo i due assi restanti sarebbe nullo; sappiamo che questo non è possibile, perch\'e non possiamo determinare con precisione assoluta contemporaneamente due componenti del momento angolare.
L'unica eccezione a questo si ha nel caso banale in cui $j=0$, in cui $\op{\vec L}=0$.

In questa rappresentazione del momento angolare, inoltre, $\op L^2$ è un multiplo dell'identità (detto anche \emph{operatore di Casimir}): è degenere dato che ad ogni suo autostato corrispondono $2j+1$ autostati linearmente indipendenti (di $\op L_3$), come si poteva anche capire dal teorema \ref{t:degenerazione} oppure dal lemma di Schur.
La rappresentazione è inoltre determinata completamente determinata dal numero $j$.\footnote{Ogni rappresentazione del gruppo delle rotazioni è univocamente determinata da un numero intero o semiintero positivo o nullo. Tutte le rappresentazioni irriducibili si ottengono in questo modo, mentre quelle riducibili si ricavano come somma diretta o prodotto di queste.}
Ad esempio la rappresentazione con $j=\frac12$ è data dalle matrici di Pauli: si ha infatti $\sigma_3=\frac12\begin{psmallmatrix}1&0\\0&-1\end{psmallmatrix}$ e $\frac14(\sigma_1^2+\sigma_2^2+\sigma_3^2)=\op L^2=\frac34=j(j+1)$.

