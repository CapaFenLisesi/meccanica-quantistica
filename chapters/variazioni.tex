\chapter{Metodi variazionali e di approssimazione}
\section{Il limite classico}
Nella meccanica classica la posizione e l'impulso sono grandezze precise e ben determinate, cosa che sappiamo non accade mai in ambito quantistico.
In alcuni casi potremmi riportarci al primo caso, ad esempio prendendo $\avg{q}$ e $\avg{p}$ alla stregua di variabili classiche se $\Delta q$ e $\Delta p$ sono trascurabili, in quanto i risultati veri mostrerebbero delle fluttuazioni attorno ai valori medi insignificanti.

Ammesso di trovare una situazione ``ideale'', essa può conservarsi nel tempo?
Traduciamo l'equazione di Newton usando i valori medi in $m\ddot{\avg{q}}=\avg{F(q)}$: in linea di principio però $F(\avg{q})\ne\avg{F(q)}$, cosa che accade solo se $F$ è lineare in $q$.
Sviluppando l'espressione della forza centrando in $\avg q$ troviamo $F(q)=F(\avg{q})+(q-\avg{q})F'(\avg{q})+o(q-\avg{q})$ che assume una forma \emph{approssimativamente lineare} quando il resto $o(q-\avg{q})$ è trascurabile.
Tale resto contiene come primo termine, oltretutto, lo scarto quadratico $(q-\avg{q})^2$, che è la quantità da minimizzare per poter compiere l'approssimazione con i valori medi.

Prendiamo il sistema di una particella libera: classicamente abbiamo le equazioni del moto $\dot{p}(t)=0$ e $\dot{q}(t)=\frac{p(t)}m$.
Chiamiamo direttamente $p(t)=p$, che è costante, ottenendo $q(t)=q(0)+\frac{p}{m}t$.
Risulta dunque
\begin{multline}
	\delta q(t)^2=\avg{q^2(t)}-\avg{q(t)}^2=\avg{q^2}+\frac{\avg{p^2}}{m^2}t^2+\frac{t}{m}(\avg{pq}+\avg{qp})-\avg{q}^2-\frac{\avg{p}^2}{m^2}t^2-\frac2{m}\avg{p}\avg{q}t=\\
	=\Delta q(0)^2+\frac{t}{m}(\avg{pq}+\avg{qp}-2\avg{p}\avg{q})+\frac{\Delta p(0)^2}{m^2}t^2
\end{multline}
che è l'equazione di una parabola.
Inesorabilmente quindi avremo $\Delta q(t)\sim\frac{\Delta p}{m}t$ per tempi molto grandi; possiamo concludere che $\Delta q(t)\sim 0$ se la massa della particella è molto grande.

\section{Approssimazione WKB}
Mostriamo il metodo di approssimazione dovuto a Wentzel, Kramers e Brillouin, detto anche \emph{approssimazione semiclassica}.
Consideriamo un sistema di una particella, in una dimensione: se il potenziale è costante, la soluzione dell'equazione di Schr\"odinger è una sovrapposizione di onde piane date da
\begin{equation}
	\psi(x)=\psi(0)e^{\pm ipx/\hbar}
\end{equation}
dove $p=\sqrt{2m(E-V)}$.
Supponiamo ora che $V(x)$, seppur non costante, vari ``lentamente'' in una regione di spazio: qui la funzione d'onda potrebbe essere approssimata proprio come un'onda piana.
Quanto lentamente deve variare il potenziale?
Per trovare la corretta approssimazione pensiamo al fatto che $\hbar$ sia una quantità piccola, trascurabile.
Scriviamo la funzione d'onda nella forma
\begin{equation}
	\psi(x)=e^{i\phi(x)/\hbar}
	\label{eq:WKB-funzione-prova}
\end{equation}
dove $\phi$ è una generica funzione complessa, che sviluppiamo in una serie di potenze di $\hbar$: ogni termine contiene una potenza $\hbar^k$ che moltiplica una certa funzione $\phi_k$, che possiamo vedere come una correzione sempre più fine, all'aumentare di $k$, alla vera funzione d'onda.
Ai nostri fini ci interessano i primi termini: $\phi=\phi_0+\phi_1\hbar+o(\hbar)$.
Prima di tutto calcoliamo le derivate della \eqref{eq:WKB-funzione-prova}: $\psi'=\frac{i}{\hbar}\phi'\psi$ e $\psi''=\frac{i}{\hbar}\phi''\psi-\frac1{\hbar^2}(\phi')^2\psi$.
Riscriviamo l'equazione di Schr\"odinger indipendente dal tempo come
\begin{equation}
	-\frac{\hbar^2}{2m}\psi''+\big[V-E\big]\psi=0\qqq \psi''+\frac{p^2}{\hbar^2}\psi=0
\end{equation}
dove $[p(x)]^2=\frac{2m}{\hbar^2}[E-V(x)]$ e inserendo la \eqref{eq:WKB-funzione-prova} otteniamo
\begin{equation}
	\frac{i}{\hbar}\phi''\psi-\frac1{\hbar^2}(\phi')^2\psi+\frac{p^2}{\hbar^2}\psi=0
\end{equation}
in cui possiamo semplificare $\psi$ in quanto non è mai nulla (è un'esponenziale).
Sostituiamo dunque $\phi$ con il suo sviluppo in serie di potenze di $\hbar$ fino al primo ordine: risulta
\begin{equation}
	\begin{gathered}
		\frac{i}{\hbar}\big[\phi_0''+\phi_1''\hbar+o(\hbar)\big]-\frac1{\hbar^2}\big[\phi_0'+\phi_1'+o(\hbar)\big]^2+\frac{p^2}{\hbar^2}=0\\
		\frac{i}{\hbar}\phi_0''+i\phi_1''+o(1)-\frac1{\hbar^2}\big[(\phi_0')^2+(\phi_1')^2\hbar^2+2\phi_0'\phi_1'\hbar+o(\hbar)\big]+\frac{p^2}{\hbar^2}=0\\
		\frac{i}{\hbar}\phi_0''+i\phi_1''+o(1)-\frac{(\phi_0')^2}{\hbar^2}-(\phi_1')^2-\frac{2\phi_0'\phi_1'}{\hbar}+o\Big(\frac1{\hbar}\Big)=0.
	\end{gathered}
\end{equation}
Moltiplicando tutto per $\hbar^2$ abbiamo
\begin{equation}
	i\hbar\phi_0''+i\hbar^2\phi_1''+o(\hbar^2)-(\phi_0')^2-\hbar^2(\phi_1')^2-2\hbar\phi_0'\phi_1'+p^2+o(\hbar)=0.
\end{equation}
Il termine $o(\hbar^2)$ e tutti gli altri che contengono $\hbar^2$ sono dunque ``assorbiti'' in $o(\hbar)$: abbiamo infine
\begin{equation}
	p^2-(\phi_0')^2+(i\phi_o''-2\phi_0'\phi_1')\hbar+o(\hbar)=0.
	\label{eq:WKB-schroedinger-approssimata}
\end{equation}

Possiamo effettuare dunque un'approssimazione all'ordine zero, cioè considerando solo i termini con la potenza $\hbar^0$ e trascurando i restanti: troviamo $\phi_0'=\abs{p}$, dunque la soluzione dell'equazione di Schr\"odinger con questa approssimazione è
\begin{equation}
	\psi(x)=Ae^{\pm\frac{i}{\hbar}\int p(x')\,\dd x'},
	\label{eq:soluzione-WKB-ordine-0}
\end{equation}
nel senso che una soluzione generale è combinazione lineare delle due onde (una per ciascun segno dell'esponente).
L'integrale è lasciato indefinito in quanto un'eventuale costante arbitraria è assorbita in $A$:
\begin{equation}
	\psi(x)=Ae^{\pm\frac{i}{\hbar}\int p(x')\,\dd x'+c}=Ae^ce^{\pm\frac{i}{\hbar}\int p(x')\,\dd x'}=\tilde{A}e^{\pm\frac{i}{\hbar}\int p(x')\,\dd x'}.
\end{equation}
A questo punto possiamo anche scegliere l'estremo inferiore di integrazione come un generico $x_0$ (nel dominio adatto), trovando
\begin{equation}
	\psi(x)=\psi(0)e^{\pm\frac{i}{\hbar}\int_{x_0}^x p(x')\,\dd x'}.
\end{equation}

Compiamo il passo successivo e consideriamo anche il termine del primo ordine: la funzione d'onda sarà dunque $\psi(x)=A\exp\big(\frac{i}{\hbar}\phi_0(x)+i\phi_1(x)\big)$.
Uguagliando a zero i due termini otteniamo per $\phi_0$ la stessa soluzione trovata poco fa, mentre per il termine in $\hbar$ abbiamo
\begin{equation}
	i\phi_0''=2\phi_0'\phi_1'\qqq \frac{\phi_0''}{\phi_0'}=-2i\phi_1'\qqq \log\phi_0'=-2i\phi_1+k
\end{equation}
per un $k\in\C$ generico.
Risolvendo per $\phi_1$ troviamo, ricordando che $\phi_0'=p$,
\begin{equation}
	\phi_1=\frac{i}2\log\phi_0'+c=i\log\sqrt{\abs{p}}+k'
\end{equation}
perciò la funzione d'onda approssimata è
\begin{equation}
	\psi(x)=Be^{\pm\frac{i}{\hbar}\int^xp(x')\,\dd x'+ik'-\log\sqrt{\abs{p}}}=\frac{B}{\sqrt{p}}e^{\pm\frac{i}{\hbar}\int^xp(x')\,\dd x'},
	\label{eq:soluzione-WKB-ordine-1}
\end{equation}
intendendo anche qui che la soluzione generale è combinazione lineare delle due funzioni, per ciascun segno.

È necessario infine discutere della validità di questa approssimazione: essa è valida se il termine in $\hbar$ è trascurabile, ossia se
\begin{equation}
	\hbar\abs{\drv{}{x}\frac1{p(x)}}\ll 1.
	\label{eq:condizione-approssimazione-WKB}
\end{equation}

