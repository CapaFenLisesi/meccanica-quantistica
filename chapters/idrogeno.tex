\chapter{L'atomo di idrogeno}
In questo capitolo affrontiamo finalmente uno dei problemi ``classici'' completamente risolubili della meccanica quantistica, lo studio dello spettro dell'atomo di idrogeno.
Cominciamo dal problema base del problema a due corpi.

\section{Il problema a due corpi}
Affrontiamo per ora il problema nella visione della meccanica classica.
Prendiamo un sistema composto da due corpi, a cui assegnamo le coordinate canoniche $(\vec x_1,\vec p_1)$ e $(\vec x_2,\vec p_2)$, legati da un potenziale di interazione $V(\norm{\vec x_1-\vec x_2})$.
Il sistema è descritto dalla funzione hamiltoniana
\begin{equation}
	\mathcal H=\frac{\vec p_1^2}{2m_1}+\frac{\vec p_2^2}{2m_2}+V(\norm{\vec x_1-\vec x_2})
	\label{eq:hamiltoniana-due-corpi}
\end{equation}
ma è più conveniente passare con una trasformazione canonica al sistema del centro di massa, studiando cos\`i il moto del centro di massa e della distanza tra i due corpi
\begin{equation}
	\vec X\defeq\frac{m_1\vec x_1+m_2\vec x_2}{m_1+m_2}\qeq\vec x\defeq\vec x_1-\vec x_2.
	\label{eq:coordinate-moto-relativo}
\end{equation}
Gli impulsi coniugati risultano essere
\begin{equation}
	\vec P=\vec p_1+\vec p_2\qeq\vec p=\frac{m_2\vec p_1-m_1\vec p_2}{m_1+m_2}
	\label{eq:impulsi-moto-relativo}
\end{equation}
per cui introducendo la massa totale $M\defeq m_1+m_2$ e la massa ridotta $\mu\defeq\frac{m_1m_2}{m_1+m_2}$ abbiamo l'hamiltoniana
\begin{equation}
	\mathcal H=\frac{\vec P^2}{2M}+\frac{\vec p^2}{2\mu}+V(\norm{\vec x})
	\label{eq:hamiltoniana-moto-relativo}
\end{equation}
evidentemente scomponibile come somma di due hamiltoniane distinte, una per il centro di massa (nelle coordinate $\vec X,\vec P$) e una del moto relativo dei due corpi (nelle coordinate $\vec x,\vec p$), che commutano.
La prima delle due tra l'altro è l'hamiltoniana di una particella libera: essendo $\vec X$ ciclica l'impulso totale $\vec P$ si conserva.
Possiamo dunque studiare il problema separatamente, e d'ora in poi ci interessiamo soltanto al moto relativo.
Per quanto riguarda la massa ridotta, se uno dei due corpi ha massa molto minore possiamo trascurare quest'ultima: è il caso (che studieremo) del sistema protone-elettrone, in cui $m_e\ll m_p$, e potremo approssimare $\mu\approx m_p$.
Inoltre il potenziale $V(\norm{\vec x})$ è centrale, quindi il momento angolare si conserva e $\{H,\vec L\}=0$: possiamo quindi decomporre il termine $\vec p^2$ come $\vec p^2=p_r^2+\frac{L^2}{r^2}$ ottenendo l'hamiltoniana
\begin{equation}
	\mathcal H=\frac{p_r^2}{2\mu}+\frac{L^2}{2\mu r^2}+V(r)
	\label{eq:hamiltoniana-radiale}
\end{equation}
in cui entrambe le variabili angolari sono cicliche.
Possiamo ulteriormente semplificare la funzione accorpando il termine $\frac{L^2}{2\mu r^2}$, che dipende solo dalla posizione, nel potenziale, formando cos\`i il \emph{potenziale efficace} $V_\textup{eff}(r)$.
Ci siamo ridotti quindi a un problema unidimensionale nella variabile $r$, con l'unico accorgimento che essa non varia in $\R$ ma in $[0,+\infty)$.

Applicando quando ricavato al caso quantistico, in dimensione $d=3$ l'operatore $\op{\vec p}^2$ è scritto in rappresentazione di Schr\"odinger delle coordinate, secondo la \eqref{eq:laplaciano-momento-angolare}, come
\begin{equation}
	-\hbar^2\frac1{r^2}\drp{}{r}r^2\drp{}{r}=-\frac{\hbar^2}{r^2}\bigg(2r\drp{}{r}+r^2\ddrp{}{r}\bigg)=-\hbar^2\bigg(\frac2{r}\drp{}{r}+\ddrp{}{r}\bigg).
\end{equation}
Con l'uguaglianza $p_r^2=\frac1{r}p_r^2r$ possiamo infine riscrivere l'equazione di Schr\"odinger in questa rappresentazione come
\begin{equation}
	-\frac{\hbar^2}{2\mu}\frac1{r}\ddrp{}{r}\big(r\psi(r,\phi,\theta)\big)+\frac{\hbar^2l(l+1)}{r^2}\psi(r,\phi,\theta)+V(r)\psi(r,\phi\theta)=E\psi(r,\phi,\theta).
	\label{eq:schrodinger-moto-relativo}
\end{equation}
La ciclicità delle variabili angolari ci permette di fattorizzare la funzione d'onda in una parte radiale $R(r)$ e una angolare $Y(\phi,\theta)$.
Sappiamo dal capitolo precedente che la parte radiale $R$ soddisfa un'equazione dipendente dall'energia $E$ del sistema e dal numero quantico $l$ associato all'autovalore di $\op L^2$, mentre la parte angolare $Y$ soddisfa due equazioni dipendenti dall'autovalore $m$ di $\op L_3$ e da $l$.
Identifichiamo dunque le soluzioni accettabili della \eqref{eq:schrodinger-moto-relativo} con questi tre numeri, scrivendo
\begin{equation}
	\psi(r,\phi,\theta)=\psi_{E,l,m}(r,\phi,\theta)=R_{E,l}(r)Y_{l,m}(\phi,\theta).
	\label{eq:fattorizzazione-wf-radiale-angolare}
\end{equation}
Nell'equazione \eqref{eq:schrodinger-moto-relativo} possiamo quindi semplificare il fattore $Y_{l,m}$ ottenendo un'equazione solo per $R$.
Notiamo inoltre che la derivata seconda in $r$ agisce non su $R(r)$ soltanto, ma su $rR(r)$: moltiplicando i due membri a sinistra per $r$ (che commuta con i restanti termini) abbiamo dunque un'equazione differenziale per $rR(r)$, ossia
\begin{equation}
	-\frac{\hbar^2}{2\mu}\ddrp{}{r}\big(rR(r))+\frac{\hbar^2l(l+1)}{r^2}rR(r)+V(r)rR(r)=ErR(r).
\end{equation}
Appare naturale a questo punto definire $u(r)\defeq rR(r)$, ottenendo l'equazione
\begin{equation}
	-\frac{\hbar^2}{2\mu}u''(r)+V_\textup{eff}(r,l)u(r)=Eu(r),\hspace{1cm}\text{con }V_\textup{eff}(r,l)=\frac{\hbar^2l(l+1)}{r^2}+V(r).
	\label{eq:schrodinger-moto-relativo-potenziale-efficace}
\end{equation}
Questa definizione non semplifica solo l'equazione di Schr\"odinger, ma anche la condizione di normalizzazione: supponendo di aver normalizzato la parte angolare con
\begin{equation}
	\int_0^{2\pi}\int_0^\pi\abs{Y_{l,m}(\phi,\theta)}^2\sin\theta\,\dd\phi\,\dd\theta=1,
	\label{eq:normalizzazione-parte-angolare}
\end{equation}
la normalizzazione per la funzione d'onda diventa in questo modo
\begin{multline}
	1=\int_{\R^3}\abs{\psi(\vec x)}^2\,\dd^3x=\int_0^{+\infty}\int_0^{2\pi}\int_0^\pi\abs{R_{E,l}(r)Y_{l,m}(\phi,\theta)}^2r^2\sin\theta\,\dd\phi\,\dd\theta\,\dd r=\\
	=\int_0^{+\infty}r^2\abs{R_{E,l}(r)}^2\,\dd r=\int_0^{+\infty}\abs{u_{E,l}(r)}^2\,\dd r
	\label{eq:normalizzazione-parte-radiale}
\end{multline}
che tanto somiglia al caso unidimensionale.
Non c'è alcuna ragione per cui $R$ debba essere nulla per $r=0$, ma perlomeno deve essere continua e limitata quindi necessariamente si deve avere
\begin{equation}
	0=\lim_{r\to 0^+}rR(r)=\lim_{r\to 0^+}u(r)
\end{equation}
che è un'utile condizione al contorno per la funzione $u$.

\section{L'atomo di idrogeno}
Per l'atomo di idrogeno abbiamo l'energia potenziale elettrostatica $V(r)=-\frac{e^2}{r}$, posseduta da un elettrone in orbita attorno ad un protone.
Più in generale, possiamo considerare il moto dell'elettrone in un campo elettrostatico centrale, generato da una carica puntiforme $Ze$ nell'origine, ossia con il potenziale
\begin{equation}
	V(r)=-\frac{Ze^2}{r}.
	\label{eq:potenziale-elettrostatico}
\end{equation}
Il sistema è dunque caratterizzato (classicamente) dalla funzione hamiltoniana
\begin{equation}
	\mathcal H=\frac{\vec p^2}{2\mu}-\frac{Ze^2}{r}
	\label{eq:hamiltoniana-idrogeno}
\end{equation}
e il corrispondente operatore si scrive, nella rappresentazione di Schr\"odinger della posizione (in coordinate sferiche), diagonalizzando contemporaneamente $\op L^2$ e $\op L_3$ come
\begin{equation}
	-\frac{\hbar^2}{2\mu}\frac1{r}\ddrp{}{r}r+\frac{\hbar^2l(l+1)}{2\mu r^2}-\frac{Ze^2}{r}.
	\label{eq:hamiltoniano-idrogeno-schroedinger}
\end{equation}
Abbiamo già visto che possiamo approssimare la massa ridotta con la massa del protone, o in generale della carica puntiforme al centro dell'orbita, per cui d'ora in poi scriveremo $m$ al posto di $\mu$ intendendo questa approssimazione.
Vale quanto detto nella sezione precedente per la fattorizzazione della funzione d'onda, l'equazione di Schr\"odinger per $u(r)=rR(r)$ e le condizioni di normalizzazione.

Il potenziale efficace, dato dall'equazione \eqref{eq:schrodinger-moto-relativo-potenziale-efficace}, tende a zero per $r\to+\infty$, a $+\infty$ per $r\to 0^+$ e ammette un minimo assoluto negativo.
Se dunque $E<0$ l'hamiltoniano ammetterà uno spettro discreto, e si trovano degli stati legati; se invece $E>0$ avrà uno spettro continuo, ma non degenere, con stati non legati.
Esaminiamo il comportamento asintotico delle soluzioni: per $r\to+\infty$ il potenziale diventa trascurabile rispetto a qualunque valore di $E$ che non sia zero, perciò approssimiamo l'equazione di Schr\"odinger come
\begin{equation}
	u''(r)=\frac{2m\abs{E}}{\hbar^2}u(r)=\beta^2u(r)
	\label{eq:schroedinger-idrogeno-approx-inf}
\end{equation}
in cui $\beta\defeq\frac{\sqrt{2m\abs{E}}}{\hbar}$.
La soluzione generale dell'equazione è $u(r)=c_1e^{-\beta r}+c_2e^{\beta r}$ in cui chiaramente dobbiamo porre $c_2=0$.
Per $r\to 0^+$, invece, sono i termini $E+\frac{Ze^2}{r}$ ad essere trascurabili rispetto al termine proporzionale a $1/r^2$, quindi approssimiamo l'equazione come
\begin{equation}
	-\frac{\hbar^2}{2m}u''(r)+\frac{\hbar^2l(l+1)}{2mr^2}u(r)=0\qqq u''(r)-\frac{l(l+1)}{r^2}u(r)=0.
	\label{eq:schroedinger-idrogeno-approx-0}
\end{equation}
Ipotizziamo in questo caso che $u(r)\sim r^s$: otteniamo $u''(r)=s(s-1)r^{s-2}$ da cui l'equazione
\begin{equation}
	s(s-1)r^{s-2}-\frac{l(l+1)}{r^2}r^s=0\qqq s(s-1)=l(l+1)
\end{equation}
da cui $s=-l$ oppure $s=l+1$.
Escludendo $s=-l$ poich\'e deve risultare $u(0)=0$ (e $l$ non è negativo), otteniamo che asintoticamente per $r\to 0^+$ si ha $u(r)\sim r^{l+1}$, ossia $R(r)\sim r^l$: la probabilità di misurare la posizione (della particella in orbita) vicino al nucleo è tanto più piccola maggiore è il momento angolare orbitale del sistema.

Sapendo che la funzione $u(r)$ decade all'infinito come $e^{-\beta r}$, proviamo a risolvere l'equazione di Schr\"odinger con una funzione $u(r)=f(r)e^{-\beta r}$.
Troviamo, poich\'e $E=-\beta^2\frac{\hbar^2}{2m}$,
\begin{equation}
	\begin{gathered}
		-\frac{\hbar^2}{2m}\big[f''(r)-2\beta f'(r)+\beta^2 f(r)\big]e^{-\beta r}+\frac{\hbar^2l(l+1)}{2mr^2}f(r)e^{-\beta r}-\frac{Ze^2}{r}f(r)e^{-\beta r}+\beta^2\frac{\hbar^2}{2m}f(r)e^{-\beta r}=0\\
		-\frac{\hbar^2}{2m}\big[f''(r)-2\beta f'(r)\big]+\bigg[\frac{\hbar^2l(l+1)}{2mr^2}-\frac{Ze^2}{r}\bigg]f(r)=0\\
		f''(r)-2\beta f'(r)-\frac{l(l+1)}{r^2}f(r)+\frac{2mZe^2}{\hbar^2r}f(r)=0.
	\end{gathered}
\end{equation}
Sviluppiamo ora $f$ in serie di potenze di $r$: tenendo conto che per $r\to 0^+$ si ha $u(r)\sim r^{l+1}$, scriviamo $f(r)=r^{l+1}\sum_{k=0}^{+\infty}a_kr^k$, supponendo di conseguenza anche che $a_0\ne 0$ (possiamo fissare $a_0=1$).
Inserendo lo sviluppo nell'equazione trovata otteniamo
\begin{equation}
	\begin{gathered}
		\sum_{k=0}^{+\infty}a_k\bigg[(k+l+1)(k+l)r^{k+l-1}-2\beta(k+l+1)r^{k+l}-l(l+1)r^{k+l-1}+\frac{2mZe^2}{\hbar^2}r^{k+l}\bigg]=0\\
		\sum_{k=0}^{+\infty}a_k\big[(k+l+1)(k+l)-l(l+1)\big]r^{k+l-1}+\sum_{k=0}^{+\infty}a_k\bigg[\frac{2mZe^2}{\hbar^2}-2\beta(k+l+1)\bigg]r^{k+l}=0
	\end{gathered}
\end{equation}
e traslando gli indici della prima somma ($k\mapsto k+1$) troviamo
\begin{equation}
	\sum_{k=0}^{+\infty}\bigg\{a_{k+1}\big[(k+l+2)(k+l+1)-l(l+1)\big]-a_k\bigg[2\beta(k+l+1)-\frac{2mZe^2}{\hbar^2}\bigg]\bigg\}r^{k+l}=0
\end{equation}
da cui la relazione di ricorrenza
\begin{equation}
	a_{k+1}=\frac{2\beta(k+l+1)-\frac{2mZe^2}{\hbar^2}}{(k+l+2)(k+l+1)-l(l+1)}a_k,\hspace{1cm}\text{con }a_0=1.
\end{equation}
Per grandi valori di $k$ possiamo approssimarla con
\begin{equation}
	a_{k+1}\sim\frac{2\beta(k+l+1)}{(k+l+2)(k+l+1)}a_k\sim\frac{2\beta}{k}a_k
\end{equation}
e iterando fino a $k=0$ (ossia sostituendo $a_k$ per ricorrenza con $\frac{2\beta}{k-1}a_{k-1}$ e cos\`i via) otteniamo
\begin{equation}
	a_k\sim\frac{(2\beta)^k}{k!}
\end{equation}
La serie di potenze che ne risulta per $f(r)$ converge, però il limite è $e^{2\beta r}$ da cui segue $u(r)\sim e^{\beta r}$ per $r\to+\infty$, un risultato chiaramente non accettabile.
Deve esistere quindi un certo indice $\bar{k}\in\N$ tale per cui $a_{\bar{k}}=0$, da cui $a_k=0$ $\forall k\ge\bar{k}$ in modo da troncare la serie.
Con questa ipotesi la $f(r)$ è dunque un polinomio di grado $\bar{k}+l-1$, inoltre risulta
\begin{equation}
	\beta(\bar{k}+l+1)=\frac{mZe^2}{\hbar^2}\qqq\frac{\sqrt{2m\abs{E}}}{\hbar}(\bar{k}+l+1)=\frac{mZe^2}{\hbar^2}
\end{equation}
e definendo $n\defeq\bar{k}+l+1$ abbiamo
\begin{equation}
	\sqrt{2m\abs{E}}=\frac{mZe^2}{n\hbar}\qqq E=-\frac{Z^2me^4}{2n^2\hbar^2}
	\label{eq:energia-idrogeno}
\end{equation}
che mostra la quantizzazione dei livelli energetici degli stati legati (ricordando che in questo caso $E<0$).
Notiamo che i livelli non sono equidistanti, ma la separazione diminuisce più $E$ si avvicina a zero.
Il numero $n$ è chiamato \emph{numero quantico principale} e con esso (al posto di $E$) etichettiamo le funzioni d'onda, che scriveremo come $\psi_{n,l,m}$ con $n\in\N$, $l\in\N_0$ e $m\in[-l,l]\subset\Z$.
Notiamo inoltre che la grandezza $\frac{\hbar^2}{me^2}$ ha le dimensioni di una lunghezza: non a caso è detta \emph{raggio di Bohr} (indicato con $r_B$).
Con esso scriviamo la funzione d'onda dello stato $\ket{n,l,m}$
\begin{equation}
	\psi_{n,l,m}(r,\phi,\theta)=e^{-Zr/nr_B}r^lY_{l,m}(\phi,\theta)\sum_{j=0}^{\bar{k}}a_jr^j.
	\label{eq:wf-idrogeno}
\end{equation}
\begin{figure}
	\tikzsetnextfilename{autofunzioni-radiali-idrogeno}
	\centering
	\begin{tikzpicture}
		\begin{axis}[
				standard,
				height=.5\linewidth, width=.8\linewidth,
				enlargelimits,
				xlabel=$r$,
				xmin=0, xmax=8,
				ymin=-0.1, ymax=0.6,
				xtick={0,1,2,3,4,5,6,7,8},
				ytick={0.2,0.4,0.6}
			]
			\addplot[thick, samples=1000, domain=0:8] function {(x*2*exp(-x))**2}; %1,0
			\addplot[thick, samples=1000, dashed, domain=0:8] function {(x*1/(2*sqrt(2))*(2-x)*exp(-x/2))**2}; %2,0
			\addplot[thick, samples=1000, densely dotted, domain=0:8] function {(x*1/(2*sqrt(6))*x*exp(-x/2))**2}; %2,1
			%\addplot[thick, samples=1000, dotted, domain=0:5] function {2/(81*sqrt(3))*(27-18*x+2*x**2)*exp(-x/3)}; %3,0
			%\addplot[thick, samples=1000, densely dotted, domain=0:5] function {4/(81*sqrt(6))*(6-x)**x*exp(-x/3)}; %3,1
			\legend{$n=1$, $l=0$\\$n=2$, $l=0$\\$n=2$, $l=1$\\}
		\end{axis}
	\end{tikzpicture}
	\caption{Alcune delle prime funzioni di densità di probabilità radiali $r^2\abs{R_{n,l}(r)}^2$ delle funzioni d'onda dell'atomo di idrogeno, con $r$ in unità di $r_B$.}
	\label{fig:autofunzioni-radiali-idrogeno}
\end{figure}

Vediamo quindi, dal fattore $\exp\big(\frac{Z}{n}\frac{r}{r_B}\big)$, che all'aumentare di $Z$ la funzione d'onda è ``più grande'' vicino al nucleo, mentre per $n$ grande tende ad ``allontanarsi''.

Con l'aiuto della \eqref{eq:ricorsione-momenti-pdf-posizione-unidimensionale} possiamo calcolare alcuni valori medi negli autostati $\ket{n,l,m}$, che ci torneranno anche utili più avanti.
Per avere un hamiltoniano della forma $\frac1{2m}\op p_r^2+V(\op r)$ è necessario prendere come potenziale il potenziale efficace\footnote{
	Trovandoci in un autostato dell'hamiltoniano, possiamo sostituire il quadrato del momento angolare orbitale $\op{\vec L}^2$ con il suo autovalore $l(l+1)\hbar^2$.
}
\begin{equation}
	V_\textup{eff}(r)=\frac{l(l+1)\hbar^2}{2mr^2}-\frac{Ze^2}{r}
\end{equation}
per cui il secondo membro della \eqref{eq:ricorsione-momenti-pdf-posizione-unidimensionale} è
\begin{equation}
	\begin{split}
		&\frac12\avg{r^kV'}+k\avg{r^{k-1}(V-E)}=\\
		=&\frac12\bigg(-\frac{l(l+1)\hbar^2}{m}\avg{r^{k-3}}+Ze^2\avg{r^{k-2}}\bigg)+k\bigg(\frac{l(l+1)\hbar^2}{2m}\avg{r^{k-3}}-Ze^2\avg{r^{k-2}}-E\avg{r^{k-1}}\bigg)=\\
		=&\frac{l(l+1)\hbar^2}{2m}(k-1)\avg{r^{k-3}}+Ze^2\bigg(\frac12-k\bigg)\avg{r^{k-2}}-kE\avg{r^{k-1}},
	\end{split}
\end{equation}
per cui abbiamo l'equazione di ricorsione
\begin{equation}
	\frac{\hbar^2}{8m}\big[k(k-1)(k-2)-4l(l+1)(k-1)\big]\avg{r^{k-3}}+Ze^2\bigg(k-\frac12\bigg)\avg{r^{k-2}}+kE\avg{r^{k-1}}=0.
	\label{eq:ricorsione-momenti-pdf-radiale-atomo-idrogeno}
\end{equation}
\begin{itemize}
	\item Per $k=1$ troviamo $\frac{Ze^2}2\avg{r^{-1}}+E=0$, ossia $\avg{r^{-1}}=-2E/Ze^2$.\footnote{
			È importante ricordare che $\avg{1/r}\ne1/\avg{r}$, dato che $1/r$ non è una funzione lineare in $r$.
			Solitamente però sono dello stesso ordine di grandezza.
		}
		Sostituendo ad $E$ il valore ricavato nella \eqref{eq:energia-idrogeno} troviamo
		\begin{equation}
			\avg{r^{-1}}=-\frac2{Ze^2}\bigg(-\frac{mZ^2e^4}{2n^2\hbar^2}\bigg)=\frac{mZe^2}{n^2\hbar^2}=\frac1{n^2}\frac{Z}{r_B}.
		\end{equation}
		È interessante come questo risultato dipenda da solo uno dei tre numeri quantici introdotti.
	\item Per $k=2$ si trova l'equazione
		\begin{equation}
			=-\frac{\hbar^2}{2m}l(l+1)\avg{r^{-1}}+\frac32Ze^2+2E\avg{r}=-\frac{l(l+1)Ze^2}{2n^2}+\frac32Z^2-\frac{mZ^2e^4}{n^2\hbar^2}\avg{r}
		\end{equation}
		da cui si ottiene il valor medio della distanza dell'elettrone dal centro
		\begin{equation}
			\avg{r}=\frac{n^2\hbar^2}{mZ^2e^4}\bigg[\frac32Ze^2-\frac{l(l+1)Ze^2}{2n^2}\bigg]=\frac{\hbar^2}{mZe^2}\bigg(\frac32n^2-\frac{l(l+1)}2\bigg)=\frac{r_B}{2Z}\big[3n^2-l(l+1)\big]
		\end{equation}
	\item Per $k=3$ infine abbiamo
		\begin{equation}
			\begin{split}
				0&=\frac{\hbar^2}{8m}\big[6-8l(l+1)\big]+\frac52Ze^2\avg{r}+3E\avg{r^2}=\\
				&=\frac{\hbar^2}{4m}\big[3-4l(l+1)\big]+\frac52Ze^2\frac{r_B}{2Z}\big[3n^2-l(l+1)\big]-\frac{3mZ^2e^4}{2n^2\hbar^2}\avg{r^2}=\\
				&=\frac{\hbar^2}{4m}\big[3-4l(l+1)\big]+\frac{5\hbar^2}{4m}\big[3n^2-l(l+1)\big]-\frac{3mZ^2e^4}{2n^2\hbar^2}\avg{r^2}
			\end{split}
		\end{equation}
		da cui otteniamo
		\begin{equation}
			\avg{r^2}=\frac{n^2\hbar^4}{6mZ^2e^4}\big[3-4l(l+1)+15n^2-5l(l+1)\big]=\frac{n^2r_B^2}{2Z^2}\big[5n^2-3l(l+1)+1\big].
		\end{equation}
\end{itemize}
Per le potenze $r^k$ con $k$ negativo non possiamo usare questo metodo, ma possiamo affidarci al teorema \ref{t:dipendenza-hamiltoniano-parametro}: scegliamo come parametro il numero quantico $l$, ricordando che $n=\bar{k}+l+1$.
Allora
\begin{equation}
	\drp{\op H}{l}=\frac{(2l+1)\hbar^2}{2mr^2},
	\hspace{1cm}
	\drp{E}{l}=-\frac{mZ^2e^4}{2\hbar^2}\drp{}{l}\frac1{(\bar{k}+l+1)^2}=\frac{mZ^2e^4}{(\bar{k}+l+1)^3\hbar^2}=\frac{mZ^2e^4}{n^3\hbar^2}
\end{equation}
e con il teorema sopracitato abbiamo l'equazione
\begin{equation}
	\frac{mZ^2e^4}{n^3\hbar^2}=\bra{n,l,m}\frac{(2l+1)\hbar^2}{2mr^2}\ket{n,l,m}=\frac{(2l+1)\hbar^2}{2m}\avg{r^{-2}}
\end{equation}
da cui
\begin{equation}
	\avg{r^{-2}}=\frac{2m^2Z^2e^4}{n^3(2l+1)\hbar^4}=\frac{2Z^2}{n^3(2l+1)r_B^2}.
\end{equation}
Per calcolare i valori medi di $r^k$ con $k<-2$ bisogna infine controllare, prima di tutto, la funzione d'onda nell'origine, ossia che l'integrale
\begin{equation}
	\int_0^{+\infty}r^{2l}\frac1{r^k}r^2\,\dd r
\end{equation}
converga: questo accade per $k<2l+3$.

Lo stato fondamentale $\ket{1,0,0}$ (se $n=1$ allora $\bar{k}=l=0$, non potendo essere negativi) è caratterizzato dalla funzione d'onda $\psi_{1,0,0}(r,\phi,\theta)=e^{Zr/r_B}$: il raggio di Bohr fornisce una ``scala'' per le dimensioni del sistema.
Per $n=2$ abbiamo $\bar{k}+l=1$, da cui le possibili opzioni $l=0,\bar{k}=1$ oppure $l=1,\bar{k}=0$.
Lo stato con $l=0$ ha necessariamente anche $m=0$, dunque corrisponde a $\ket{2,0,0}$.
Se invece $l=1$, possiamo avere $m=\pm 1$: in totale per $l=1$ troviamo dunque tre stati indipendenti, caratterizzati dalle tre armoniche sferiche $Y_{1,-1}$, $Y_{1,0}$ e $Y_{1,1}$.
Per $n=3$ troviamo nove stati, e cos\`i via: ad ogni livello energetico corrispondono più stati indipendenti, con momento angolare differente.
Il livello $n$-esimo è quindi degenere, e ammette in totale
\begin{equation}
	\sum_{l=0}^{n-1}(2l+1)=2\sum_{l=0}^{n-1}l+\sum_{l=0}^{n-1}1=2\frac{n(n-1)}2+n=n^2
\end{equation}
stati.
Essendo il sistema a simmetria centrale, in effetti, tutte le componenti del momento angolare commutano con l'hamiltoniano, ma non tra di loro; questo, per il teorema \ref{t:degenerazione}, è indice della degenerazione dell'hamiltoniano.

\section{Una simmetria nascosta}
Per quanto visto finora, la simmetria sferica del potenziale del problema a due corpi fa s\`i che il momento angolare orbitale $\vec L$ sia una costante del moto, portando alla degenerazione dei livelli di energia, fissati i numeri quantici $n$ e $l$, in $2l+1$ stati.
Per l'atomo di idrogeno abbiamo visto però un'ulteriore degenerazione: all'$n$-esimo livello di energia esistono $n^2$ differenzi stati, al variare di $l$ tra $0$ e $n-1$, in quella che è chiamata \emph{degenerazione accidentale}.
Come vedremo in questa sezione, essa è collegata ad un'altra simmetria del sistema.

\paragraph{Il problema di Keplero}
Ciò che caratterizza l'atomo di idrogeno è il potenziale $-\alpha/r$, del più generale problema di Keplero: vediamone le caratteristiche principali.
Sappiamo che si conservano l'energia $E$ e il momento angolare orbitale $\vec L$, dunque
\begin{equation}
	E=H=\frac1{2m}p_r^2+\frac{L^2}{2mr^2}-\frac{\alpha}{r}=\frac{m}{2}\dot{r}^2+\frac{L^2}{2mr^2}-\frac{\alpha}{r}\defeq\frac{m}{2}\dot{r}^2+V_\textup{eff}(r)
	\label{eq:energia-keplero}
\end{equation}
da cui otteniamo
\begin{equation}
	\dot{r}=\sqrt{\frac2{m}\big[E-V_\textup{eff}(r)\big]}
	\label{eq:velocita-radiale-keplero}
\end{equation}
e da $L=mr^2\dot{\phi}$ otteniamo un'equazione differenziale per $\phi$ in funzione di $r$,
\begin{equation}
	\drv{\phi}{r}=\drv{\phi}{t}\drv{t}{r}=\frac{\dot{\phi}}{\dot{r}}.
\end{equation}
Integrando in $r$, dopo un'inversione si ottiene l'equazione dell'orbita
\begin{equation}
	\frac{L^2}{m\alpha^2}\frac1{r}=1+\sqrt{1+\frac{2EL^2}{m\alpha^2}}\cos\phi
	\label{eq:orbita-keplero}
\end{equation}
che è l'equazione di una conica
\begin{equation}
	r=\frac{P}{1+\epsilon\cos\phi}
	\label{eq:conica-parametrica}
\end{equation}
di parametro $P=L^2/m\alpha^2$ ed eccentricità $\epsilon=\sqrt{1+\frac{2EL^2}{m\alpha^2}}$.
Notiamo che si ha un ellisse se l'eccentricità è minore di $1$, che accade per $E<0$ in cui abbiamo degli stati legati.
In questa situazione, dalla \eqref{eq:conica-parametrica} vediamo inoltre che il raggio minimo (ossia la distanza minima dal fuoco da cui è misurato) è $r_\textup{min}=\frac{P}{1+\epsilon}$ mentre il raggio massimo è $r_\textup{max}=\frac{P}{1-\epsilon}$.
Indicando con $a$ il semiasse maggiore e con $c$ la semidistanza focale, dalla relazione $\epsilon=c/a$ che definisce l'eccentricità abbiamo anche $r_\textup{min}=a-c=a(1-\epsilon)$ e $r_\textup{max}=a+c=a(1+\epsilon)$, da cui
\begin{equation}
	a=\frac{P}{1-\epsilon^2}=\frac{\alpha}{2\abs{E}}\hspace{1cm}\text{e}\hspace{1cm}b=a\sqrt{1-\epsilon^2}=\frac{P}{\sqrt{1-\epsilon^2}}=\frac{L}{\sqrt{2m\abs{E}}}
	\label{eq:semiassi-keplero}
\end{equation}
dove $b$ è il semiasse maggiore, tale che $a^2=b^2+c^2$.

\paragraph{Il vettore di Lenz}
Una caratteristica unica del problema di Keplero è che il vettore
\begin{equation}
	\vec A=\alpha\frac{\vec x}{r}+\frac1{m}\vec L\times\vec p,
	\label{eq:vettore-runge-lenz}
\end{equation}
con $r=\norm{\vec x}$, detto \emph{vettore di Lenz}, è una costante del moto.\footnote{Noto anche come vettore di Runge-Lenz, vettore di Laplace-Runge-Lenz o solo vettore di Laplace.}\footnote{Questa definizione è una delle tante possibili. Una formulazione equivalente si ha se $\vec A$ è moltiplicato per $m$ (che è un'altra costante del moto, quindi non cambia i risultati se non per un fattore moltiplicativo) o se ne si cambia il segno.}
È un vettore che, dal centro di massa del sistema (il fuoco in cui è misurato l'angolo), punta verso il periapside.
Nel caso l'orbita sia circolare non esiste un periapside in quanto tutti i punti sono equidistanti dal centro, e in tal caso $\vec A$ è nullo.
Questo vettore ha delle proprietà interessanti: per prima cosa abbiamo
\begin{equation}
	\scalar{\vec x}{\vec A}=\alpha r+\frac1{m}\scalar{\vec x}{(\vec L\times\vec p)}=\alpha r+\frac1{m}\scalar{\vec L}{(\vec p\times\vec x)}=\alpha r-\frac{L^2}{m}.
\end{equation}
Posto $\scalar{\vec x}{\vec A}=Ar\cos\phi$, si ottiene proprio l'equazione dell'orbita \eqref{eq:conica-parametrica} nella forma
\begin{equation}
	\frac{L^2}{\alpha m}\frac1{r}=1-\frac{A}{\alpha}\cos\phi
	\label{eq:orbita-keplero-vettore-lenz}
\end{equation}
che mostra anche come l'eccentricità dell'orbita sia data da $A/\alpha$.
Infatti risulta
\begin{multline}
	A^2=\alpha^2+\frac1{m^2}\scalar{(\vec L\times\vec p)}{(\vec L\times\vec p)}+2\frac{\alpha}{rm}\scalar{\vec x}{(\vec L\times\vec p)}=\\
	=\alpha^2+\frac{L^2p^2}{m^2}-\frac{2\alpha L^2}{rm}=\alpha^2+\frac{2L^2}{m}\bigg(\frac{p^2}{2m}-\frac{\alpha}{r}\bigg)=\alpha^2+\frac{2EL^2}{m}
\end{multline}
da cui
\begin{equation}
	\frac{A}{\alpha}=\sqrt{1+\frac{2EL^2}{m\alpha^2}}
\end{equation}
che è proprio l'eccentricità dell'orbita.
La conservazione di questo vettore non corrisponde, come per il momento angolare, a una caratteristica geometrica del sistema: non c'è nell'hamiltoniana una coordinata ciclica di cui esso è il momento coniugato, e la sua conservazione deve essere calcolata esplicitamente con le parentesi di Poisson.
D'altro canto, a questa conservazione corrisponde una simmetria del sistema maggiore a quella data dal gruppo $SO(3)$ delle rotazioni generate da $\vec L$.
Questo fatto è stato sfruttato da Wolfgang Pauli per calcolare lo spettro dell'atomo di idrogeno senza usare l'equazione di Schr\"odinger, come abbiamo fatto in precedenza.

Associamo quindi a questo vettore l'operatore $\op{\vec A}$ definito da
\begin{equation}
	\op{\vec A}\defeq\frac{\alpha\op{\vec x}}{\op r}-\frac1{2m}(\op{\vec L}\times\op{\vec p}-\op{\vec p}\times\op{\vec L})
	\hspace{.5cm}\text{cioè}\hspace{.5cm}
	\op A_i\defeq \frac{\alpha \op x_i}{\op r}+\frac1{2m}\epsilon_{ijk}(L_jp_k+p_kL_j).
	\label{eq:operatore-vettore-lenz}
\end{equation}
con $\op r=\sqrt{\op x_k\op x_k}$.
È necessario ``simmetrizzare'' la definizione \eqref{eq:vettore-runge-lenz}, poich\'e $\op{\vec L}$ e $\op{\vec p}$ non sono compatibili (in generale), e si avrebbe un'ambiguità su come ordinare i due fattori.
In pratica dunque sostituiamo $\epsilon_{ijk}L_jp_k$ con $\frac12\epsilon_{ijk}(L_jp_k+p_kL_j)$ o equivalentemente, senza riferirsi alle coordinate, $\op{\vec L}\times\op{\vec p}$ con $\frac12(\op{\vec L}\times\op{\vec p}-\op{\vec p}\times\op{\vec L})$.

Essendo $\frac{\op{\vec x}}{\op r}$, $\vec{\op L}$ e $\op{\vec p}$ degli operatori vettoriali, lo sono anche $\op{\vec L}\times\op{\vec p}$ e $\op{\vec p}\times\op{\vec L}$, di conseguenza anche $\op{\vec A}$ lo è, essendo combinazione lineare di tali operatori.
Ciò significa che, sotto ad una rotazione, esso si trasforma secondo l'equazione
\begin{equation}
	[\op L_i,\op A_j]=i\hbar\epsilon_{ijk}A_k.
	\label{eq:commutazione-A-L}
\end{equation}
Completiamo il quadro delle costanti del moto con i commutatori di $\op{\vec A}$ con s\'e stesso e con l'hamiltoniano: risultano
\begin{equation}
	[\op H,\op A_i]=0\hspace{1cm}\text{e}\hspace{1cm}[\op A_i,\op A_j]=-i\hbar\epsilon_{ijk}\op L_k\frac2{m}\op H
\end{equation}
la prima coerentemente con il fatto che il vettore di Lenz è una costante del moto.
Anticipiamo inoltre il risultato $\scalar{\op{\vec A}}{\op{\vec L}}=\scalar{\op{\vec L}}{\op{\vec A}}=0$: si può vedere facilmente dal fatto che il vettore del momento angolare, per definizione, è ortogonale a $\vec x$, e analogamente $\vec L\times\vec p$ e $\vec p\times\vec L$ sono ortogonali a $\vec L$.
Riassumendo, abbiamo le relazioni di commutazione
\begin{equation}
	\begin{aligned}
		[\op H,\op L_i]&=0\\
		[\op H,\op A_i]&=0\\
		[\op L_i,\op L_j]&=i\hbar\epsilon_{ijk}\op L_k\\
		[\op L_i,\op A_j]&=i\hbar\epsilon_{ijk}\op A_k\\
		[\op A_i,\op A_j]&=-i\hbar\epsilon_{ijk}\op L_k\frac2{m}\op H.
	\end{aligned}
	\label{eq:commutazione-hamiltoniano-momento-angolare-lenz}
\end{equation}

Siamo interessati, a questo punto, a ``normalizzare'' il vettore di Lenz in modo da semplificare l'ultima di queste relazioni.
Chiaramente dividerlo per la radice quadrata dell'hamiltoniano è un processo difficile da definire, quindi semplifichiamo il problema restringendoci in un autospazio $\hilbert_E$ di $\op H$, corrispondente all'autovalore $E$.
In esso, applicare $\op H$ corrisponde semplicemente a moltiplicare per $E$, e cos\`i risolviamo la questione della radice quadrata.
In virtù delle prime due relazioni nella \eqref{eq:commutazione-hamiltoniano-momento-angolare-lenz}, se $\ket{\psi}$ è in tale autostato allora vi appartengono anche $\op A_i\ket{\psi}$ e $\op L_i\ket{\psi}$ per $i=1,2,3$, per il lemma \ref{l:autostati-compatibili}.
È lecito dunque restringere i tre operatori in questo autospazio.
Definiamo dunque l'operatore, in $\hilbert_E$,
\begin{equation}
	\op{\vec U}=\sqrt{-\frac{m}{2E}}\op{\vec A}
\end{equation}
e i sei operatori
\begin{equation}
	\op{\vec J}_1\defeq\frac{\op{\vec L}+\op{\vec U}}2\hspace{1cm}\text{e}\hspace{1cm}\op{\vec J}_2\defeq\frac{\op{\vec L}-\op{\vec U}}2.
	\label{eq:momento-angolare-SO4}
\end{equation}
Con essi abbiamo, al posto delle \eqref{eq:commutazione-hamiltoniano-momento-angolare-lenz}, le relazioni
\begin{equation}
	\begin{aligned}
		[\op J_{1i},\op J_{1j}]&=
			\frac{i\hbar}4\big([\op L_i,\op L_j]+[\op U_i,\op L_j]+[\op L_i,\op U_j]+[\op U_i,\op U_j]\big)=
			\frac{i\hbar}4\epsilon_{ijk}(\op L_k+\op U_k+\op U_k+\op L_k)=
			i\hbar\epsilon_{ijk}\op J_{1k}\\
		[\op J_{2i},\op J_{2j}]&=
			i\hbar\epsilon_{ijk}\op J_{2k}\\
		[\op J_{1i},\op J_{2j}]&=
			\frac{i\hbar}4\big([\op L_i,\op L_j]-[\op L_i,\op U_j]+[\op U_i,\op L_j]-[\op U_i,\op U_j]\big)=
			\frac{i\hbar}4\epsilon_{ijk}(\op L_k-\op U_k+\op U_k-\op L_k)=
			0.
	\end{aligned}
\end{equation}
Questo mostra che i tre $\op J_{1i}$ sono la base di un'algebra di Lie $\mathscr L_1$ isomorfa a $\mathfrak{su}(2)$ (si hanno le stesse costanti di struttura a meno di costanti moltiplicative), e lo stesso vale per i tre $\op J_{2i}$ che sono la base di $\mathscr L_2\cong\mathfrak{su}(2)$.
Poich\'e per ogni $X\in\mathscr L_1$ e $Y\in\mathscr L_2$ vale $[X,Y]=0$, le due algebre sono in somma diretta, perciò i sei operatori $\op{\vec J}_1$ e $\op{\vec J}_2$ sono la base di un'algebra isomorfa a $\mathfrak{su}(2)\oplus\mathfrak{su}(2)$.
A sua volta $\mathfrak{su}(2)\oplus\mathfrak{su}(2)$ è isomorfa a $\mathfrak{so}(4)$, l'algebra di Lie associata al gruppo $SO(4)$, che è quindi il gruppo di simmetria del sistema.
Abbiamo quindi trovato che $\op J_{1i}$ e $\op J_{2i}$ per $i=1,2,3$ sono sei costanti del moto che generano il gruppo delle rotazioni in quattro dimensioni.

Concludiamo dunque questa parte ricavando da questi operatori lo spettro dell'atomo.
Sfruttando le regole di commutazione, possiamo costruire anche qui degli operatori a scala che ``alzano e abbassano'' gli autovalori dei loro autostati.
Diamo qui solamente i risultati, omettendo la dimostrazione che è analoga a quella già svolta per il momento angolare nella sezione \ref{sec:autovalori-momento-angolare}.
Se ci troviamo in un autostato $\ket{\lambda,m_1,m_2}$ simultaneo per $\op{\vec J}_1^2$, $\op J_{13}$ e $\op J_{23}$, con
\begin{equation}
	\begin{aligned}
		\op{\vec J}_1^2\ket{\lambda,m_1,m_2}&=\lambda\hbar^2\ket{\lambda,m_1,m_2},\\
		\op J_{13}\ket{\lambda,m_1,m_2}		&=m_1\hbar\ket{\lambda,m_1,m_2},\\
		\op J_{23}\ket{\lambda,m_1,m_2}		&=m_2\hbar\ket{\lambda,m_1,m_2}
	\end{aligned}
\end{equation}
allora esiste un numero $j\in\big\{0,\frac12,1,\frac32,\dots\big\}$ tale che $m_1,m_2\in\{-j,-j+1,\dots,j\}$ e $\lambda=j(j+1)$.
Possiamo anche indicare questo autostato direttamente con $\ket{j,m_1,m_2}$.
La scelta di $j$ determina un insieme di $2j+1$ autostati linearmente indipendenti tutti con autovalore $j(j+1)\hbar^2$ per $\op{\vec J}_1^2$ (che è degenere).
Dato che $\scalar{\op{\vec L}}{\op{\vec U}}=\scalar{\op{\vec U}}{\op {\vec L}}=0$ (poich\'e $\op{\vec U}$ è proporzionale ad $\op{\vec A}$) si ha inoltre
\begin{multline}
	\op{\vec J}_1^2=\op{\vec J}_2^2=
	\frac14(\op{\vec L}^2+\op{\vec U}^2)=
	\frac14\bigg(\op{\vec L}^2-\frac{m}{2E}\op{\vec A}^2\bigg)=
	\frac14\bigg[\op{\vec L}^2-\frac{m}{2E}\bigg(\frac2{m}E(\op{\vec L}^2+\hbar^2)+\alpha^2\bigg)\bigg]=\\=
	-\frac14\bigg(\hbar^2+\frac{m\alpha^2}{2E}\bigg)=
	-\frac{\hbar^2}4\bigg(1+\frac{m\alpha^2}{2\hbar^2E}\bigg).
\end{multline}
L'energia dello stato $\ket{j,m_1,m_2}$ è dunque
\begin{equation}
	E=-\frac{m\alpha^2}{2\hbar^2(2j+1)^2}=-\frac{m\alpha^2}{2\hbar^2n^2}
\end{equation}
dove abbiamo posto $n\defeq 2j+1$.
Notiamo che per l'atomo di idrogeno abbiamo $\alpha=e^2$ (oppure $Ze^2$ nella sua generalizzazione con una qualsiasi carica puntiforme al centro) per cui abbiamo ritrovato esattamente la \eqref{eq:energia-idrogeno}.
Ricalcando i calcoli già fatti per il momento angolare, vediamo anche qui che per ogni $j$ si hanno proprio $(2j+1)^2=n^2$ autostati indipendenti con la stessa energia.

