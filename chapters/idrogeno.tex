\chapter{L'atomo di idrogeno}
In questo capitolo affrontiamo finalmente uno dei problemi ``classici'' completamente risolubili della meccanica quantistica, lo studio dello spettro dell'atomo di idrogeno.
Cominciamo dal problema base del problema a due corpi.

\section{Il problema a due corpi}
Affrontiamo per ora il problema nella visione della meccanica classica.
Prendiamo un sistema composto da due corpi, a cui assegnamo le coordinate canoniche $(\vec x_1,\vec p_1)$ e $(\vec x_2,\vec p_2)$, legati da un potenziale di interazione $V(\norm{\vec x_1-\vec x_2})$.
Il sistema è descritto dalla funzione hamiltoniana
\begin{equation}
	\mathcal H=\frac{\vec p_1^2}{2m_1}+\frac{\vec p_2^2}{2m_2}+V(\norm{\vec x_1-\vec x_2})
	\label{eq:hamiltoniana-due-corpi}
\end{equation}
ma è più conveniente passare con una trasformazione canonica al sistema del centro di massa, studiando cos\`i il moto del centro di massa e della distanza tra i due corpi
\begin{equation}
	\vec X\defeq\frac{m_1\vec x_1+m_2\vec x_2}{m_1+m_2}\qeq\vec x\defeq\vec x_1-\vec x_2.
	\label{eq:coordinate-moto-relativo}
\end{equation}
Gli impulsi coniugati risultano essere
\begin{equation}
	\vec P=\vec p_1+\vec p_2\qeq\vec p=\frac{m_2\vec p_1-m_1\vec p_2}{m_1+m_2}
	\label{eq:impulsi-moto-relativo}
\end{equation}
per cui introducendo la massa totale $M\defeq m_1+m_2$ e la massa ridotta $\mu\defeq\frac{m_1m_2}{m_1+m_2}$ abbiamo l'hamiltoniana
\begin{equation}
	\mathcal H=\frac{\vec P^2}{2M}+\frac{\vec p^2}{2\mu}+V(\norm{\vec x})
	\label{eq:hamiltoniana-moto-relativo}
\end{equation}
evidentemente scomponibile come somma di due hamiltoniane distinte, una per il centro di massa (nelle coordinate $\vec X,\vec P$) e una del moto relativo dei due corpi (nelle coordinate $\vec x,\vec p$), che commutano.
La prima delle due tra l'altro è l'hamiltoniana di una particella libera: essendo $\vec X$ ciclica l'impulso totale $\vec P$ si conserva.
Possiamo dunque studiare il problema separatamente, e d'ora in poi ci interessiamo soltanto al moto relativo.
Per quanto riguarda la massa ridotta, se uno dei due corpi ha massa molto minore possiamo trascurare quest'ultima: è il caso (che studieremo) del sistema protone-elettrone, in cui $m_e\ll m_p$, e potremo approssimare $\mu\approx m_p$.
Inoltre il potenziale $V(\norm{\vec x})$ è centrale, quindi il momento angolare si conserva e $\{H,\vec L\}=0$: possiamo quindi decomporre il termine $\vec p^2$ come $\vec p^2=p_r^2+\frac{L^2}{r^2}$ ottenendo l'hamiltoniana
\begin{equation}
	\mathcal H=\frac{p_r^2}{2\mu}+\frac{L^2}{2\mu r^2}+V(r)
	\label{eq:hamiltoniana-radiale}
\end{equation}
in cui entrambe le variabili angolari sono cicliche.
Possiamo ulteriormente semplificare la funzione accorpando il termine $\frac{L^2}{2\mu r^2}$, che dipende solo dalla posizione, nel potenziale, formando cos\`i il \emph{potenziale efficace} $V_\textup{eff}(r)$.
Ci siamo ridotti quindi a un problema unidimensionale nella variabile $r$, con l'unico accorgimento che essa non varia in $\R$ ma in $[0,+\infty)$.

Applicando quando ricavato al caso quantistico, in dimensione $d=3$ l'operatore $\op{\vec p}^2$ è scritto in rappresentazione di Schr\"odinger delle coordinate, secondo la \eqref{eq:laplaciano-momento-angolare}, come
\begin{equation}
	-\hbar^2\frac1{r^2}\drp{}{r}r^2\drp{}{r}=-\frac{\hbar^2}{r^2}\bigg(2r\drp{}{r}+r^2\ddrp{}{r}\bigg)=-\hbar^2\bigg(\frac2{r}\drp{}{r}+\ddrp{}{r}\bigg).
\end{equation}
Con l'uguaglianza $p_r^2=\frac1{r}p_r^2r$ possiamo infine riscrivere l'equazione di Schr\"odinger in questa rappresentazione come
\begin{equation}
	-\frac{\hbar^2}{2\mu}\frac1{r}\ddrp{}{r}\big(r\psi(r,\phi,\theta)\big)+\frac{\hbar^2l(l+1)}{r^2}\psi(r,\phi,\theta)=E\psi(r,\phi,\theta).
	\label{eq:schrodinger-moto-relativo}
\end{equation}
La ciclicità delle variabili angolari ci permette di fattorizzare la funzione d'onda in una parte radiale $R(r)$ e una angolare $Y(\phi,\theta)$.
Sappiamo dal capitolo precedente che la parte radiale $R$ soddisfa un'equazione dipendente dall'energia $E$ del sistema e dal numero quantico $l$ associato all'autovalore di $\op L^2$, mentre la parte angolare $Y$ soddisfa due equazioni dipendenti dall'autovalore $m$ di $\op L_3$ e da $l$.
Identifichiamo dunque le soluzioni accettabili della \eqref{eq:schrodinger-moto-relativo} con questi tre numeri, scrivendo
\begin{equation}
	\psi(r,\phi,\theta)=\psi_{E,l,m}(r,\phi,\theta)=R_{E,l}(r)Y_{l,m}(\phi,\theta).
	\label{eq:fattorizzazione-wf-radiale-angolare}
\end{equation}
Nell'equazione \eqref{eq:schrodinger-moto-relativo} possiamo quindi semplificare il fattore $Y_{l,m}$ ottenendo un'equazione solo per $R$.
Notiamo inoltre che la derivata seconda in $r$ agisce non su $R(r)$ soltanto, ma su $rR(r)$: moltiplicando i due membri a sinistra per $r$ (che commuta con i restanti termini) abbiamo dunque un'equazione differenziale per $rR(r)$, ossia
\begin{equation}
	-\frac{\hbar^2}{2\mu}\ddrp{}{r}\big(rR(r))+\frac{\hbar^2l(l+1)}{r^2}rR(r)=ErR(r).
\end{equation}
Appare naturale a questo punto definire $u(r)\defeq rR(r)$, ottenendo l'equazione
\begin{equation}
	-\frac{\hbar^2}{2\mu}u''(r)+V_\textup{eff}(r,l)u(r)=Eu(r),\hspace{1cm}\text{con }V_\textup{eff}(r,l)=\frac{\hbar^2l(l+1)}{r^2}+V(r).
	\label{eq:schrodinger-moto-relativo-potenziale-efficace}
\end{equation}
Questa definizione non semplifica solo l'equazione di Schr\"odinger, ma anche la condizione di normalizzazione: supponendo di aver normalizzato la parte angolare con
\begin{equation}
	\int_0^{2\pi}\int_0^\pi\abs{Y_{l,m}(\phi,\theta)}^2\sin\theta\,\dd\phi\,\dd\theta=1,
	\label{eq:normalizzazione-parte-angolare}
\end{equation}
la normalizzazione per la funzione d'onda diventa in questo modo
\begin{multline}
	1=\int_{\R^3}\abs{\psi(\vec x)}^2\,\dd^3x=\int_0^{+\infty}\int_0^{2\pi}\int_0^\pi\abs{R_{E,l}(r)Y_{l,m}(\phi,\theta)}^2r^2\sin\theta\,\dd\phi\,\dd\theta\,\dd r=\\
	=\int_0^{+\infty}r^2\abs{R_{E,l}(r)}^2\,\dd r=\int_0^{+\infty}\abs{u_{E,l}(r)}^2\,\dd r
	\label{eq:normalizzazione-parte-radiale}
\end{multline}
che tanto somiglia al caso unidimensionale.
Non c'è alcuna ragione per cui $R$ debba essere nulla per $r=0$, ma perlomeno deve essere continua e limitata quindi necessariamente si deve avere
\begin{equation}
	0=\lim_{r\to 0^+}rR(r)=\lim_{r\to 0^+}u(r)
\end{equation}
che è un'utile condizione al contorno per la funzione $u$.

\section{L'atomo di idrogeno}
Per l'atomo di idrogeno abbiamo l'energia potenziale elettrostatica $V(r)=-\frac{e^2}{r}$, posseduta da un elettrone in orbita attorno ad un protone.
Più in generale, possiamo considerare il moto dell'elettrone in un campo elettrostatico centrale, generato da una carica puntiforme $Ze$ nell'origine, ossia con il potenziale
\begin{equation}
	V(r)=-\frac{Ze^2}{r}.
	\label{eq:potenziale-elettrostatico}
\end{equation}
Il sistema è dunque caratterizzato (classicamente) dalla funzione hamiltoniana
\begin{equation}
	\mathcal H=\frac{\vec p^2}{2\mu}-\frac{Ze^2}{r}
	\label{eq:hamiltoniana-idrogeno}
\end{equation}
e il corrispondente operatore si scrive, nella rappresentazione di Schr\"odinger della posizione (in coordinate sferiche), diagonalizzando contemporaneamente $\op L^2$ e $\op L_3$ come
\begin{equation}
	-\frac{\hbar^2}{2\mu}\frac1{r}\ddrp{}{r}r+\frac{\hbar^2l(l+1)}{2\mu r^2}-\frac{Ze^2}{r}.
	\label{eq:hamiltoniano-idrogeno-schroedinger}
\end{equation}
Abbiamo già visto che possiamo approssimare la massa ridotta con la massa del protone, o in generale della carica puntiforme al centro dell'orbita, per cui d'ora in poi scriveremo $m$ al posto di $\mu$ intendendo questa approssimazione.
Vale quanto detto nella sezione precedente per la fattorizzazione della funzione d'onda, l'equazione di Schr\"odinger per $u(r)=rR(r)$ e le condizioni di normalizzazione.

Il potenziale efficace, dato dall'equazione \eqref{eq:schrodinger-moto-relativo-potenziale-efficace}, tende a zero per $r\to+\infty$, a $+\infty$ per $r\to 0^+$ e ammette un minimo assoluto negativo.
Se dunque $E<0$ l'hamiltoniano ammetterà uno spettro discreto, e si trovano degli stati legati; se invece $E>0$ avrà uno spettro continuo, ma non degenere, con stati non legati.
Esaminiamo il comportamento asintotico delle soluzioni: per $r\to+\infty$ il potenziale diventa trascurabile rispetto a qualunque valore di $E$ che non sia zero, perciò approssimiamo l'equazione di Schr\"odinger come
\begin{equation}
	u''(r)=\frac{2m\abs{E}}{\hbar^2}u(r)=\beta^2u(r)
	\label{eq:schroedinger-idrogeno-approx-inf}
\end{equation}
in cui $\beta\defeq\frac{\sqrt{2m\abs{E}}}{\hbar}$.
La soluzione generale dell'equazione è $u(r)=c_1e^{-\beta r}+c_2e^{\beta r}$ in cui chiaramente dobbiamo porre $c_2=0$.
Per $r\to 0^+$, invece, sono i termini $E+\frac{Ze^2}{r}$ ad essere trascurabili rispetto al termine proporzionale a $1/r^2$, quindi approssimiamo l'equazione come
\begin{equation}
	-\frac{\hbar^2}{2m}u''(r)+\frac{\hbar^2l(l+1)}{2mr^2}u(r)=0\qqq u''(r)-\frac{l(l+1)}{r^2}u(r)=0.
	\label{eq:schroedinger-idrogeno-approx-0}
\end{equation}
Ipotizziamo in questo caso che $u(r)\sim r^s$: otteniamo $u''(r)=s(s-1)r^{s-2}$ da cui l'equazione
\begin{equation}
	s(s-1)r^{s-2}-\frac{l(l+1)}{r^2}r^s=0\qqq s(s-1)=l(l+1)
\end{equation}
da cui $s=-l$ oppure $s=l+1$.
Escludendo $s=-l$ poich\'e deve risultare $u(0)=0$ (e $l$ non è negativo), otteniamo che asintoticamente per $r\to 0^+$ si ha $u(r)\sim r^{l+1}$, ossia $R(r)\sim r^l$: la probabilità di misurare la posizione (della particella in orbita) vicino al nucleo è tanto più piccola maggiore è il momento angolare orbitale del sistema.

Sapendo che la funzione $u(r)$ decade all'infinito come $e^{-\beta r}$, proviamo a risolvere l'equazione di Schr\"odinger con una funzione $u(r)=f(r)e^{-\beta r}$.
Troviamo, poich\'e $E=-\beta^2\frac{\hbar^2}{2m}$,
\begin{equation}
	\begin{gathered}
		-\frac{\hbar^2}{2m}\big[f''(r)-2\beta f'(r)+\beta^2 f(r)\big]e^{-\beta r}+\frac{\hbar^2l(l+1)}{2mr^2}f(r)e^{-\beta r}-\frac{Ze^2}{r}f(r)e^{-\beta r}+\beta^2\frac{\hbar^2}{2m}f(r)e^{-\beta r}=0\\
		-\frac{\hbar^2}{2m}\big[f''(r)-2\beta f'(r)\big]+\bigg[\frac{\hbar^2l(l+1)}{2mr^2}-\frac{Ze^2}{r}\bigg]f(r)=0\\
		f''(r)-2\beta f'(r)-\frac{l(l+1)}{r^2}f(r)+\frac{2mZe^2}{\hbar^2r}f(r)=0.
	\end{gathered}
\end{equation}
Sviluppiamo ora $f$ in serie di potenze di $r$: tenendo conto che per $r\to 0^+$ si ha $u(r)\sim r^{l+1}$, scriviamo $f(r)=r^{l+1}\sum_{k=0}^{+\infty}a_kr^k$, supponendo di conseguenza anche che $a_0\ne 0$ (possiamo fissare $a_0=1$).
Inserendo lo sviluppo nell'equazione trovata otteniamo
\begin{equation}
	\begin{gathered}
		\sum_{k=0}^{+\infty}a_k\bigg[(k+l+1)(k+l)r^{k+l-1}-2\beta(k+l+1)r^{k+l}-l(l+1)r^{k+l-1}+\frac{2mZe^2}{\hbar^2}r^{k+l}\bigg]=0\\
		\sum_{k=0}^{+\infty}a_k\big[(k+l+1)(k+l)-l(l+1)\big]r^{k+l-1}+\sum_{k=0}^{+\infty}a_k\bigg[\frac{2mZe^2}{\hbar^2}-2\beta(k+l+1)\bigg]r^{k+l}=0
	\end{gathered}
\end{equation}
e traslando gli indici della prima somma ($k\mapsto k+1$) troviamo
\begin{equation}
	\sum_{k=0}^{+\infty}\bigg\{a_{k+1}\big[(k+l+2)(k+l+1)-l(l+1)\big]-a_k\bigg[2\beta(k+l+1)-\frac{2mZe^2}{\hbar^2}\bigg]\bigg\}r^{k+l}=0
\end{equation}
da cui la relazione di ricorrenza
\begin{equation}
	a_{k+1}=\frac{2\beta(k+l+1)-\frac{2mZe^2}{\hbar^2}}{(k+l+2)(k+l+1)-l(l+1)}a_k,\hspace{1cm}\text{con }a_0=1.
\end{equation}
Per grandi valori di $k$ possiamo approssimarla con
\begin{equation}
	a_{k+1}\sim\frac{2\beta(k+l+1)}{(k+l+2)(k+l+1)}a_k\sim\frac{2\beta}{k}a_k
\end{equation}
e iterando fino a $k=0$ (ossia sostituendo $a_k$ per ricorrenza con $\frac{2\beta}{k-1}a_{k-1}$ e cos\`i via) otteniamo
\begin{equation}
	a_k\sim\frac{(2\beta)^k}{k!}
\end{equation}
La serie di potenze che ne risulta per $f(r)$ converge, però il limite è $e^{2\beta r}$ da cui segue $u(r)\sim e^{\beta r}$ per $r\to+\infty$, un risultato chiaramente non accettabile.
Deve esistere quindi un certo indice $\bar{k}\in\N$ tale per cui $a_{\bar{k}}=0$, da cui $a_k=0$ $\forall k\ge\bar{k}$ in modo da troncare la serie.
Con questa ipotesi la $f(r)$ è dunque un polinomio di grado $\bar{k}+l-1$, inoltre risulta
\begin{equation}
	\beta(\bar{k}+l+1)=\frac{mZe^2}{\hbar^2}\qqq\frac{\sqrt{2m\abs{E}}}{\hbar}(\bar{k}+l+1)=\frac{mZe^2}{\hbar^2}
\end{equation}
e definendo $n\defeq\bar{k}+l+1$ abbiamo
\begin{equation}
	\sqrt{2m\abs{E}}=\frac{mZe^2}{n\hbar}\qqq E=-\frac{Z^2me^4}{2n^2\hbar^2}
\end{equation}
che mostra la quantizzazione dei livelli energetici degli stati legati (ricordando che in questo caso $E<0$).
Notiamo che i livelli non sono equidistanti, ma la separazione diminuisce più $E$ si avvicina a zero.
Il numero $n$ è chiamato \emph{numero quantico principale} e con esso (al posto di $E$) etichettiamo le funzioni d'onda, che scriveremo come $\psi_{n,l,m}$ con $n\in\N$, $l\in\N_0$ e $m\in[-l,l]\subset\Z$.
Notiamo inoltre che la grandezza $\frac{\hbar^2}{me^2}$ ha le dimensioni di una lunghezza: non a caso è detta \emph{raggio di Bohr} (indicato con $r_B$).
Con esso scriviamo la funzione d'onda dello stato $\ket{n,l,m}$
\begin{equation}
	\psi_{n,l,m}(r,\phi,\theta)=e^{-Zr/nr_B}r^lY_{l,m}(\phi,\theta)\sum_{j=0}^{\bar{k}}a_jr^j.
	\label{eq:wf-idrogeno}
\end{equation}
Vediamo quindi, dal fattore $\exp\big(\frac{Z}{n}\frac{r}{r_B}\big)$, che all'aumentare di $Z$ la funzione d'onda è ``più grande'' vicino al nucleo, mentre per $n$ grande tende ad ``allontanarsi''.

Lo stato fondamentale $\ket{1,0,0}$ (se $n=1$ allora $\bar{k}=l=0$, non potendo essere negativi) è caratterizzato dalla funzione d'onda $\psi_{1,0,0}(r,\phi,\theta)=e^{Zr/r_B}$: il raggio di Bohr fornisce una ``scala'' per le dimensioni del sistema.
Per $n=2$ abbiamo $\bar{k}+l=1$, da cui le possibili opzioni $l=0,\bar{k}=1$ oppure $l=1,\bar{k}=0$.
Lo stato con $l=0$ ha necessariamente anche $m=0$, dunque corrisponde a $\ket{2,0,0}$.
Se invece $l=1$, possiamo avere $m=\pm 1$: in totale per $l=1$ troviamo dunque tre stati indipendenti, caratterizzati dalle tre armoniche sferiche $Y_{1,-1}$, $Y_{1,0}$ e $Y_{1,1}$.
Per $n=3$ troviamo nove stati, e cos\`i via: ad ogni livello energetico corrispondono più stati indipendenti, con momento angolare differente.
Il livello $n$-esimo è quindi degenere, e ammette in totale
\begin{equation}
	\sum_{l=0}^{n-1}(2l+1)=2\sum_{l=0}^{n-1}l+\sum_{l=0}^{n-1}1=2\frac{n(n-1)}2+n=n^2
\end{equation}
stati.
Essendo il sistema a simmetria centrale, in effetti, tutte le componenti del momento angolare commutano con l'hamiltoniano, ma non tra di loro; questo, per il teorema \ref{t:degenerazione}, è indice della degenerazione dell'hamiltoniano.

